%% LyX 1.6.5 created this file.  For more info, see http://www.lyx.org/.
%% Do not edit unless you really know what you are doing.
\documentclass[english,compress]{beamer}
\usepackage[T1]{fontenc}
\usepackage[latin9]{inputenc}
\setcounter{secnumdepth}{3}
\setcounter{tocdepth}{3}
\setbeamertemplate{navigation symbols}{}

\setbeamercolor{frametitle}{fg=black,bg=white}
\setbeamercolor{title}{fg=black,bg=yellow!85!orange}
\usetheme{AnnArbor}
%\usetheme{Warsaw}
%\usetheme{Szeged}
%\usetheme{CambridgeUS}
%\useinnertheme{umbctribullets}
\setbeamertemplate{blocks}[rounded][shadow=true]
%\useinnertheme{rounded}
%\usecolortheme{albatross}

\makeatletter
%%%%%%%%%%%%%%%%%%%%%%%%%%%%%% Textclass specific LaTeX commands.
 % this default might be overridden by plain title style
\newcommand\makebeamertitle{\frame{\maketitle}}
\DeclareMathOperator{\LP}{\mathbb{L}}
% 
% \AtBeginDocument{
 %  \let\origtableofcontents=\tableofcontents
 %  \def\tableofcontents{\@ifnextchar[{\origtableofcontents}{\gobbletableofcontents}}
 %  \def\gobbletableofcontents#1{\origtableofcontents}
 %}

\makeatother

\usepackage{babel}
\institute[M.Sc Student]{M.Sc Seminar \\ Faculty of Science, University of Malta}
\title[Compactness and Connectedness]{Compactness and Connectedness in Lexicographic Products of LOTS}
\author[Lin Li]{Lin Li}
\date{May 12, 2010}
\begin{document}

\begin{frame}
\titlepage
\end{frame}

\begin{frame}{Contents}
\tableofcontents
\end{frame}

\section{Linear Order}
\begin{frame}{Linearly Ordered Sets}
  \begin{definition}<+->
    A linearly ordered set $(X, <)$ is a pair, where $X$ is a set and $<$ is a subset of $X \times X$ such that:
    \begin{itemize}
    \item $\forall x \in X$, $(x,x) \notin <$ $\qquad \qquad \qquad \qquad \qquad \qquad \qquad \! \!$ (Non-reflexive)
    \item $x < y$ and $y < z$ $\Rightarrow$ $x < z$ $\qquad \qquad \qquad \qquad \qquad \;$ (Transitive)
    \item $\forall x, y \in X$, we have $x <y$ or $y <x$ or $x = y$ $\qquad  \;  \;$ (Totality) 
    \end{itemize}
  \end{definition}
  \begin{definition}<+->
    A subset $K$ of a linearly ordered set $(X, <)$ is called convex if $\forall a, b \in K, \; [a,b] \subseteq K$
  \end{definition}
  \begin{definition}<+->
    Say $U \subseteq X$, and $C$ is a convex subet of $U$. Then $C$ is called a convex component of $U$ if for all convex subsets $K$ of $U$,
    we have $K \cap C \neq \phi \Rightarrow K \subseteq C$.
  \end{definition}
\end{frame}

\section{LOTS and GO-space}
\subsection{Introduction}
\begin{frame}{LOTS and GO-spaces}
  \begin{definition}<+->
    Let $(X,<)$ be a linearly ordered set, the topology $\lambda (<)$ of $X$ generated by the subbase $(\leftarrow, a)$, $(b, \rightarrow)$, $\forall a,b \in X$ is called the linear order topology. And the pair $(X,\lambda (<))$ is called a LOTS.
  \end{definition}
  \begin{definition}<+->
    Let $(X,<)$ be a linearly ordered set, $\tau$ is a generalized order topology on $X$ if it has a convex base and it is stronger than the linear order topology.
  \end{definition}
  \begin{lemma}<+->
  \begin{itemize}
  \item A GO-space is a closed subspace of a LOTS.
  \item A GO-space is a dense subspace of a LOTS.
  \end{itemize}
  \end{lemma}
\end{frame}

\begin{frame}
  \frametitle{GO-space as a closed subspace of a LOTS}
  \begin{theorem}
    Let $(X,<,\tau)$ be a GO-space, $X^{*}$ a subspace of the LOTS $X \times \mathbb{Z}$ defined as
    \begin{align*}
      X^{*} =& \; X \times \{0\} \\
      &\cup \{(x,n) \, | \, [x, \rightarrow[ \in \tau \backslash \lambda (<), \, n < 0 \} \\
      &\cup \{(x,n) \, | \, ] \leftarrow, x] \in \tau \backslash \lambda (<), \, n > 0 \}
    \end{align*}
    Then $(X, <, \tau)$ is homeomorphic to the closed subspace $X \times \{0\}$ of $X^{*}$.
  \end{theorem}
\end{frame}

\subsection{Properties of GO-space}
\begin{frame}
  \frametitle{Properties of GO-Spaces}

  \begin{definition}<+->
    A $T_{1}$ space $X$ is called collectionwise normal if for any discrete collection $\{F_{\alpha}\}_{\alpha \in \mathcal{A}}$ of closed subsets $F_{\alpha}$ of $X$, there exists a discrete collection $\{U_{\alpha}\}_{\alpha \in \mathcal{A}}$ of open subsets such that $F_{\alpha} \subseteq U_{\alpha}$. 
  \end{definition}
  \begin{theorem}<+->
    A GO-space is hereditarily collectionwise normal.
  \end{theorem}
\end{frame}



\subsection{Gaps and Pseudo gaps}
\begin{frame}
  \frametitle{Gaps and Pseudo gaps in GO-spaces}
  \begin{definition}<+->
    Let $X$ be a GO-space, $A$ and $B$ open disjoint subsets of $X$. The ordered pair $(A,B)$ is a gap in $X$ if
    \begin{itemize}
    \item $\forall a \in A$ and $\forall b \in B$, $a < b$.
    \item $A$ has no maximal point and $B$ has no minimal point.
    \end{itemize}
    If $A = \phi$, then $(A,B)$ is called a left end gap, and if $B = \phi$, then it is called a right end gap. 
  \end{definition}
  \begin{definition}<+->
     Let $X$ be a GO-space, $A$ and $B$ non-empty open disjoint subsets of $X$, then the ordered pair $(A,B)$ is called a pseudo gap if
     \begin{itemize}
     \item $\forall a \in A$ and $\forall b \in B$, $a < b$.
     \item either $A$ has the maximal point and $B$ has no minimal point, or $A$ has no maximal and $B$ has the minimal.
     \end{itemize}
  \end{definition}
\end{frame}

\subsection{Compactness}
\begin{frame}
  \frametitle{Compactness of GO-spaces}
  \begin{theorem}<+->
    A GO-space $X$ is compact iff it has no gaps or pseudo gaps.
  \end{theorem}
  \begin{proof}<+->
   $(\Rightarrow)$  Say $X$ is compact and has a gap $(A,B)$. Since $A$ has no maximal point, then for any $a \in A$, $\exists \, a' \in A$ such that $a < a'$. Then for every $a \in A$ we take $]\leftarrow,a'[$, Then the open cover \{ $]\leftarrow, a'[$ | $\forall a \in A$ \} $\cup$ \{$B$ \} of $X$ has no finite subcover. Contradiction.
    \pause
    
    $(\Leftarrow)$ Suppose $X$ has no gaps or pseudo gaps, then $X$ has the minimal point $a$ and the maximal point $b$. Now, let $\mathcal{U}$ be an open cover consisting of convex subsets. Define $Y = \{$ $x \in X$ | there exists finitely many $U_{1}$, $U_{2}$, $\ldots$ $U_{k}$ such that $a \in U_{1}$, $x \in U_{k}$ and  $U_{i} \cup U_{i+1}$ is convex  for $i \leq k$ \}. If $Y \neq X$, then the pair $(Y, X \backslash Y)$ forms a gap in $X$. Contradiction.
   
 \end{proof}
\end{frame}

\subsection{Connectedness}
\begin{frame}
  \frametitle{Connectedness of GO-Spaces}
  \begin{definition}<+->
    Let $X$ be a GO-space, $A,B \subseteq X$ be open and disjoint. Then the ordered pair $(A,B)$ is a jump if
    \begin{itemize}
    \item $\forall a \, \in A$ and $\forall b \in B$, we have $a < b$
    \item $A$ has the maximal point and $B$ has the minimal point
    \end{itemize}
  \end{definition}
  \begin{theorem}<+->
    A GO-space is connected iff it has no jumps, pseudo gaps or gaps except for possible end gaps.
  \end{theorem}
\end{frame}

\section{Lexicographic Products}
\subsection{Introduction}
\begin{frame}
  \frametitle{Lexicographic products}
  \begin{definition}<1->
    Let $(X,<_{X})$, $(Y,<_{Y})$ be linearly ordered sets, the lexicographic product $X \cdot Y$ is defined as the Cartesian product supplied with the lexicographic order $<$ defined by $(x_{1},y_{1}) < (x_{2}, y_{2})$ if either $x_{1} < x_{2}$ or $x_{1} = x_{2}\, , \, y_{1} < y_{2}$. And the topology is defined as the linear topology with respect to the lexicographic order.
  \end{definition}
  \begin{definition}<2->
    Let $(X_{\alpha}, <_{\alpha})$ be linearly ordered sets $\forall \alpha < \mu$, where $\mu$ is a limit ordinal. Then the topology on the lexicographic product $X = \LP_{\alpha < \mu}X_{\alpha}$ is defined as the linear topology with respect to the lexicographic order.
  \end{definition}
\end{frame}

\subsection{Useful Lemma}
\begin{frame}
  \frametitle{Dedekind compactification}
  \begin{definition}<+->
    Let $X$ be a GO-space, for each gap $(A, B)$ in $X$, we define a new point $c_{(A,B)} = (A,B)$. Then define $X^{+} = X \, \cup$ \{ $c_{(A,B)}$ | $c_{(A,B)}$ is a gap in $X$ \} and a linear order $<'$ such that:
    \begin{itemize}
    \item $\forall x,y \in X$, $x <' y$ if $ x < y$
    \item $x <' c_{(A,B)}$ if $x \in A$
    \item $c_{(A,B)} <' x$ if $x \in B$.
    \end{itemize}
  \end{definition}
   \begin{lemma}<+->
    Let $X_{\alpha}$ be LOTS  $\forall \alpha < \mu$, where $\mu$ is a limit ordinal, $X = \LP_{\alpha < \mu} X_{\alpha}$ the lexicographic product. Let $c$ be a gap in $X$, then there exists a gap $c_{\alpha_{0}}$ in $X_{\alpha_{0}}$ such that $c = (c_{\alpha})_{\alpha < \mu}$ where $c_{\alpha} = x_{\alpha}$ for $\alpha \neq \alpha_{0}$, and $x_{\alpha} \in X_{\alpha}$. 
  \end{lemma}
\end{frame}

\subsection{Compactness}
\begin{frame}
  \frametitle{Compactness of Lexicographic products of LOTS}
  \begin{theorem}<+->
   Let $X = \LP_{\alpha < \mu} X_{\alpha}$ be the lexicographic product, where $\mu$ is a limit ordinal and $X_{\alpha}$ are LOTS $\forall \alpha < \mu$. Then $X$ is compact iff $X_{\alpha}$ is compact for all $\alpha < \mu$.
 \end{theorem}
 \begin{theorem}<+->
   Let $X = \LP_{\alpha < \mu} X_{\alpha} $ be the lexicographic product,  where $\mu$ is a limit ordinal and $X_{\alpha}$ are LOTS containing more than one element $\forall \alpha < \mu$. If $\mu$ is not cofinal with $\omega_{0}$, then no points in $X$ has countable local base.
 \end{theorem}
\end{frame}

\subsection{Connectedness}
\begin{frame}[t]
   \frametitle{Connectedness of Lexicographic products}
  \begin{theorem}
    The lexicographic product $X = \LP_{\alpha < \mu}X_{\alpha}$ is connected iff one of the following collections of condition is satisfied:
    \only<1>{
    \begin{enumerate}
    \item $\forall \alpha > 0$, $X_{\alpha}$ has a left endpoint.
    \item $\forall \alpha \geq \omega_{0}$, $X_{\alpha}$ has both a left endpoint and a right end point.
    \item $\forall \alpha < \mu$, $X_{\alpha}$ has no gaps, except for possible end gaps.
    \item $\forall \alpha < \mu$, if $X_{\beta}$ have both a left and right endpoints $\forall \beta > \alpha$, then $X_{\beta}$ has no jumps.
    \item $\forall \alpha < \mu$, if $X_{\beta}$ does not have a right endpoint for some $\beta > \alpha$, then each bounded strictly decreasing sequence in $X_{\alpha}$ is finite
    \end{enumerate}
  }
  \only<2>{
  \begin{enumerate}
    \item $\forall \alpha > 0$, $X_{\alpha}$ has a right endpoint.
    \item $\forall \alpha \geq \omega_{0}$, $X_{\alpha}$ has both a left endpoint and a right end point.
    \item $\forall \alpha < \mu$, $X_{\alpha}$ has no gaps, except for possible end gaps.
    \item $\forall \alpha < \mu$, if $X_{\beta}$ have both a left and right endpoints $\forall \beta > \alpha$, then $X_{\beta}$ has no jumps.
    \item $\forall \alpha < \mu$, if $X_{\beta}$ does not have a left endpoint for some $\beta > \alpha$, then each bounded strictly increasing sequence in $X_{\alpha}$ is finite
    \end{enumerate}
    }
  \end{theorem}
\end{frame}

\section*{The End}
\begin{frame}
\begin{center}
\setlength{\fboxrule}{2pt}
\fcolorbox{black!15!red}{white}{\Huge{THANK YOU!}}
\end{center}
\end{frame}


\end{document}
