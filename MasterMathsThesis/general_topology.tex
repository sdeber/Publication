\documentclass[12pt,oneside,english]{amsbook}
\usepackage[T1]{fontenc}
\usepackage[latin9]{inputenc}
\usepackage{amsmath}
\usepackage{amsthm}
\usepackage{amssymb}
\usepackage{setspace}
\onehalfspacing

\makeatletter
%%%%%%%%%%%%%%%%%%%%%%%%%%%%%% Textclass specific LaTeX commands.
\numberwithin{equation}{section} %% Comment out for sequentially-numbered
\numberwithin{figure}{section} %% Comment out for sequentially-numbered
\theoremstyle{plain}
\numberwithin{section}{chapter}

\newtheorem{thm}{Theorem}[section]
\theoremstyle{plain}
\newtheorem{lem}[thm]{Lemma}
\newtheorem{eg}[thm]{Example}
\newtheorem{corollary}[thm]{Corollary}
\newtheorem{defn}[thm]{Definition}
\newtheorem{remark}[thm]{Remark}
\DeclareMathOperator{\intersect}{\bigcap}
\DeclareMathOperator{\union}{\bigcup}
\DeclareMathOperator{\LP}{\mathbb{L}}
% \pagenumbering{alph}
\setcounter{secnumdepth}{4}
\makeatother
\setcounter{tocdepth}{2}

\usepackage{babel}


\begin{document}
\thispagestyle{empty}
\newpage
\begin{center}
\begin{huge}
\textbf{Lexicographic Products \\ \vspace{-0.1in} On \\ \vspace{0.1in}Linearly Ordered Topological Spaces}
\end{huge}
\end{center}

\vspace{2in}
\begin{center}
  \begin{Large} October 2010 \end{Large}
\end{center}
\vspace{2in}

\begin{center}
\begin{Large} Lin Li \end{Large}
\end{center}
\begin{center} 
\begin{Large}Mathematics Department %\\ Faculty of Science \\ University of Malta \\ (Master Course)
\end{Large}
\end{center}
\begin{center} 
\begin{Large}Faculty of Science
\end{Large}
\end{center}
\begin{center} 
\begin{Large}University of Malta
\end{Large}
\end{center}
\begin{center} 
\begin{Large}(Master Course)
\end{Large}
\end{center}
\vspace{2in}


\newpage
\thispagestyle{empty}
I would like to express my gratitude to my supervisor Professor David Buhagiar who guided me to the final goal. He has been my lecturer since the very first year of my study in University of Malta. It was his knowledge and patience that enabled me to complete this study successfully. I also would like to give a great thank to my family, they have been always supporting me in my study and career.

\newpage
\setcounter{page}{1}
\tableofcontents{}


\chapter*{Preface}
  A linear order $<$ defined on a set $X$ is a relation such that:
  \begin{itemize}
  \item For all $x \in X$, $x < x$ does not hold.  $\qquad \! \!$ (Non-reflexivity)
  \item $x < y$ and $y < z$ $\Rightarrow$ $x < z$ $\qquad \qquad \qquad \qquad \;$ (Transitivity)
  \item For all $x, y \in X$, we have $x <y$ or $y <x$ or $x = y$ $  \;  \;$ (Totality) 
  \end{itemize}
  The linear order topology defined on $X$ is the topology generated by the family of all open intervals in $X$, and the obtained topological space is called a linearly ordered topological space (LOTS). A subspace of a LOTS is called a generalized ordered space (or GO-space). Clearly, a LOTS is also a GO-space. In this thesis, we study several well-known properties like compactness, paracompactness, etc in GO-spaces. In addition, we will study lexicographic products of LOTS.
  
  In Chapter 1, we provide some utility results for the study of properties of GO-spaces. The most important results are those about gaps and pseudo gaps in GO-spaces. As we will see, in the following chapters, gaps and pseudo gaps are intensively used to characterize different properties of GO-spaces. In addition, gaps also play a critical role in our study of lexicographic products of LOTS. Since gaps can be considered to be points in the Dedekind compactification of a LOTS or GO-space, our study about gaps are mainly in the context of Dedekind compactification. In Theorem \ref{lem:lgap:1}, we show that a gap in the lexicographic product must be generated from a gap in one of its factor spaces. This result is quite intuitive, but the proof is not so simple. Then in Lemma \ref{lem:lgap:2} and Lemma \ref{lem:lgap:3}, we discuss circumstances, in which a gap in a factor space will give a gap in the lexicographic product. By using these results, we can greatly simplify the proofs of compactness, paracompactness and connectedness of lexicographic products given in Chapter 7. In addition to the aforementioned properties, we also discuss the property of countable compactness in lexicographic products in Chapter 7. As we can see, the results about gaps given in Chapter 1 offer great help to the study. 

  In Chapter 2, we discuss compactness of general topological spaces and give a discussion about compactness of GO-spaces. In Chapter 3 and Chapter 4, we discuss the Lindel\"{o}f property and countable compactness of GO-spaces respectively. During the study of paracompactness in Chapter 5, in addition to the characterization of paracompactness by using open covers, we find that R.Engelking and D. Lutzer gave a very interesting way to characterize paracompactness of GO-spaces by using stationary sets. In Chapter 6, we discuss connectedness of GO-spaces as well as several types of disconnectedness, namely zero dimensionality, strongly zero dimensionality, total disconnectedness and hereditarily disconnectedness. As we can see, these four types of disconnectedness are equivalent in GO-spaces.

  In Chapter 7, we concentrate on the study of lexicographic products. We discuss connectedness, compactness, paracompactness and countable compactness. Gaps are intensively used to characterize these properties in lexicographic products. Referring to countable compactness, we give two theorems, Theorem \ref{lp:thm:6} gives a result about the countable compactness of a lexicographic product when all of its factor spaces are countably compact and none of them have end gaps. Theorem \ref{lp:thm:7} characterizes the countable compactness of the lexicographic product of two countably compact spaces who may have end gaps in them. We also give some counterexamples to show that Theorem \ref{lp:thm:6} does not hold if one of the factor spaces has end gaps.

  Theorem \ref{lem:lgap:1}, Lemma \ref{lem:lgap:2} and Lemma \ref{lem:lgap:3} in Chapter 1 and all of Section 7.3 in Chapter 7 are the results of the author.


\chapter{Introduction to LOTS and GO-spaces}
\section{Preliminaries}
\begin{defn}
  A linearly ordered set $(X, <)$ is a pair, where $X$ is a set and $<$ is a binary operator defined on $X$ such that:
  \begin{itemize}
  \item For all $x \in X$, $x < x$ does not hold.  $\qquad \! \!$ (Non-reflexive)
  \item $x < y$ and $y < z$ $\Rightarrow$ $x < z$ $\qquad \qquad \qquad \qquad \;$ (Transitive)
  \item For all $x, y \in X$, we have $x <y$ or $y <x$ or $x = y$ $  \;  \;$ (Total) 
  \end{itemize}
\end{defn}
Two points $x,y$ with $x < y$ in a linearly ordered set $(X,<)$ are called neighbours if there exist no points $z$ in $X$ such that $x < z < y$, $x$ is called the left neighbour point of $y$ and $y$ is the right neighbour point of $x$. The maximal and minimal points of $X$, if they exist, are also called the right and left end points respectively.

Two important concepts which later on we are going to use extensively are convex subsets and convex components of a linearly ordered set.

\begin{defn}
  A subset $K$ of a linearly ordered set $(X, <)$ is called convex if for all $a, b \in K, \; [a,b] \subseteq K$
\end{defn}

\begin{defn}
Suppose $U \subseteq X$, and $C$ is a convex subset of $U$. Then $C$ is called a convex component of $U$ if for all convex subsets $K$ of $U$,
we have $K \cap C \neq \phi \Rightarrow K \subseteq C$.
\end{defn}

Thus, a convex component of a subset $U$ of a linearly ordered set $X$ is a maximal convex subset of $U$. Every subset of a linearly ordered set can be decomposed into its convex components. We will see below that once we define the linear order topology or a generalized order topology on $(X,<)$, then one can show that a convex component of an open subset $U$ of $X$ is open in $X$.

\begin{defn}
  Let $(X,<)$ be a linearly ordered set, the topology $\lambda (<)$ on $X$ generated by the subbase $\{\,]\leftarrow, a[$ , $]b, \rightarrow[ \; | \; a,b \in X\}$ is called the linear order topology. The pair $(X,\lambda (<))$ is called a linearly ordered topological space (or LOTS).
\end{defn}

In the rest part of this thesis, we simply refer to topological spaces as spaces. 
\begin{defn}
  Let $(X,<)$ be a linearly ordered set, $\tau$ is a generalized order topology (or GO-space topology) on $X$ if it has a convex base and it is stronger than the linear order topology.
\end{defn}

It is important to note that the linear order topology of the induced order on a subset of a LOTS is not necessarily the same as the subspace topology of the subset. In fact, it can be proved that the subspace topology of a subset of a LOTS is a GO-space topology.
\begin{thm}\label{X*}
  A GO-space is a closed subspace of a LOTS.
\end{thm}
\begin{proof}
  Let $(X,<,\tau)$ be a GO-space, $X^{*}$ a subspace of the LOTS $X \times \mathbb{Z}$ defined by
  \begin{align*}
    X^{*} =& \; X \times \{0\} \\
    &\cup \{(x,n) \; | \; [x, \rightarrow[ \, \in \tau \backslash \lambda (<), \, n < 0 \} \\
    &\cup \{(x,n) \, | \, ] \leftarrow, x] \in \tau \backslash \lambda (<), \, n > 0 \}
  \end{align*}
  Here the order topology on $X \times \mathbb{Z}$ is induced by the linear order of the lexicographic product of $X$ and $\mathbb{Z}$ with the usual order, see Definition \ref{def:lexicographic_product}. Then $X \times \{0\}$ is closed in $X^{*}$ because for each $(x,n)$ such that $n \neq 0$, we have $(x,n) \in U = \, ](x, n-1), (x, n+1)[$ and $U \cap (X \times \{0\}) = \phi$. It is not difficult to see that $(X, <, \tau)$ is homeomorphic to the closed subspace $X \times \{0\}$ of $X^{*}$.
\end{proof}

\begin{thm}
  A GO-space is a dense subspace of a LOTS.
\end{thm}
\begin{proof}
  Let $(X,<,\tau)$ be a GO-space, $X^{**}$ a subspace of the LOTS $X \times \mathbb{Z}$ defined by
  \begin{align*}
    X^{**} =& \; X \times \{0\} \\
    &\cup \{(x,-1) \; | \; [x, \rightarrow[ \, \in \tau \backslash \lambda (<)\} \\
    &\cup \{(x,1) \; | \; ] \leftarrow, x] \in \tau \backslash \lambda (<) \}
  \end{align*}
  Then $X$ is homeomorphic to the dense subspace $X \times \{0\}$.
\end{proof}

It is obvious that a subspace of a LOTS is a GO-space. Combining with the above two theorems, we can conclude that any GO-space topology coincides with the subspace topology of  a LOTS. Also a GO-space $X$ is a LOTS if and only if $X^{*} = X^{**} = X$.

\begin{thm}
  Let $(X,\lambda (<))$ be a LOTS, $S \subseteq X$. Then the linear order topology on $S$ of the induced order is the same as the subspace topology if $S$ is a convex subset of $X$.
\end{thm}
\begin{proof}
  Let $\tau$ be the topology on $S$ as a subspace of $X$, and $\lambda (<_{S})$ be the linear order topology of the induced order. Clearly $\lambda (<_{S}) \subseteq \tau$. We just need to prove that $\tau \subseteq \lambda (<_{S})$.

 Consider $]x, \rightarrow[$ open in $X$, then $U = \, ]x, \rightarrow[ \, \cap S$ is open in $(S, \tau)$. If $x \in S$, then $U$ is an open ray in $(S, \lambda (<_{S}))$. If $x \notin S$, then since $S$ is convex, we have $]x, \rightarrow[ \, \cap S$ is either equal to $\phi$ or $S$ which are again open in $(S, \lambda (<_{S}))$. A similar argument applies for sets of the type $]\leftarrow, y[$ . Hence we have $\tau \subseteq \lambda (<_{S})$.
\end{proof}

\begin{thm}
  Let $(X,\lambda (<))$ be a LOTS and $Y$ a dense subset of $X$. Then the linear order topology on $Y$ of the induced order is the same as the subspace topology if and only if the following condition holds: each neighbour point of $X$ belongs to $Y$ if and only if its left and/or right neighbour in $X$ also belongs to $Y$.
\end{thm}
\begin{proof}
  Suppose $(Y,\tau) = (Y, \lambda (<_{Y}))$, where $\tau$ is the subspace topology, and let $y',y \in X$ be neighbour points with $y \in Y$. Without loss of generality, say $y' < y$. then $]y',\rightarrow[ \, \in \lambda (<)$, therefore $]y',\rightarrow[ \, \cap Y \in \tau$. If $y' \notin Y$, then since $\tau = \lambda (<_{Y})$, we have either $y$ is the left end point of $Y$ or there exists $y_1 \in Y$ such that $y_1$ is the left neighbour of $y$ in $Y$. Since $y'$ is the left neighbour point of $y$ in $X$, we have $y_1 < y'$.  In the former case, we have $]\leftarrow, y'] \, \cap Y = \phi$. In the latter case, $y' \in \, ]y_1,y[$ and $]y_1,y[ \, \cap Y = \phi$. In both cases, $y' \notin \overline{Y}$ contradicting that $Y$ is dense in $X$. Therefore, $y' \in Y$.

  Conversely, suppose for every pair of neighbour points $x_1,x_2 \in X$, we have either $x_1,x_2 \in Y$ or $x_1,x_2 \notin Y$. Suppose $y_1 \in Y$ and $U = [y_1, \rightarrow[ \, \cap Y$ is open in the subspace topology, then there exists $y_2 \in X$ such that $U = \, ]y_2, \rightarrow[ \, \cap Y$. If $y_1$ has a left neighbour point $y_1'$ in $X$, then $y_1' \in Y$, hence $]y_1', \rightarrow[ \, = \, [y_1,\rightarrow[$ is open in $(Y,\lambda (<_{Y}))$. Suppose $y_1$ has no left neighbour point in $X$, and $U$ is not open in $(Y, \lambda (<_{Y}))$, then $y_1$ is not the left end point of $Y$ and it has no left neighbour in $Y$, which implies that there exists $x \in \, ]y_2, y_1[$ and $]y_2,y_1[ \, \cap Y = \phi$. Therefore, $x \notin X = \overline{Y}$, contradiction.   
\end{proof}


\begin{lem}
  Let $(X,<,\tau)$ be a GO-space, then every non-empty $U \subseteq X$ can be decomposed into its convex components. If $U$ is open in $X$, then every convex component of $U$ is also open in $X$.
\end{lem}
\begin{proof}
  Let  $U \subseteq X$, and $x' \in U$. Define $C = \{x \in U \; | \; [x,x'] \subseteq U $ or $[x',x] \subseteq U \}$. Then $C$ is a convex component and $C \neq \phi$ since $x' \in C$. Since $U$ is open in $X$, for any $x \in C \subseteq U$, there exists a convex neighbourhood $C_x$ of $x$ such that $C_x \subseteq U$. Since $C$ is a convex component of $U$, we have $C_x \subseteq C$. Therefore, $C$ is open in $X$.
\end{proof}

\begin{lem}
  Let $(X,<,\tau)$ be a GO-space, $F$ a closed subset of $X$. Then a convex component $F_{\alpha}$ of $F$ is also closed in $X$.
\end{lem}
\begin{proof}
  We first decompose $F$ into its convex components $\{F_{\alpha}\}_{\alpha \in \mathcal{A}}$. Suppose $F_{\alpha_{0}}$ is not closed in $X$, let $x \in \overline{F_{\alpha_{0}}} \backslash F_{\alpha_0}$. Since $F$ is closed, then $x \in F$, hence there exists $F_{\alpha_{1}}$ such that $x \in F_{\alpha_{1}}$. Without loss of generality, say $x < y$ for all $y \in F_{\alpha_{0}}$. Assume that there exists $a \in X$ such that for all $y \in F_{\alpha_{0}}$, $x < a < y$, then $] \leftarrow, a[ \, \cap F_{\alpha_{0}} = \phi$, contradicting $x \in \overline{F_{\alpha_{0}}}$. Therefore, $x$ is the maximal point of $F_{\alpha_{1}}$ and there exists no point $a$ in $X$ such that for all $y \in F_{\alpha_{0}}$ $x < a < y$, which implies $F_{\alpha_{0}} \cup F_{\alpha_{1}}$ is convex, contradicting $F_{\alpha_{0}}$ and $F_{\alpha_{1}}$ are distinct convex components of $F$.
\end{proof}

\section{Gaps, Pseudo gaps and Jumps}
Gaps, Pseudo gaps and Jumps are important concepts in LOTS and GO-spaces. When we discuss compactness, paracompactness and connectedness, compared with open covers and partitions, it is more convenient to work with these concepts.

\begin{defn}
  Let $X$ be a GO-space, $A$ and $B$ disjoint subsets of $X$ such that $X = A \cup B$. Then the ordered pair $(A,B)$ is a gap in $X$ if
  \begin{itemize}
  \item for all $a \in A$ and for all $b \in B$, $a < b$.
  \item $A$ has no maximal point and $B$ has no minimal point.
  \end{itemize}
  If $A = \phi$, then $(A,B)$ is called a left end gap, and if $B = \phi$, then it is called a right end gap. 
\end{defn}
Note that if $(A,B)$ is a gap in $X$, then $A$ and $B$ are closed and open subsets of $X$.
\begin{defn}
  Let $X$ be a GO-space, $A$ and $B$ non-empty open disjoint subsets of $X$ such that $X = A \cup B$, then the ordered pair $(A,B)$ is called a pseudo gap if
  \begin{itemize}
  \item for all $a \in A$ and for all $b \in B$, $a < b$.
  \item either $A$ has a maximal point and $B$ has no minimal point, or $A$ has no maximal point and $B$ has a minimal point.
  \end{itemize}
\end{defn}
One can note that pseudo gaps pertain only to GO-spaces that are not LOTS.

An important type of gap (pseudo gap) is a $Q-$gap ($Q-$pseudo gap). A gap (pseudo gap) $(A,B)$ is called a $Q-gap$ ($Q-$pseudo gap) if $A$ has a discrete cofinal subset $A'$ and $B$ has a discrete coinitial subset $B'$. Q-gaps or Q-pseudo gaps can lead to many interesting results, and as we are going to discuss in later chapters, one of the important applications is that they can be used to characterize paracompact GO-spaces.

\begin{defn}
  Let $X$ be a GO-space, $A$ and $B$ open disjoint subsets of $X$ such that $X = A \cup B$. Then the ordered pair $(A,B)$ is a jump in $X$ if
  \begin{itemize}
  \item for all $a \in A$ and for all $b \in B$, $a < b$.
  \item $A$ has a maximal point and $B$ has a minimal point.
  \end{itemize}
\end{defn}


\section{Dedekind compactification}

Intuitively, gaps in a LOTS are holes in the LOTS. We can use the following construction to fill the holes up with newly defined points representing gaps.  
\begin{defn}\label{dedekind}
  Let $X$ be a LOTS, for each gap $(A, B)$ in $X$, we define a new point $c_{(A,B)}$ representing the gap $(A,B)$. Then define $X^{+} = X \cup \{c_{(A,B)} \;| \; (A,B) \text{ is a gap in } X \}$ and a linear order $<^+$ such that:
  \begin{itemize}
  \item For all $x,y \in X$, $x <^+ y$ if $ x < y$
  \item $x <^+ c_{(A,B)}$ if $x \in A$
  \item $c_{(A,B)} <^+ x$ if $x \in B$.
  \item If $(A,B)$ and $(C,D)$ are gaps and there exists $x \in X$ such that $c_{(A,B)} <^+ x <^+ c_{(C,D)}$, then $c_{(A,B)} <^+ c_{(C,D)}$
  \end{itemize}
\end{defn}
When we study lexicographic products, we sometimes do not distinguish between a gap $(A,B)$ and its corresponding point $c_{(A,B)}$ in the Dedekind compactification.

It will follow from Theorem \ref{go:compactness:1} that the construction given in Definition \ref{dedekind} is a compactification of the LOTS $X$. In case that $X$ is not a LOTS but only a GO-space, we first construct the LOTS $X^{**}$, then take the Dedekind compactification $(X^{**})^{+}$. The Dedekind compactification is used later when we discuss the gaps in the lexicographic product of LOTS.

Let $X$ be a LOTS, and $U \subseteq X$. A gap $(A,B)$ is said to be covered by $U$ if there exists a convex set $V \subseteq U$ in $X$ such that $V \cap A \neq \phi$ and $V \cap B \neq \phi$. A cover $\mathcal{U}$ is said to cover the gap $(A,B)$ if there exists $U \in \mathcal{U}$ such that $U$ covers $(A,B)$. 

\begin{lem}
  An open cover $\mathcal{U}$ of a LOTS $X$ has a finite subcover if every gap of $X$ is covered by $\mathcal{U}$.
\end{lem}
\begin{proof}
  Consider the Dedekind compactification $X^+$ of $X$. For each $U \in \mathcal{U}$, define $U^+ = U \cup\{\text{all gaps of } X \text{ that are covered by } U\}.$ Then $\mathcal{U}^+ = \{ U^+ \; | \; U \in \mathcal{U}\}$ is an open cover of $X$. Since $X^+$ is compact and $\mathcal{U}^+$ is an open cover of $X^+$, $\mathcal{U}^+$ has a finite subcover $\mathcal{V}^+$. Then $\mathcal{V} = \{V^+ \cap X \; | \; V^+ \in \mathcal{V}^+ \}$ is a finite subcover of $\mathcal{U}$.
\end{proof}

\section{Monotonically normal spaces}

\begin{defn}
  A space $X$ is monotonically normal if there exists a function $G$ which maps each ordered pair $(H,K)$ of disjoint closed subsets of $X$ to an open subset $U_{(H,K)} \subseteq X$ with the follow properties:
  \begin{enumerate}
  \item $H \subseteq U_{(H,K)} \subseteq \overline{U_{(H,K)}} \subseteq X \backslash K$.
  \item If $(H',K')$ is an ordered pair of disjoint closed subsets such that $H \subseteq H'$ and $K' \subseteq K$, then $U_{(H,K)} \subseteq U_{(H',K')}$.
  \end{enumerate}
The function $G$ is called a monotone normality operator.
\end{defn}

\begin{lem} \label{lem:mononormal:1}
  Let $X$ be a monotonically normal space, then there exists a monotone normality operator $G$ satisfying $G(A,B) \cap G(B,A) = \phi$ for any ordered pair $(A,B)$ of disjoint closed subsets of $X$.
\end{lem}
\begin{proof}
  Let $G'$ be any monotone normality operator on $X$, $(A,B)$ an ordered pair of disjoint closed subsets.

  Consider $G(A,B) = G'(A,B) \backslash \overline{G'(B,A)}$. We first show $G$ is a monotone normality operator. Clearly, $G(A,B)$ is open in $X$, $A \subseteq G'(A,B)$. In addition, we also have $\overline{G'(B,A)} \subseteq X \backslash A$ implies $A \subseteq G(A,B)$, and $G(A,B) \subseteq G'(A,B)$ implies $\overline{G(A,B)} \subseteq \overline{G'(A,B)} \subseteq X \backslash B$. Now, consider $(A',B')$ an ordered pair of disjoint closed subsets of $X$ such that $A \subseteq A'$ and $B' \subseteq B$, then we have $G'(A,B) \subseteq G'(A',B')$ and $G'(B',A') \subseteq G'(B,A)$. Therefore, $G(A,B) \subseteq G(A',B')$, hence $G$ is a monotone normality operator on $X$.

  Now, $G(A,B) \cap G(B,A) = G'(A,B) \cap X \backslash \overline{G'(B,A)} \cap G'(B,A) \cap X \backslash \overline{G'(A,B)} = \phi$.
\end{proof}

\begin{thm} \label{thm:mononormal:2}
  Let $X$ be a $T_1$ space, then the following properties of $X$ are equivalent to monotone normality of $X$.
  \begin{enumerate}
  \item There exists a function $G$ which maps each ordered pair $(A,B)$ of separated subsets of $X$ to an open set $G(A,B)$ such that
    \begin{enumerate}
    \item $A \subseteq G(A,B) \subseteq \overline{G(A,B)} \subseteq X \backslash B$.
    \item If $(A',B')$ is an ordered pair of separated subsets of $X$ such that $A \subseteq A'$, $B' \subseteq B$, then $G(A,B) \subseteq G(A',B')$.
    \end{enumerate}
  \item There exists a function $H$ which assigns to each ordered pair $(p,C)$ where $C$ is a closed subset of $X$ and $p \in X \backslash C$, an open subset $H(p,C)$ such that:
    \begin{enumerate}
    \item $p \in H(p,C) \subseteq X \backslash C$.
    \item If $D \subseteq C$ is a closed and $p \notin C$, then $H(p,C) \subseteq H(p,D)$.
    \item If $q \in X$ and $p \neq q$, then $H(p,\{q\}) \cap H(q,\{p\}) = \phi$.
    \end{enumerate}
  \end{enumerate}
\end{thm}
\begin{proof}
  Since each pair of disjoint closed sets are separated, we have $(1)$ implies monotone normality, which in turns implies $(2)$.

  Now, suppose $(2)$ is true, let $(A,B)$ be a pair of separated subsets of $X$. Then we have for all $a \in A$, $a \notin \overline{B}$. Define $G(A,B)$ $=$ $\bigcup \{H(a,\overline{B}) \; | \; a \in A\}$, then $A \subseteq G(A,B)$. Let $b \in B$, then consider $H(b,\overline{A})$, we have $H(b,\overline{A}) \subseteq H(b,\{a\})$ for any $a \in A$. Similarly, $H(a,\overline{B}) \subseteq H(a,\{b\})$, but $H(b,\{a\}) \cap H(a,\{b\}) = \phi$, hence we have for all $a \in A$, $H(b,\overline{A}) \cap H(a,\overline{B}) = \phi$. Therefore, $H(b,\overline{A})$ is an open neighbourhood of $b$ disjoint from $G(A,B)$, which implies $\overline{G(A,B)} \subseteq X \backslash B$.
  
  Let $(A',B')$ be an ordered pair of separated subsets of $X$ such that $A \subseteq A'$, $B' \subseteq B$, then for any $x \notin \overline{B}$, $H(x,\overline{B}) \subseteq H(x,\overline{B'})$, hence we have $G(A,B) \subseteq \cup\{H(a,\overline{B'}) \; | \; a \in A\} \subseteq G(A',B')$.
\end{proof}

%\begin{defn}
%  Let $X$ be a space, an open cover $\mathcal{W}$ of $X$ is said to be normal if there exists a sequence $\mathcal{W}_{1}$, $\mathcal{W}_{2}$, $\ldots$ of open covers of $X$ such that
%  \begin{enumerate}
%  \item $\mathcal{W}_{1} = \mathcal{W}$.
%  \item $\mathcal{W}_{i+1}$ is a star refinement of $\mathcal{W}_{i}$ for $i \geq 1$.
%  \end{enumerate}
%\end{defn}


\begin{defn}
  A $T_{1}$ space $X$ is called collectionwise normal if for any discrete collection $\{F_{\alpha}\}_{\alpha \in \mathcal{A}}$ of closed subsets of $X$, there exists a discrete collection $\{U_{\alpha}\}_{\alpha \in \mathcal{A}}$ of open subsets such that $F_{\alpha} \subseteq U_{\alpha}$ for all $\alpha \in \mathcal{A}$. 
\end{defn}

It follows directly from the definition that every collectionwise normal space is normal.

\begin{thm}
  A $T_1$ space $X$ is collectionwise normal if and only if for every discrete family $\{F_s\}_{s \in \mathcal{S}}$ of closed subsets of $X$ there exists a family $\{U_s\}_{s \in \mathcal{S}}$ of open subsets of $X$ such that $F_s \subseteq U_s$ for every $s \in \mathcal{S}$ and $U_s \cap U_{s'} = \phi$ whenever $s \neq s'$.
\end{thm}
\begin{proof}
 The ``only if'' part is simple, we just show the ``if'' part. Say $X$ is a $T_1$ space satisfying the hypothesis. Clearly, $X$ is normal. Let $\{F_s\}_{s \in \mathcal{S}}$ be a discrete family of closed subsets of $X$. Then there exists a family $\{U_s\}_{s \in \mathcal{S}}$ of pairwise disjoint open subsets such that $F_s \subseteq U_s$. Now let $A = \bigcup_{s \in \mathcal{S}}F_s$ and $B = X \backslash \bigcup_{s \in \mathcal{S}}U_s$, then $A$ and $B$ are closed. By normality of $X$, there exists disjoint open subsets $U$, $V$ of $X$ such that $A \subseteq U$ and $B \subseteq V$. Define $\{V_s\}_{s \in \mathcal{S}}$ where $V_s = U_s \cap U$, then we have $F_s \subseteq V_s$.

  We now show $\{V_s\}_{s \in \mathcal{S}}$ is discrete. Let $x \in X$, if $x \in V$, then $V \cap V_s = \phi$, if $x \in X \backslash V \subseteq X \backslash B$, then there exists $s(x) \in \mathcal{S}$ such that $x \in U_{s(x)}$ and $U_{s(x)}$ intersects only $V_{s(x)}$. Therefore, $\{U_s\}_{s \in \mathcal{S}}$ is a discrete family of open subsets of $X$ such that $F_s \subseteq U_s$ for every $s \in \mathcal{S}$. Hence $X$ is collectionwise normal.
\end{proof}

\begin{thm} \label{thm:mononormal:3}
  A monotonically normal space is hereditarily collectionwise normal.
\end{thm}
\begin{proof}
  Let $X$ be a monotonically normal space and $G$ a monotone normality operator for $X$ such that for each ordered pair $(A,B)$ of closed disjoint subsets of $X$, we have $G(A,B) \cap G(B,A) = \phi$. Let $\mathcal{D}$ be a discrete collection of closed subsets of $X$, hence, subsets in $\mathcal{D}$ are pairwise disjoint. For each $A \in \mathcal{D}$, define $U_{A} = G(A,\bigcup \{A' \in \mathcal{D}$ $|$ $A' \neq A \})$, then for any $A_{1},A_{2} \in \mathcal{D}$ and $A_{1} \neq A_{2}$, we have $U_{A_{1}} \subseteq G(A_{1},A_{2})$ and $U_{A_{2}} \subseteq G(A_{2},A_{1})$. Therefore $U_{A_{1}} \cap U_{A_{2}} \subseteq G(A_{1},A_2) \cap G(A_{2},A_1)  = \phi$. Hence, $\mathcal{U} = \{ U_{A} \}_{A \in \mathcal{D}}$ is the required disjoint collection of open sets.  Heredity follows directly from Theorem \ref{thm:mononormal:2}.
\end{proof}

\begin{thm} \label{thm:mononormal:4}
  A LOTS is monotonically normal.
\end{thm}
\begin{proof}
  Let $(X,\lambda (<))$ be a LOTS, $<'$ be any well-ordering defined on $X$, ($<'$ does not need to have any relation to $<$). For any pair $(p,C)$, where $C$ is a closed subset of $X$ and $p \in X \backslash C$, we define an open subset $G(p,C)$ of $X$ satisfying (2) of Theorem \ref{thm:mononormal:2}. Now $X \backslash C$ is open in $X$, let $l(p,C)$ be the (open) convex component of $X \backslash C$ containing $p$. Define $l_{-} = \{y \in l(p,C) \, | \, y < p\}$ and $l_{+} = \{ y \in l(p,C) \, | \, y > p \}$. If $l_{-} \neq \phi$ then let $a$ be the $<'$-minimal point of $l_{-}$ and if $l_{+} \neq \phi$ then let $b$ be the $<'$-minimal point of $l_{+}$.
  Now, we define
\[
    G(p,C) = \left\{
    \begin{array}{@{}rcl}
      {]a,b[} & \text{if} & l_- \neq \phi \neq l_+ \\
      {[p,b[} & \text{if} & l_- = \phi \neq l_+ \\
      {]a,p]}   & \text{if} & l_- \neq \phi = l_+ \\
      \{p\}   & \text{if} & l_- = \phi = l_+
    \end{array} \right.
  \]
  Then $G(p,C)$ is an open subset of $X$, and it is clear that $G$ satisfies (2) of Theorem \ref{thm:mononormal:2}. 
\end{proof}
\begin{corollary}
  Every LOTS is hereditarily collectionwise normal.
\end{corollary}

\begin{thm}
  Every GO-space is hereditarily collectionwise normal.
\end{thm}
\begin{proof}
  It follows from the fact that every GO-space is a subspace of a LOTS.
\end{proof}

\section{Lexicographic Products of LOTS}
 
\begin{defn}\label{def:lexicographic_product}
  Let $(X,<_{X})$, $(Y,<_{Y})$ be linearly ordered sets, the lexicographic product $X \cdot Y$ is defined as the Cartesian product supplied with the lexicographic order $<$ defined by $(x_{1},y_{1}) < (x_{2}, y_{2})$ if either $x_{1} < x_{2}$ or $x_{1} = x_{2}\, , \, y_{1} < y_{2}$. The topology on $X \cdot Y$ is defined as the linear order topology with respect to the lexicographic order.
\end{defn}
We can extend the above definition in the following way.

\begin{defn}
Let $(X_{\alpha}, <_{\alpha})$ be linearly ordered sets for all $\alpha \in \mathcal{A}$, where $\mathcal{A}$ is a set of ordinal numbers. Then the topology on the lexicographic product $X = \LP_{\alpha \in \mathcal{A}}X_{\alpha}$ is defined as the linear order topology with respect to the lexicographic order, where the lexicographic product is defined as the Cartesian product supplied with the lexicographic order, i.e. if $x = (x_{\alpha})_{\alpha \in \mathcal{A}}$, $y = (y_{\alpha})_{\alpha \in \mathcal{A}} \in X$, then $x < y$ if and only if there exists $\beta \in \mathcal{A}$ such that $x_{\beta} <_{\beta} y_{\beta}$ and $x_{\alpha} = y_{\alpha}$ for all $\alpha < \beta$ and $\alpha \in \mathcal{A}$.
\end{defn}

In the above definition, we defined the lexicographic products of LOTS on a set of ordinal numbers. However, the indexing set can be an arbitrary set as long as it is well-ordered.  

\subsection{Jumps in Lexicographic Products of LOTS}
\begin{lem} \label{lem:ljump:1}
  Let $\mathcal{A}$ be a set of ordinal numbers. Let $X_{\alpha}$ be a linearly ordered set, for each $\alpha \in \mathcal{A}$. If $x , y \in \LP_{\alpha \in \mathcal{A}}X_{\alpha}$ and $x < y$, then $x$ and $y$ are neighbours in $\LP_{\alpha \in \mathcal{A}}X_{\alpha}$ if and only if the smallest ordinal $\beta \in \mathcal{A}$ such that $x_{\beta} \neq y_{\beta}$ satisfies:
  \begin{enumerate}
  \item $x_{\beta}$ and $y_{\beta}$ are neighbours in $X_{\beta}$.
  \item $\forall \alpha \in \mathcal{A}$ such that $\alpha > \beta$ we have $x_{\alpha}$ is the right end point of $X_{\alpha}$ and $y_{\alpha}$ is the left end point of $X_{\alpha}$.
  \end{enumerate}
\end{lem}
\begin{proof}
  $(\Rightarrow)$. Suppose $x,y$ are neighbours in $\LP_{\alpha \in \mathcal{A}}X_{\alpha}$, and  $\beta \in \mathcal{A}$ is the smallest ordinal such that $x_{\beta} \neq y_{\beta}$. Suppose that  $x_{\beta}$ and $y_{\beta}$ are not neighbours in $X_{\beta}$, then there exists $z_{\beta} \in X_{\beta}$ such that $x_{\beta} < z_{\beta} < y_{\beta}$. The point $a = (a_{\alpha})_{\alpha \in \mathcal{A}}$, where $a_{\alpha} = x_{\alpha}$ for $\alpha \neq \beta$ and $a_{\beta} = z_{\beta}$, has the property that $x < a < y$, contradicting $x$ and $y$ are neighbours.

  Now suppose there exists $\alpha_{0} > \beta$ such that $x_{\alpha_{0}}$ is not the right end point of $X_{\alpha_{0}}$, then there exists $b_{\alpha_{0}} \in X_{\alpha_{0}}$ such that $x_{\alpha_{0}} < b_{\alpha_{0}}$. Then for the point $b = (b_{\alpha})_{\alpha \in \mathcal{A}}$, where $b_{\alpha} = x_{\alpha}$ for $\alpha \neq \alpha_{0}$, we have $x < b < y$, contradiction. Similarly, we can show that for all $\alpha \in \mathcal{A}$, $y_{\alpha}$ is the left end point of $X_{\alpha}$.

  $(\Leftarrow)$ This is proved with the same argument as above.
\end{proof}

The above lemma gives us the relationship between jumps in a lexicographic product and jumps in the factor spaces.

\begin{corollary} \label{lem:ljump:2}
  Let $X_{\alpha}$ be LOTS for $\alpha < \mu$ where $\mu$ is a limit ordinal and $X = \LP_{\alpha < \mu}X_{\alpha}$ be the lexicographic product. Then $X$ has a jump implies there exists $\alpha_{0} < \mu$ such that $X_{\alpha_{0}}$ has a jump.
\end{corollary}
\begin{proof}
  Suppose $X$ has a jump $(A,B)$, then let $a = (a_{\alpha})_{\alpha < \mu}$ be the maximal point of $A$, $b = (b_{\alpha})_{\alpha < \mu}$ be the minimal point of $B$ and $\beta < \mu$ be the smallest ordinal such that $a_{\beta} \neq b_{\beta}$. That $a,b$ are neighbours in $X$ implies $a_{\beta}, b_{\beta}$ are neighbours in $X_{\beta}$. Therefore, $(A_{\beta}, B_{\beta})$ is a jump in $X_{\beta}$ where $A_{\beta} = \, ]\leftarrow, a_{\beta}]  = \, ]\leftarrow, b_{\beta}[ \, \subseteq X_{\beta}$ and $B_{\beta} = \, [b_{\beta}, \rightarrow[ \, = \, ]a_{\beta}, \rightarrow [ \, \subseteq X_{\beta}$.

\end{proof}

\subsection{Gaps in Lexicographic products of LOTS}


In the following proof, we are going to use the Dedekind compactification of LOTS. For a LOTS $(X, <)$, we denote its Dedekind compactification by $(X^+,<^+)$, where $X^+ = X \cup \{c \; | \; \text{c is a gap in X}\}$ and $<^+$ is the linear order extended from $X$ to $X^+$, as described in Section \ref{dedekind}.
We also define $X' = \LP_{\alpha < \mu}X_{\alpha}^+$ with the order $<'$, where $\mu$ is an ordinal number greater than 1, for $x' = (x_{\alpha}')_{\alpha < \mu}, y' = (y_{\alpha}')_{\alpha < \mu} \in X'$, $x' <' y'$ if there exists $\alpha_0 < \mu$ such that $x_{\alpha_0}' <_{\alpha_0}^+ y_{\alpha_0}'$ and $x_{\alpha}' = y_{\alpha}'$ for $\alpha < \alpha_0$.
\begin{thm} \label{lem:lgap:1}
  Let $X_{\alpha}$ be LOTS for all $\alpha < \mu$, where $\mu$ is an ordinal number greater than 1, $X =  \LP_{\alpha < \mu} X_{\alpha}$ be the lexicographic product.
 % Let $c = (A,B)$ be a gap in $X$, then there exists a gap $c_{\alpha_{0}} = (A_{\alpha_{0}}, B_{\alpha_{0}})$ in $X_{\alpha_{0}}$ for some $\alpha_0 < \mu$ such that
  %$c  = (c_{\alpha})_{\alpha < \mu}$, where $c_{\alpha}  \in X_{\alpha}$ for $\alpha \neq \alpha_{0}$.
Let $(A,B)$ be a gap in $X$. Then there exists $c  = (c_{\alpha})_{\alpha < \mu} \in \LP_{\alpha < \mu}X_{\alpha}^+$ with $A = \,]\leftarrow,c[ \, \cap X$, $B = \, ]c,\rightarrow[ \, \cap X$ such that $c_{\alpha_{0}} \in X_{\alpha_0}^+ \backslash X_{\alpha_{0}}$ for some $\alpha_0 < \mu$ and $c_{\alpha}  \in X_{\alpha}$ for $\alpha \neq \alpha_{0}$.
\end{thm}

\begin{proof}
  Define
  $$A_{0}  =  \{a_{0}  \in X_{0} \; | \; \exists \ a  = (x_{\alpha})_{\alpha < \mu}  \in A \; \text{such that} \; x_{0}  =  a_{0} \}$$
  $$B_{0}  =  \{b_{0}  \in X_{0} \; | \; \exists \ b = (x_{\alpha})_{\alpha < \mu}  \in B  \; \text{such that} \; x_{0}  =  b_{0} \}$$
  We first show that  $|A_{0} \cap B_{0}| \leq 1$ and for any  $a_{0} \in A_{0}$ and $b_{0} \in B_{0}$, $a_{0} \leq_0 b_{0}$.

  Suppose that $A_0 \cap B_0 = \{s_{0},s_{0}'\}$ and without loss of generality, assume $s_0 <_0 s_0'$. Then there exist $a = (a_{\alpha})_{\alpha < \mu} \in A$, $b = (b_{\alpha})_{\alpha < \mu} \in B$, where $a_0 = s_0'$, $b_0 = s_0$ and $a_{\alpha}, b_{\alpha}$ are points in $X_{\alpha}$ for $\alpha > 0$. Clearly in this case, we have $b < a$. Contradiction.    
  
  Hence, we can conclude $|A_{0} \cap B_{0}| \leq 1$ and it is not difficult to see that for all $a_{0} \in A_{0}$ and $b_{0} \in B_{0}$, $a_{0} \leq_0 b_{0}$.

  Now given any $\alpha_{0}$, we show that if for all $\alpha < \alpha_{0}$ we have $A_{\alpha} \cap B_{\alpha} = \{s_{\alpha}\}$, then we have $|A_{\alpha_{0}} \cap B_{\alpha_{0}}| \leq 1$ and for any $a_{\alpha_{0}}  \in A_{\alpha_{0}}$ and $b_{\alpha_{0}}  \in B_{\alpha_{0}}$, we have $a_{\alpha_{0}} \leq_{\alpha_0} b_{\alpha_{0}}$, where 
  $$A_{\alpha_0}  =  \{a_{\alpha_{0}}  \in X_{\alpha_{0}} \; | \; \exists a  =  (s_{\alpha})_{\alpha < \alpha_{0}} \cdot (a_{\alpha})_{\alpha_{0} \leq \alpha < \mu}  \in A  \}$$
  $$B_{\alpha_0}  =  \{b_{\alpha_{0}}  \in X_{\alpha_{0}} \; | \; \exists b  =  (s_{\alpha})_{\alpha < \alpha_{0}} \cdot (b_{\alpha})_{\alpha_{0} \leq \alpha < \mu}  \in B  \}.$$ 

  Suppose for all $\alpha < \alpha_{0}$, we have $A_{\alpha} \cap B_{\alpha} = \{s_{\alpha}\}$, then consider the point $x = (s_{\alpha})_{\alpha < \alpha_{0}} \cdot (x_{\alpha})_{\alpha_{0} \leq \alpha < \mu} \in X$, where for $\alpha_0 \leq \alpha < \mu$, $x_{\alpha}$ is an arbitrary point in $X_{\alpha}$.

  Since $c  =  (A,B)$ is a gap in X, we have $x \in A$ or $x \in B$, hence the sets $A_{\alpha_0}$ and $B_{\alpha_0}$ are defined (one of them can be empty).

  Suppose that $A_{\alpha_{0}} \cap B_{\alpha_{0}}  =  \{s_{\alpha_{0}}, s_{\alpha_{0}}'\}$ and $s_{\alpha_{0}} <_{\alpha_0} s_{\alpha_{0}}'$. Then there exists $b' \in B$ such that $b' =  (s_{\alpha})_{\alpha < \alpha_{0}} \cdot s_{\alpha_{0}} \cdot (b_{\alpha})_{\alpha_{0} < \alpha < \mu}$ and there exists $a' \in A$ such that $a' = (s_{\alpha})_{\alpha < \alpha_{0}} \cdot s_{\alpha_{0}}' \cdot (a_{\alpha})_{\alpha_{0} < \alpha < \mu}$, where $b_{\alpha},a_{\alpha} \in X_{\alpha}$ for all $\alpha_0 < \alpha < \mu$. Then it is clear that $b' < a'$, contradicting that $(A,B)$ is a gap in $X$. Similarly, for all $a_{\alpha_{0}} \in A_{\alpha_{0}}$ and $b_{\alpha_{0}}  \in B_{\alpha_{0}}$, we have $a_{\alpha_{0}} \leq_{\alpha_0} b_{\alpha_{0}}$.

  Next we have to show that there exists $\alpha' < \mu$ such that $A_{{\alpha}'} \cap B_{{\alpha}'}  =  \phi$.

  Suppose not, say for all $\alpha < \mu$ we have $A_{\alpha} \cap B_{\alpha}  =  \{s_{\alpha}\}$. Consider the point $s  =  (s_{\alpha})_{\alpha < \mu}  \in X$. Then $s  \in A$ or $s  \in B$. Suppose $s  \in A$. Since $(A,B)$ is a gap, $A$ has no maximal point. Therefore, there exists $a  \in A$ such that $s < a$. Let $\alpha_{1}$ be the smallest ordinal such that $s_{\alpha_{1}} <_{\alpha_1} a_{\alpha_{1}}$, i.e. $a  =  (s_{\alpha})_{\alpha < \alpha_{1}} \cdot a_{\alpha_{1}} \cdot (a_{\alpha})_{\alpha_{1} < \alpha < \mu}$. Then by the definition of $A_{\alpha_{1}}$, we have $a_{\alpha_{1}} \in A_{\alpha_{1}}$. However, by hypothesis, $A_{\alpha_{1}} \cap B_{\alpha_{1}}  =  \{s_{\alpha_{1}} \}$ and for any $a'_{\alpha_{1}}  \in A_{\alpha_{1}}$ and $b'_{\alpha_{1}} \in B_{\alpha_{1}}$, we have $a'_{\alpha_{1}} \leq_{\alpha_1} b'_{\alpha_{1}}$, but in this case we have $s_{\alpha_{1}}  \in B_{\alpha_{1}}$ and $s_{\alpha_{1}} <_{\alpha_1} a_{\alpha_{1}}$, giving a contradiction.

  Similarly we can show that $s$ is not in $B$. Hence we have a contradiction since $X  =  A \cup B$.


Now, let $\alpha_{2}$ be the smallest ordinal such that $A_{\alpha_{2}} \cap B_{\alpha_{2}} = \phi$. We then have four cases:
\begin{enumerate}
\item $A_{\alpha_{2}}$ has no maximal point and $B_{\alpha_{2}}$ has no minimal point.
\item $A_{\alpha_{2}}$ has the maximal point and $B_{\alpha_{2}}$ has no minimal point.
\item $A_{\alpha_{2}}$ has no maximal point and $B_{\alpha_{2}}$ has the minimal point.
\item $A_{\alpha_{2}}$ has the maximal point and $B_{\alpha_{2}}$ has the minimal point. 
\end{enumerate}
Let us consider the above four cases:
\begin{enumerate}
\item $A_{\alpha_{2}}$ has no maximal point and $B_{\alpha_{2}}$ has no minimal point.
  
  Since $\alpha_{2}$ is the smallest ordinal such that $A_{\alpha_{2}} \cap B_{\alpha_{2}}  =  \phi$, we have for all $\alpha < \alpha_{2}$, $A_{\alpha} \cap B_{\alpha} =  \{s_{\alpha} \}$ and for any $a_{\alpha} \in A_{\alpha}$ and $b_{\alpha} \in B_{\alpha}$, $a_{\alpha} \leq_{\alpha} b_{\alpha}$. Clearly, $A_{\alpha_{2}} \cup B_{\alpha_{2}}  =  X_{\alpha_{2}}$ and for any $a_{\alpha_{2}} \in A_{\alpha_{2}}$ and $b_{\alpha_{2}}  \in B_{\alpha_{2}}$ we have $a_{\alpha_{2}} <_{\alpha_2} b_{\alpha_{2}}$ and $A_{\alpha_{2}} \cap B_{\alpha_{2}}  =  \phi$. It easily follows that $A_{\alpha_{2}}, B_{\alpha_{2}}$ are open and convex in $X_{\alpha_{2}}$. Hence $(A_{\alpha_{2}}, B_{\alpha_{2}})$ is a gap in $X_{\alpha_{2}}$.

  Let $c_{\alpha_{2}}  =  (A_{\alpha_{2}}, B_{\alpha_{2}})$. Define $c' =(c'_{\alpha})_{\alpha < \mu}$ in $\LP_{\alpha < \mu}X_{\alpha}^+$ where $c_{\alpha}' = s_{\alpha}$ for $\alpha < \alpha_2$, $c_{\alpha_2}' = c_{\alpha_2}$ and $c_{\alpha}'$ are arbitrary points in $X_{\alpha}$ for $\alpha_2 < \alpha < \mu$. We now show that
  $ A  = \, ]\leftarrow, c'[ \, \cap X $ and $B = \, ] c', \rightarrow [ \, \cap X$.

  Let $a  =  (a_{\alpha})_{\alpha < \mu}  \in A$. Suppose $c' <' a$, and let $\alpha_{0}$ be the smallest ordinal such that $c_{\alpha_{0}}' <_{\alpha_{0}}^+ a_{\alpha_{0}}$
  \begin{enumerate}
  \item $\alpha_{0} < \alpha_{2}$.
    
    Then for all $\alpha < \alpha_{0}$, $c_{\alpha}'  =  s_{\alpha}  =  a_{\alpha}$. Therefore $a  =  (s_{\alpha})_{\alpha < \alpha_{0}} \cdot a_{\alpha_{0}} \cdot (a_{\alpha})_{\alpha_{0} < \alpha < \mu}$. Hence, we have $a_{\alpha_{0}}  \in A_{\alpha_{0}}$. Since $\alpha_{0} < \alpha_{2}$, we have $A_{\alpha_{0}} \cap B_{\alpha_{0}}  =  \{s_{\alpha_{0}} \}$, but in this case we have $s_{\alpha_{0}} <_{\alpha_0} a_{\alpha_{0}}$. Contradiction.
    
  \item $\alpha_{0}  =  \alpha_{2}$
    
    Then for all $\alpha < \alpha_{0}  =  \alpha_{2}$, we have $c_{\alpha}'  =  s_{\alpha}  =  a_{\alpha}$.
    
    Therefore, $a  =  (s_{\alpha})_{\alpha < \alpha_{2}} \cdot a_{\alpha_{2}} \cdot (a_{\alpha})_{\alpha_{2} < \alpha < \mu}$
    
    Since $\alpha_0 = \alpha_2$, $ c_{\alpha_{2}}  <_{\alpha_{2}}^+  a_{\alpha_{2}}$ which implies $a_{\alpha_{2}}  \in  B_{\alpha_{2}}$. This gives a contradiction, because $a  \in  A$ and $a  =  (s_{\alpha})_{\alpha < \alpha_{2}} \cdot a_{\alpha_{2}} \cdot (a_{\alpha})_{\alpha_{2} < \alpha < \mu}$ together imply $a_{\alpha_{2}}  \in  A_{\alpha_{2}}$, and $A_{\alpha_{2}} \cap B_{\alpha_{2}}  =  \phi$.
  \item $\alpha_{2} < \alpha_{0} < \mu$
    
    Then we have $a  =  (s_{\alpha})_{\alpha < \alpha_{2}} \cdot (a_{\alpha})_{\alpha_{2} \leq a < \mu}$.
    $c' <' a$ and $\alpha_{0} > \alpha_{2}$ implies that for all $\alpha < \alpha_{0}$, we have $c_{\alpha}'  =  a_{\alpha}$ and $a_{\alpha_{2}}  =  c_{\alpha_{2}}  \notin X_{\alpha_{2}}$, contradicting $a  \in  X$. Thus, $a  \in \, ]\leftarrow, c' [ \, \cap X$ and  we can conclude that $A  \subseteq \, ]\leftarrow, c' [ \, \cap X$. 
  \end{enumerate}
    
  Conversely, let $x  =  (x_{\alpha})_{\alpha < \mu} \in \, ] \leftarrow, c' [ \, \cap X$, then $x  <'  c'  =  (c_{\alpha}')_{\alpha < \mu}$. Let $\beta_{0} < \mu$ be the smallest ordinal such that $x_{\beta_{0}} <_{\beta_{0}}^+ c_{\beta_0}'$. Then we have the following cases:
  
    \begin{enumerate}
    \item $\beta_{0} < \alpha_{2}$
      
      Then $x  =  (s_{\alpha})_{\alpha < \beta_{0}} \cdot x_{\beta_{0}} \cdot (x_{\alpha})_{\beta_{0} < \alpha < \mu}$ and $x_{\beta_{0}} <_{\beta_0} c_{\beta_0}'$ (Note here that for $\beta_0 < \alpha_2$, $c_{\beta_0}'$ is a point in $X_{\beta_0}$, hence $x_{\beta_{0}} <_{\beta_0} c_{\beta_0}'$ iff $x_{\beta_{0}} <_{\beta_0}^+ c_{\beta_0}'$. So in this case we use $<_{\beta_0}$ instead of $<_{\beta_0}^+$). By contradiction, say $x  \in  B$, then $\beta_{0} < \alpha_{2} $ implies $A_{\beta_{0}} \cap B_{\beta_{0}}  =  \{s_{\beta_{0}} \}$ and $c_{\beta_{0}}'  =  s_{\beta_{0}}$. $x  =  (s_{\alpha})_{\alpha < \beta_{0}} \cdot x_{\beta_{0}} \cdot (x_{\alpha})_{\beta_{0} < \alpha < \mu}$ implies $x_{\beta_{0}}  \in  B_{\beta_{0}}$, but by hypothesis, since $x <^+ c' $, we have $x_{\beta_{0}} <_{\beta_0} s_{\beta_{0}}$ which implies $x_{\beta_{0}} \notin B_{\beta_{0}}$, contradiction. Therefore, we have $x \in A$.

    \item $\beta_{0}  =  \alpha_{2}$
      
      Then $x  =  (s_{\alpha})_{\alpha < \alpha_{2}} \cdot x_{\alpha_{2}} \cdot (x_{\alpha})_{\alpha_{2} < \alpha < \mu}$ and $x_{\alpha_{2}} <_{\alpha_2}^+ c_{\alpha_{2}}'  =  c_{\alpha_{2}}$. Suppose $x  \in  B$, then by definition of $B_{\alpha_{2}}$, $c_{\alpha_{2}} <_{\alpha_2}^+ x_{\alpha_{2}}$, contradiction. Hence, $x  \in  A$.

    \item $\alpha_{2} < \beta_{0} < \mu$
      
      Then $x  =  (s_{\alpha})_{\alpha < \alpha_{2}} \cdot (x_{\alpha})_{\alpha_{2} \leq \alpha < \mu}$. By hypothesis, $x <' c'  =  (s_{\alpha})_{\alpha < \alpha_{2}} \cdot c_{\alpha_{2}} \cdot (c_{\alpha}')_{\alpha_{2} < \alpha < \mu}$. Since $\beta_{0}$ is the smallest ordinal such that $x_{\beta_{0}} <_{\beta_{0}}^+ c_{\beta_{0}}'$ and $\alpha_{2} < \beta_{0}$, we have for $\alpha < \beta_{0}$, $x_{\alpha}  =  c_{\alpha}'$ and $x_{\alpha_{2}}  =  c_{\alpha_{2}}  \notin  X_{\alpha_{2}}$, contradiction. 
      \end{enumerate}

      Therefore, we can conclude that $x  \in  A$. 
   

    \item $A_{\alpha_{2}}$ has the maximal point and $B_{\alpha_{2}}$ has no minimal point.
      Let $a_{\alpha_{2}}'$ be the maximal point in $A_{\alpha_{2}}$. First of all, if $\mu = \alpha_2 + 1$, then the point $(s_{\alpha})_{\alpha < \alpha_2} \cdot a_{\alpha_2}'$ is the maximal point for $A$, contradicting that $(A,B)$ is a gap in $X$. Hence there exists an ordinal $\alpha$ with $\alpha_2 < \alpha < \mu$.
      
    Now we prove that there exists $\alpha'$ such that $\alpha_{2} < \alpha' < \mu$ and $X_{\alpha'}$ has no right end point $r_{\alpha'}$. Suppose not, say for every $\alpha_2 < \alpha < \mu$, $X_{\alpha}$ has the right end point $r_{\alpha}$. Consider the point $y = (y_{\alpha})_{\alpha < \mu} = (s_{\alpha})_{\alpha < \alpha_{2}} \cdot a_{\alpha_{2}}' \cdot (r_{\alpha})_{\alpha_{2} < \alpha < \mu}  \in  X$.

    Suppose $y \in A$, then since $A$ has no maximal point, there exists $a \in  A$ such that $y < a$. Let $a  =  (a_{\alpha})_{\alpha < \mu}$ and $\beta_{0}$ be the smallest ordinal such that $y_{\beta_{0}} <_{\beta_{0}} a_{\beta_{0}}$.
      \begin{enumerate}
      \item $\beta_{0} < \alpha_{2}$
        
        Then we have $y_{\beta_{0}}  =  s_{\beta_{0}} <_{\beta_{0}} a_{\beta_{0}}$, but $A_{\beta_{0}} \cap B_{\beta_{0}}  =  \{ s_{\beta_{0}} \}$. Consequently, $a_{\beta_{0}}  \in  B_{\beta_{0}}$ gives a  contradiction.
      \item $\beta_{0}  =  \alpha_{2}$
        
        Then there exists $a_{\alpha_{2}}  \in  X_{\alpha_{2}}$ such that $a_{\alpha_{2}}' <_{\alpha_{2}} a_{\alpha_{2}}$ and $a_{\alpha} = s_{\alpha}$ for $\alpha < \alpha_2$. Therefore, by the definition of $A_{\alpha_2}$, we have $a_{\alpha_{2}}  \in  A_{\alpha_{2}}$, contradicting that $a_{\alpha_{2}}'$ is the maximal point in $A_{\alpha_{2}}$.

      \item $\alpha_{2} < \beta_{0} < \mu$
        
        Impossible because $r_{\alpha}$ are the right end points for $X_{\alpha}$ for all $\alpha_{2} < \alpha < \mu$.
      \end{enumerate}
      Now, suppose that $y  \in  B$. Then by the definition of $B_{\alpha_{2}}$, we have $a_{\alpha_{2}}'  \in  B_{\alpha_{2}}$, giving a contradiction since $A_{\alpha_{2}} \cap B_{\alpha_{2}}  =  \phi$. Then the proof is done. Next, without loss of generality, let $\alpha_2 < \alpha' < \mu$ be the smallest ordinal such that $X_{\alpha'}$ has the right end gap $c_{\alpha'}$, and we define $c' =(c'_{\alpha})_{\alpha < \mu}$ in $\LP_{\alpha < \mu}X^+_{\alpha}$ where $c_{\alpha}' = s_{\alpha}$ for $\alpha < \alpha_2$, $c_{\alpha_2}' = a_{\alpha_2}'$, $c_{\alpha}' = r_{\alpha}$ for $\alpha_2 < \alpha < \alpha'$, $c_{\alpha'}' = c_{\alpha'}$ and finally $c_{\alpha}'$ are arbitrary points in $X_{\alpha}$ for $\alpha' < \alpha < \mu$.

      Then we show that $A  = \, ] \leftarrow , c' [ \, \cap X$ and $B  = \, ] c' , \rightarrow [ \, \cap X$.

      $(\Rightarrow)$  Let $a  \in  A$ and $a  =  (a_{\alpha})_{\alpha < \mu}$. Suppose $c' <' a$, let $\alpha_{0}$ be the smallest ordinal such that $c_{\alpha_{0}}' <_{\alpha_{0}}^+ a_{\alpha_{0}}$.
      \begin{enumerate}
      \item $\alpha_{0} < \alpha_{2}$
        
        Then for all $\alpha < \alpha_{0}$, we have $a_{\alpha}  =  c_{\alpha}'  =  s_{\alpha}$. Therefore, we have $a  =  (s_{\alpha})_{\alpha < \alpha_{0}} \cdot (a_{\alpha})_{\alpha_{0} \leq \alpha < \mu}  \in  A$. By the definition of $A_{\alpha_{0}}$, we can conclude that $a_{\alpha_{0}}  \in  A_{\alpha_{0}}$. But $\alpha_{0} < \alpha_{2}$ implies $c_{\alpha_{0}}'  =  s_{\alpha_{0}}$. Consequently, we have $s_{\alpha_{0}} <_{\alpha_0} a_{\alpha_{0}}  \in  A_{\alpha_{0}}$, contradicting that $s_{\alpha_{0}}$ is the maximal point for $A_{\alpha_{0}}$.
        
      \item $\alpha_{0}  =  \alpha_{2}$
        
        Then $a  =  (s_{\alpha})_{\alpha < \alpha_{2}} \cdot a_{\alpha_{2}} \cdot (a_{\alpha})_{\alpha_{2} < \alpha < \mu} \in  A$ and $c_{\alpha_{2}}' <_{\alpha_{2}}^+ a_{\alpha_{2}}$ where $c_{\alpha_{2}}'  =  a_{\alpha_{2}}'$. By the definition of $A_{\alpha_{2}}$, we have $a_{\alpha_{2}}  \in  A_{\alpha_{2}}$ and $a_{\alpha_{2}}' <_{\alpha_2} a_{\alpha_{2}}$ contradicting that $a_{\alpha_{2}}'$ is the maximal point in $A_{\alpha_{2}}$.
      \item $\alpha_{2} < \alpha_{0} < \alpha'$
        
        This is impossible because $r_{\alpha}$ are the right end points of $X_{\alpha}$ for all $\alpha_{2} < \alpha < \alpha'$.
      \item $\alpha_{0}  =  \alpha'$
        
        This is again impossible since $a  \in  X$.
      \item $\alpha_{0} > \alpha'$
        
        Impossible since for all $a_{\alpha'}  \in  X_{\alpha'}$ we have $a_{\alpha'} <_{\alpha'}^+ c_{\alpha'}$.
      \end{enumerate}

      Therefore, we can conclude that $a <' c'$.

      Conversely, let $x  \in \, ] \leftarrow, c' [ \, \cap X$ and $x  =  (x_{\alpha})_{\alpha < \mu}$, then we have $x <' c'$.

      Suppose $x  \in  B$. Let $\alpha_{0}$ be the smallest ordinal such that $x_{\alpha_{0}} <_{\alpha_{0}}^+ c_{\alpha_{0}}'$.
      \begin{enumerate}
      \item $\alpha_{0} < \alpha_{2}$
        
        Then $x  =  (s_{\alpha})_{\alpha < \alpha_{0}} \cdot (x_{\alpha})_{\alpha_{0} \leq \alpha < \mu}$. By assumption, $x  \in  B$, then by the definition of $B_{\alpha_{0}}$, we have $x_{\alpha_{0}}  \in  B_{\alpha_{0}}$, but $x_{\alpha_{0}} <_{\alpha_0} c_{\alpha_{0}}'  =  s_{\alpha_{0}}$ contradicting that $s_{\alpha_{0}}$ is the minimal point for $B_{\alpha_{0}}$.
      \item $\alpha_{0}  =  \alpha_{2}$
        
        Then we have $x  =  (s_{\alpha})_{\alpha < \alpha_{2}} \cdot x_{\alpha_{2}} \cdot (x_{\alpha})_{\alpha_{2} < \alpha < \mu}  \in  B$.

        Hence. $x_{\alpha_{2}}  \in  B_{\alpha_{2}}$, but $x_{\alpha_{2}} <_{\alpha_2} c_{\alpha_{2}}'  =  a_{\alpha_{2}}'  \in  A_{\alpha_{2}}$ which implies that $x_{\alpha_{2}}  \in  A_{\alpha_{2}}$, contradicting that $A_{\alpha_{2}} \cap B_{\alpha_{2}}  =  \phi$.
      \item $\alpha_{2} < \alpha_{0} < \alpha'$
        
        We have $x  =  (s_{\alpha})_{\alpha < \alpha_{2}} \cdot a_{\alpha_{2}}' \cdot (x_{\alpha})_{\alpha_{2} < \alpha < \alpha_0} \cdot x_{\alpha_{0}} \cdot (x_{\alpha})_{\alpha_{0} < \alpha < \mu}  \in  B$. By the definition of $B_{\alpha_{2}}$, we have $r_{\alpha_{2}}  \in  B_{\alpha_{2}}$. Contradiction.
        \item $\alpha_{0}  =  \alpha'$
          
         Again, since $x \in B$, then we have $a_{\alpha_2}' \in B$, a contradiction.
        \item $\alpha_{0} > \alpha'$
          
          Impossible since $c_{\alpha'}$ is the right end gap for $X_{\alpha'}$
        \end{enumerate}

        Therefore, we can conclude $A  = \, ] \leftarrow, c' [ \, \cap X$.

        Since $X  =  X \cap X'$, we get $B  =  X \backslash A  =  (X' \backslash \, ] \leftarrow, c' [ ) \cap  X  = \, ] c', \rightarrow [ \, \cap \, X$.

        For the remaining two cases, we can apply similar technique to prove them.
    \end{enumerate}
\end{proof}

\begin{remark}
It can be seen from Theorem \ref{lem:lgap:1} that every element $x = (x_{\alpha})_{\alpha < \mu} \in \LP_{\alpha < \mu}X_{\alpha}$ defines an embedding $i_x : X^+ \to \LP_{\alpha < \mu}X_{\alpha < \mu}^+$ in the following way. For a gap $c = (A,B)$ in $X$ let $i_x(c) = (c_{\alpha})_{\alpha < \mu}$, where $c_{\alpha_0}$ is a gap in $X_{\alpha_0}$, $c_{\alpha} \in X_{\alpha}$ for $\alpha \neq \alpha_0$ and $c_{\alpha} = x_{\alpha}$ for $\alpha > \alpha_0$.
\end{remark}

\begin{lem} \label{lem:lgap:2}
Let $X$, $Y$ be LOTS, $x \in X$, $y \in Y$, and $c_{X}$,$c_{Y}$ be gaps
of X and Y respectively. Then
\begin{enumerate}
\item $(c_{X},y)$ and $(c_{X},c_{Y})$ give a gap in $X \cdot Y$.
\item If $c_{Y}$ is not an end gap for $Y$, then $(x,c_{Y})$ gives a gap in $X \cdot Y$.
\item If $c_{Y}$ is the left end gap in $Y$ and $x$ has no left neighbour point in $X$, then $(x,c_{Y})$ is a gap for $X \cdot Y$.
\item If $c_{Y}$ is the left end gap in $Y$, $x$ has the left neighbour point $x'$ in $X$, and $Y$ has no right end point, then  $(x,c_{Y})$ is a gap for $X \cdot Y$.
\item If $c_{Y}$ is the right end gap in $Y$ and $x$ has no right neighbour point in $X$, then $(x,c_{Y})$ is a gap for $X \cdot Y$.
\item If $c_{Y}$ is the right end gap in $Y$, $x$ has the right neighbour point $x'$ in $X$, and $Y$ has no left end point, then $(x,c_{Y})$ is a gap for $X \cdot Y$.
\end{enumerate}
\end{lem}

\begin{proof}


  \begin{enumerate}
  \item Let $A  = \, ] \leftarrow, (c_{X},y) [ \, \cap X \cdot Y$ and $B  = \, ](c_{X},y), \rightarrow [ \, \cap X \cdot Y $.
    
    If $c_{X}$ is the left end gap for $X$, then $A  =  \phi$. Let $b = (x_{1},y_{1})  \in X \cdot Y$. Since $c_{X}$ is the left end gap for $X$, there exists $x'  \in  X$ such that $x' <_{X} x_{1}$, then the point $b'  =  (x',y_{1}) < b$ and $b'  \in  B$, hence $B$ has no minimal point. It is obvious that $B$ is open in $X \cdot Y$. Hence $c  =  (c_{X},y)  =  (A,B) $ is a gap for $X \cdot Y$. Similarly, we can show that if $c_{X}$ is the right end gap, then $(c_{X}, y)$ also gives a gap in $X \cdot Y$.
    Now, suppose that $c_{X}$ is not an end gap, and let $a  =  (a_{X}, a_{Y})  \in  A$ and $b  =  (b_{X},b_{Y})  \in  B$. Then we have there exists $a_{X}'  \in  X$, $b_{X}'  \in  X$ such that $a_{X} <_{X} a_{X}'$ and $b_{X}' <_{X} b_{X}$, therefore we have $a < a'  =  (a_{X}',a_{Y})$ and $b'  =  (b_{X}',b_{Y}) < b$, which implies that $A$ has no maximal point and $B$ has no minimal point, hence $(A,B)$ is a gap for $X \cdot Y$. The same proof applies for $(c_{X},c_{Y})$
    
  \item Say $c_{Y}$ is not an end gap for $Y$. Let $A  = \,  ] \leftarrow, (x,c_{Y}) [ \, \cap X \cdot Y$ and $B  = \, ] (x,c_{Y}), \rightarrow [ \, \cap X \cdot Y$.
    
    Since $c_{Y}$ is not an end gap, for all $y_{1},y_{2}  \in  Y$ such that $y_{1} <_{Y}^+ c_{Y}$ and $ c_{Y} <_{Y}^+ y_{2} $, there exist $y_{1}', y_{2}'  \in  Y$ such that $y_{1} <_{Y}^+ y_{1}' <_{Y}^+ c_{Y}$ and $c_{Y} <_{Y}^+ y_{2}' <_{Y}^+ y_{2}$, hence we have $y_{1} <_{Y} y_{1}'$ and $y_{2}' <_{Y} y_{2}$.
    Therefore, for any $a  =  (a_{X},a_{Y})  \in  A$ and $b  =  (b_{X},b_{Y})  \in  B$, there exist $a'  =  (a_{X}, a_{Y}')  \in  A$ and $b'  =  (b_{X},b_{Y}')  \in B$ such that $a < a'$ and $b' < b$.
    Hence, we have that $A,B$ are both open and $A$ has no maximal point and $B$ has no minimal point, therefore $(x,c_{Y})$ gives a gap for $X \cdot Y$.
  \item Say $c_{Y}$ is the left end gap in $Y$, and $x  \in  X$ has no left neighbour point and let $A  = \,  ] \leftarrow, (x,c_{Y}) [ \, \cap X \cdot Y$ and $B  = \, ] (x , c_{Y}), \rightarrow [ \, \cap X \cdot Y$. Then by the above proof, we know that $B$ has no minimal point and is open in $X \cdot Y$.
    
    Now, we shall proof that $A$ has no maximal point.
    
    Let $a  =  (a_{X}, a_{Y})  \in  A$, then $a_{X} <_{X} x$. Since $x$ has no left neighbour point in $X$, there exists $a_{X}'  \in  A_{X}$ such that $a_{X} <_{X} a_{X}' <_{X} x$, hence for the point $a'  =  (a_{X}', a_{Y})$, we have $a < a'$ and $a' <' c = (x, c_{Y})$, which implies $a'  \in  A$. Therefore $A$ has no maximal point, so that $(x,c_{Y})$ gives a gap for $X \cdot Y$.
  \item Say $c_{Y}$ is the left end gap in $Y$, $x  \in  X$ has the left neighbour point $x'$ and $Y$ has no right end point, i.e. $Y$ has a right end gap.
    
    Let $A  = \,  ] \leftarrow, (x,c_{Y}) [ \, \cap X \cdot Y$ and $B  = \, ] (x,c_{Y}), \rightarrow [ \, \cap X \cdot Y$.
    Again, $B$ is open in $X \cdot Y$ and it has no minimal point.
    Without loss of generality, let $a  =  (x', a_{Y})  \in  A$, since $Y$ has no right end point, there exists $y'  \in  Y$ such that $a_{Y} <_{Y} y'$. Now consider the point $a'  =  (x', y')  \in  X \cdot Y$, we have $a < a'$ and $a' <' (x, c_{Y})$ which implies $a' \in  A$.

    Therefore, $A$ has no maximal point and it is open in $X \cdot Y$. Hence $(x, c_{Y})$ gives a gap in $X \cdot Y$
    \item For the last two, we can apply similar proof.
  \end{enumerate}
\end{proof}

\begin{remark}\label{remk:lem:lgap:2}
  It is not difficult to see that a gap $(A,B)$ in $X \cdot Y$ must have one of the form listed in (1) - (6) of Lemma \ref{lem:lgap:2}.
\end{remark}

In Lemma \ref{lem:lgap:2}, we can see that end gaps in one space do not necessarily give gaps in the lexicographic product. Let us see a counter example.
\begin{eg}
  Let $X  =  [ 0,\omega_0 ] \cdot ] 0,\omega_0 ]$, then consider the point(not in $X$) (1,0). This point does not give a gap in $X$. Because
  $A  = \, ] \leftarrow, (1,0) [ $ has the right end point $(0,\omega_0)$.

  However, if $X  =  [ 0,\omega_0 ] \cdot ] 0,\omega_0 [$, then the point (1,0) gives a gap in $X$.
\end{eg}


\begin{lem} \label{lem:lgap:3}
  Let $X_{\alpha}$ be LOTS for all $\alpha < \mu$, where $\mu$ is an ordinal number, $(c_{\alpha})_{\alpha < \mu}$ be a point in $\LP_{\alpha < \mu}X_{\alpha}^+$. Let $\alpha_{0}$ be the smallest ordinal such that $c_{\alpha_0}$ is a point in $X_{\alpha_0}^+ \backslash X_{\alpha_0}$, i.e. $c_{\alpha_0}$ is a gap in $X_{\alpha_0}$. Then the pair of sets 
  $$A  = \, ] \leftarrow, (c_{\alpha})_{\alpha < \mu} [ \, \cap X$$
  $$B  = \, ](c_{\alpha})_{\alpha < \mu}, \rightarrow [ \, \cap X$$
  give a gap in $X$ if one of the following conditions is true.
  \begin{enumerate}
  \item $\alpha_{0}  =  0$.
  \item $\alpha_{0} \neq 0$ and $c_{\alpha_{0}}$ is not an end gap.
  \item $\alpha_{0} \neq 0$, $c_{\alpha_{0}}$ is the left end gap for $X_{\alpha_{0}}$, and the point $(c_{\alpha})_{\alpha < \alpha_{0}}  \in \LP_{\alpha < \alpha_{0}} X_{\alpha}$ has no left neighbour point.
  \item $\alpha_{0} \neq 0$, $c_{\alpha_{0}}$ is the left end gap for $X_{\alpha_{0}}$, and the point $(c_{\alpha})_{\alpha < \alpha_{0}}  \in \LP_{\alpha < \alpha_{0}} X_{\alpha}$ has the left neighbour point and the LOTS $\LP_{\alpha_{0} \leq \alpha < \mu} X_{\alpha}$ has no right end point.
  \item $\alpha_{0} \neq 0$, $c_{\alpha_{0}}$ is the right end gap for $X_{\alpha_{0}}$, and the point $(c_{\alpha})_{\alpha < \alpha_{0}}  \in \LP_{\alpha < \alpha_{0}} X_{\alpha}$ has no right neighbour point.
  \item $\alpha_{0} \neq 0$, $c_{\alpha_{0}}$ is the right end gap for $X_{\alpha_{0}}$, and the point $(c_{\alpha})_{\alpha < \alpha_{0}}  \in \LP_{\alpha < \alpha_{0}} X_{\alpha}$ has the right neighbour point and the LOTS  $\LP_{\alpha_{0} \leq \alpha < \mu} X_{\alpha}$ has no left end point.
  \end{enumerate}
\end{lem}

\begin{proof}
  Observe the fact that for any $\alpha_{0} < \mu$, $\LP_{\alpha < \mu} X_{\alpha}  =  \LP_{\alpha < \alpha_{0}} X_{\alpha} \cdot \LP_{\alpha_{0} \leq \alpha < \mu} X_{\alpha}$. Then the condition (n) in this lemma follows directly from the condition (n) in Lemma \ref{lem:lgap:2}.  
\end{proof}

\begin{remark}
  From Theorem \ref{lem:lgap:1}, we know that in the lexicographic product $X = \LP_{\alpha < \mu}X_{\alpha}$ of GO-spaces $X_{\alpha}$, if $X$ has a gap $c = (c_{\alpha})_{\alpha < \mu}$, then there exists a (smallest) ordinal $\alpha_0$ such that $c_{\alpha_0}$ is a gap in $X_{\alpha_0}$. It is not difficult to see that $c_{\alpha_0}$ has to satisfy one of the conditions in Lemma \ref{lem:lgap:3}.
\end{remark}
>From Lemma \ref{lem:lgap:3}, we can see that for a lexicographic product $X = \LP_{\alpha < \mu}X_{\alpha}$ , in general $X^+$ is not the same as $\LP_{\alpha < \mu}X_{\alpha}^+$. In fact we have $X^+ \subseteq \LP_{\alpha < \mu}X_{\alpha}^+$, but not the converse. However, if for all $\alpha < \mu$, $X_{\alpha}$ has no end gaps, then by Lemma \ref{lem:lgap:3}, we have $X^+ = \LP_{\alpha < \mu}X_{\alpha}^+$. 
  
\chapter{Compactness}
\section{Preliminaries}
\begin{defn}
  A space $X$ is said to be compact if it is Hausdorff and every open cover of $X$ has a finite subcover.
\end{defn}

\begin{thm}
  A Hausdorff space $X$ is compact if and only if every family of closed subsets which has the finite intersection property (f.i.p) has non-empty intersection.
\end{thm}

\begin{proof}
  Let $X$ be compact, $\mathcal{F}  =  \{F_{\alpha}\}_{\alpha \in \mathcal{A}}$ be a family of closed subsets with f.i.p and suppose $\bigcap \mathcal{F} = \phi$.

  Then we have $X \backslash \bigcap_{\alpha \in \mathcal{A}} F_{\alpha} = X = \bigcup_{\alpha  \in  \mathcal{A}}(X \backslash F_{\alpha})$, therefore the family $\{X \backslash F_{\alpha} \}_{\alpha \in \mathcal{A}}$ is an open cover for $X$. Hence, there exist $\alpha_{1} \, , \, \ldots \, \alpha_{n} \in \mathcal{A}$ such that $X \backslash F_{\alpha_{1}}, \, \ldots $ $ X \backslash F_{\alpha_{n}}$ covers $X$. Thus $X \backslash \bigcap_{i  =  1}^{n}F_{\alpha_{i}}  =  X  =  \bigcup_{i  =  1}^{n}(X \backslash F_{\alpha_{i}})$, so that $\bigcap_{i =  1}^{n}F_{\alpha_{i}}  = \phi$, contradicting that $\mathcal{F}$ has f.i.p.

  Conversely, suppose for all $\mathcal{F}$ with f.i.p, we have $\bigcap \mathcal{F} \neq \phi$ and assume that $X$ is not compact. Then there exists an open cover $\{U_{\alpha}\}_{\alpha  \in  \mathcal{A}}$  of $X$ that has no finite subcover.

  Now $X \backslash \bigcup_{\alpha \in \mathcal{A}}U_{\alpha}  =  \phi  =  \bigcap_{\alpha  \in  \mathcal{A}}(X \backslash U_{\alpha})$ implies $\mathcal{F}  =  \{X \backslash U_{\alpha} \}_{\alpha \in \mathcal{A}}$ does not have f.i.p. Therefore, there exists $\alpha_{1}\, , \, \ldots \, , \, \alpha_{n}$ such that $\bigcap_{i =  1}^{n}(X \backslash U_{\alpha_{i}})  =  \phi$. Then we have $X \backslash \bigcap_{i = 1}^n(X \backslash U_{\alpha_{i}})  =  X  =  \bigcup_{i  =  1}^{n} U_{\alpha_{i}}$, a contradiction.
\end{proof}

\begin{thm}
  Every closed subspace of a compact space is compact.
\end{thm}
\begin{proof}
  Let $X$ be a compact space and $C$ a subspace of $X$. Then $C$ is evidently Hausdorff. Let $\mathcal{C}$ be an open cover of $C$. Then for any $U  \in  \mathcal{C}$, there exists $U'$ open in $X$ such that $U  =  U' \cap C$. Hence, $\mathcal{C}'  =  \{U' \; | \; U \in \mathcal{C} \}\cup \{X \backslash C \}$ is an open cover for $X$.

  Since $X$ is compact, we have  $\mathcal{C}'$ has a finite subcover $\{U_{1}', \, U_{2}', \, \ldots \, U_{n}',$ $ X \backslash C \}$. Then $ U_1  =  U_{1}' \cap C, \, \ldots  \, , U_{n}  =  U_{n}'\cap C $ is a finite subcover for $\mathcal{C}$ and therefore $C$ is compact.
\end{proof}

\begin{thm}
  If a subspace $A$ of a space $X$ is compact, then for every family $\{U_{\alpha}\}_{\alpha  \in  \mathcal{A}}$ of open subsets of $X$ such that $A  \subseteq  \bigcup_{\alpha  \in  \mathcal{A}}U_{\alpha}$, there exists a finite set $\{ \alpha_{1}, \, \alpha_{2} \, \ldots \, \alpha_{n} \}  \subset \mathcal{A}$ such that $A  \subseteq  \bigcup_{i  =  1}^{n}U_{i}$. 
\end{thm}

\begin{proof}
  Suppose $A  \subseteq  \bigcup_{\alpha  \in  \mathcal{A}}U_{\alpha}$, then $\mathcal{C}  =  \{U_{\alpha} \cap A\}_{\alpha  \in  \mathcal{A}}$ is an open cover for $A$. Since $A$ is compact, there exist $\alpha_{1}, \, \ldots \, \alpha_{n} \in \mathcal{C}$ such that $A = \bigcup_{i  =  1}^{n}(U_{\alpha_{i}} \cap A)$. Therefore, $A  \subseteq  \bigcup_{i  =  1}^{n}U_{\alpha_{i}}$. 
\end{proof}

\begin{corollary}
  Let $X$ be a Hausdorff space and {$F_{1}$, $F_{2}$, $\ldots$ ,$F_{k}$} a family of closed subsets of $X$, then the subspace $F  =  \bigcup_{i  =  1}^{k}F_{i}$ is compact if and only if  all subspaces $F_{i}$ are compact. 
\end{corollary}
\begin{proof}
  ($\Rightarrow$) Suppose $F$ is compact, then since $F_{i}$ is closed in $X$, $F_{i}  =  F_{i} \cap F$ is closed in $F$. Therefore, $F_{i}$ is compact.

  ($\Leftarrow$) Suppose $F_{i}$ is compact, let $\mathcal{C}$ be an open cover for $F$, then there exists a finite subfamily $\mathcal{C}_{i}$ of $\mathcal{C}$ that covers $F_{i}$, then $\bigcup_{i  =  1}^{k} \mathcal{C}_{i}$ is a finite subcover for $F$.
\end{proof}

\begin{corollary}
  Let $U$ be an open subset of a space $X$. If a family $\{F_{\alpha} \}$ of closed subsets of $X$ contains at least one compact set, in particular, if $X$ is compact, and if $\bigcap_{\alpha  \in  \mathcal{A}} F_{\alpha}  \subseteq  U$, then there exists a finite set $\{\alpha_0$, $\alpha_{1}$, $\alpha_{2}$, $\ldots$, $\alpha_{k}\} \subset \mathcal{A}$ such that $\bigcap_{i  =  0}^{k} F_{\alpha_i}  \subseteq  U$.
\end{corollary}
\begin{proof}
  Suppose $F_{\alpha_{0}}$ is compact.% Consider $F_{\alpha_{0}} \cap X$, $\{ F_{\alpha} \cap F_{\alpha_{0}} \}_{\alpha  \in  \mathcal{A}}$, $U \cup F_{\alpha_{0}}$.
   Then $F_{\alpha} \cap F_{\alpha_{0}}$ is a compact subspace of $F_{\alpha_{0}}$ and $U \cap F_{\alpha_{0}}$ is open in $F_{\alpha_{0}}$.

  Now $\bigcap_{\alpha  \in  \mathcal{A}} F_{\alpha}  \subseteq  U$ implies $\bigcap_{\alpha  \in  \mathcal{A}}(F_{\alpha} \cap F_{\alpha_{0}})  \subseteq  F_{\alpha_{0}} \cap U$, so that $F_{\alpha_{0}} \backslash (F_{\alpha_{0}} \cap U)  \subseteq  \bigcup_{\alpha  \in  \mathcal{A}}\{F_{\alpha_{0}} \backslash (F_{\alpha} \cap F_{\alpha_{0}}) \}$. Since $F_{\alpha_{0}} \cap U$ is open in $F_{\alpha_0}$, it follows that  $F_{\alpha_{0}} \backslash (F_{\alpha_{0}} \cap U)$ is closed in $F_{\alpha_0}$, therefore it is compact.

  Thus, we can conclude that there exists $\alpha_{1}, \, \alpha_{2}, \, \ldots \, \alpha_{n}$ such that $F_{\alpha_{0}} \backslash (F_{\alpha_{0}} \cap U) \subseteq  \bigcup_{i  =  1}^{n} \{F_{\alpha_{0}} \backslash (F_{\alpha_{i}} \cap F_{\alpha_{0}}) \}$. Consequently, $\bigcap_{i  =  1}^{n} (F_{\alpha_{i}} \cap F_{\alpha_{0}})  \subseteq  F_{\alpha_{0}} \cap U$, and we have $\bigcap_{i  =  0}^{n} F_{\alpha_{i}}  \subseteq  U$.
\end{proof}

\begin{thm}
  Let $X$ be a regular space, $A$ a compact subspace of $X$ and $B  \subseteq  X$, such that $A \cap B =  \phi$ and $B$ is closed. Then there exist $U, \, V$ open in $X$ such that $A  \subseteq  U$, $B  \subseteq  V$ and $U \cap V  =  \phi$. 
\end{thm}
\begin{proof}
  Since $X$ is regular, we have for all $x  \in  A$, there exist $U_{x}, \, V_{x}$ open in $X$ such that $x  \in  U_{x}$, $B  \subseteq  V_{x}$ and $U_{x} \cap V_{x}  =  \phi$. Because $A  \subseteq  \bigcup_{x  \in  A}U_{x}$ and $A$ is compact, we have $A  \subseteq  \bigcup_{i  =  1}^{n}U_{x_{i}}$ for some $x_1$, $\ldots$, $x_n \in A$. Consider the corresponding $V_{x_{i}}$, then it is not difficult to see that $V = \bigcap_{i = 1}^n V_{x_i}$ and $U = \bigcup_{i = 1}^n U_{x_i}$ are open in $X$, $B \subseteq V$, $A \subseteq U$ and $U \cap V = \phi$.
\end{proof}

\begin{corollary}
  Let $X$ be a Hausdorff space, $A$, $B$ be compact subspaces and $A \cap B  =  \phi$, then there exist $U, \, V$ open in $X$ such that $A  \subseteq  U$, $B  \subseteq  V$ and $U \cap V  =  \phi$.
\end{corollary}
\begin{proof}
  $X$ is Hausdorff implies $X$ is $T_{1}$. Consider $\{x\}  \subseteq  B$, then $\{x\}$ is closed in $X$, and by the above theorem, there exist $U_{x}, \, V_{x}$ open in $X$ and $U_{x} \cap V_{x}  =  \phi$ such that $A  \subseteq  U_{x}$ and $\{x\}  \subseteq  V_{x}$. Since $B  \subseteq  \bigcup_{x  \in  B} V_{x}$ and $B$ is compact, there exist finitely many $ x_{1}, \, x_{2} \, \ldots \, x_{n} \in B$ such that $B  \subseteq  \bigcup_{i  =  1}^{n}V_{x_{i}}$.

  Then it is not difficult to see that $V = \bigcup_{i = 1}^n V_{x_i}$ and $U = \bigcap_{i = 1}^n U_{x_i}$ are open in $X$, $B \subseteq V$, $A \subseteq U$ and $U \cap V = \phi$.
\end{proof}

\begin{thm}
  If $A$ is a compact subspace of a Tychonoff space, then for every closed set $B \subseteq X$ such that $A \cap B  =  \phi$, there exists  a continuous function $f:\, X \to I$ such that $f(a)  =  0$, for all $a \in A$ and $f(b)  =  1$, for all $b  \in  B$. 
\end{thm}
\begin{proof}
  Since $X$ is Tychonoff, then for all $x  \in  A$, there exists a continuous function $f_{x}: \, X \to I$ such that $f_{x}(x)  =  0$ and $f(B)  \subseteq  \{1\}$. Consider the open subset $[0, \, \frac{1}{2})$ of $[0, \, 1]$. Then $x \in f^{-1}([0,\frac{1}{2}[)$ and the continuity of $f_x$ implies that $f^{-1}_x([0,\frac{1}{2}[)$ is open in $X$. 
  
  Since $A  \subseteq  \bigcup_{x \, \in \, A}f_{x}^{-1}([0, \, \frac{1}{2}))$ and $A$ is compact, there exist $x_{1}, \, \ldots \, , x_{n} \in A$ such that $A  \subseteq  \bigcup_{i  =  1}^{n}f_{x_{i}}^{-1}([0, \, \frac{1}{2}))$.

  Define $g: X \to I$ by $g(x)  =  \min{\{f_{x_{1}}(x), \, \ldots \, f_{x_{n}}(x)\}}$. Then $f_{x_{i}}(B)  \subseteq  \{1\}$ implies $g(B)  \subseteq  \{1\}$.

  Let $x  \in  A$, then there exist $k$ such that $x  \in  f_{x_{k}}^{-1}([0, \, \frac{1}{2}))$, so that $g(x) = \min{\{f_{x_{1}}, \, \ldots \, , f_{x_{n}}\}} < \frac{1}{2}$, and  therefore, $x  \in  g^{-1}([0, \, \frac{1}{2}))$. Consequently we have $A  \subseteq  g^{-1}([0, \, \frac{1}{2}))$.

  Finally, let $f(x)  =  2 \max{\{g(x) - \frac{1}{2}, 0\}}$. It is not difficult to see that for every $x \in A$, we have $f(x)  =  0$ and for every $x \in B$, $f(x)  =  1$ and $f$ is continuous.
\end{proof}

\begin{thm}
  Every compact subspace of a Hausdorff space $X$ is a closed subspace of $X$.
\end{thm}
\begin{proof}
  Let $A$ be a compact subspace of $X$, $x  \in  X \backslash A$ and $a  \in  A$. Since $X$ is Hausdorff, there exist $U_{a}, \, V_{a}$ disjoint open subsets of $X$ such that $x  \in U_{a}$ and $a  \in  V_{a}$. Now $A  \subseteq  \bigcup_{a  \in  A}V_{a}$ and $A$ is compact so that there exist $a_{1}, \; a_{2}, \; \ldots \; , \; a_{n} \in A$ such that $A  \subseteq  \bigcup_{i  =  1}^{n}V_{a_{i}}$. Next, consider $U = \bigcap_{i  =  1}^{n} U_{a_{i}}$ which is open in $X$, then we have $U \cap (\bigcup_{i  =  1}^{n}V_{a_{i}})  =  \phi$, $x \in U$ and $U$ does not intersect $A$. Therefore, we have for all $x  \in  X \backslash A$, $x  \notin  \overline{A}$, so that $A  =  \overline{A}$.
\end{proof}

\begin{thm}
  Every compact space is normal.
\end{thm}
\begin{proof}
  Let $X$ be compact, then by definition, $X$ is Hausdorff. Let $A, \, B$ be disjoint, closed subsets of $X$. Since $X$ is compact, $A$ and $B$ are also compact. Therefore, we have there exist $U, \, V$ disjoint open subsets of $X$ such that $A  \subseteq  U$ and $B  \subseteq  V$. Consequently $X$ is normal.
\end{proof}

\begin{thm}
  Let $X$ be a compact space, and $Y$ a Hausdorff space. If there exists a continuous surjective map $f:\, X \to Y$, then $Y$ is compact.
\end{thm}
\begin{proof}
  Let $\{U_{\alpha} \}_{\alpha  \in  \mathcal{A}}$ be an open cover for $Y$. Since $f$ is continuous, $f^{-1}(U_{\alpha})$ is open in $X$ for all $\alpha  \in  \mathcal{A}$. Since $\{f^{-1}(U_{\alpha}) \}_{\alpha  \in  \mathcal{A}}$ is an open cover for $X$ and $X$ is compact, there exist $\alpha_{1}, \, \alpha_{2}, \, \ldots \, , \alpha_{n} \in \mathcal{A}$ such that $X  =  \bigcup_{ i  =  1}^{n}f^{-1}(U_{\alpha_{i}})$. Since $f$ is onto, we have $Y  =  \bigcup_{i  =  1}^{n}U_{\alpha_{i}}$. Therefore, $Y$ is compact. 
\end{proof}

\begin{corollary}
  If $f: X \to Y$ is a continuous map from a compact space $X$ into a Hausdorff space $Y$, then for all $ A  \subseteq  X$, $\overline{f(A)}  =  f(\overline{A})$.
\end{corollary}
\begin{proof}
  Consider the continuous onto map $f|\overline{A}:\overline{A}  \to f(\overline{A})$. Since $X$ is compact, we have $\overline{A}$ is compact, and $Y$ is Hausdorff implies $f(\overline{A})$ is Hausdorff.  Therefore, $f(\overline{A})$ is compact, and we have that $f(\overline{A})$ is closed in $Y$. Since $f(A)  \subseteq  f(\overline{A})$, we have $\overline{f(A)}  \subseteq  f(\overline{A})$. Finally it follows from the continuity of $f$ that $f(\overline{A})  \subseteq  \overline{f(A)}$ and therefore $\overline{f(A)}  =  f(\overline{A})$.
\end{proof}

\begin{thm}
  Every continuous map of a compact space to a Hausdorff space is closed.
\end{thm}
\begin{proof}
  Let $X$ be compact, $Y$ be Hausdorff and $f: X \to Y$ be continuous. If $A \subseteq X$ is closed, then $A$ is a compact subspace of $X$. Hence $f(A)$ is a compact subspace of $Y$ so that $f(A)$ is closed in $Y$.
\end{proof}

\begin{thm}
  Every continuous one-to-one map of a compact space onto a Hausdorff space is a homeomorphism.
\end{thm}
\begin{proof}
  Consider $f^{-1}: Y \to X$ and let $F$ be a closed subspace of $X$. Since $f$ is bijective, we have $(f^{-1})^{-1}(F)  =  f(F)  \subseteq  Y$. Now $F$ is closed in $X$ and $X$ is compact, so that $F$ is a compact subspace. Therefore $f(F)$ is closed in $Y$ and hence, $f^{-1}$ is continuous. Consequently $f$ is a  homeomorphism. 
\end{proof}

\section{Compactness of GO-spaces}


We can now characterize compactness in GO-spaces in terms of gaps and pseudo gaps.
\begin{thm}\label{go:compactness:1}
  A GO-space is compact if and only if it has no gaps or pseudo gaps
\end{thm}
\begin{proof}
  $(\Rightarrow)$  Suppose $X$ is compact and has a gap $(A,B)$. Since $A$ has no maximal point, then the open cover $\{]\leftarrow,a[ \; | \; a \in A \} \cup \{B\}$ of $X$ has no finite subcover, giving a contradiction. Similarly, $X$ cannot have a pseudo gap.

  $(\Leftarrow)$ Suppose $X$ has no gaps or pseudo gaps. Then $X$ has no end gaps, and therefore, $X$  has the left end point $a$ and the right end point $b$. Now, without loss of generality, let $\mathcal{U}$ be an open cover consisting of convex subsets. Define $Y = \{x \in X$ | there exists finitely many $U_{1}$, $U_{2}$, $\ldots$ $U_{k} \in \mathcal{U}$ such that $a \in U_{1}$, $x \in U_{k}$ and $U_{i} \cup U_{i+1}$ is convex for $1 \leq i \leq k$ \}.

  If $Y \neq X$, then consider the pair $(Y, X \backslash Y)$.
  % Let $x \in X \backslash Y$, suppose there exists $y \in Y$ such that $x < y$, then we have there exists $i \leq k$ such that $x \in Y$, contradicting that $x \in X \backslash Y$.
  It is not difficult to see that for all $y \in Y$ and $x \in X \backslash Y$, we have $y < x$. Given  $y \in Y$, there exist $U_1$, $\ldots$, $U_k$ satisfying the definition of the set $Y$. In particular, $y \in U_k$ and for all $z \in U_k$, we have $z \in Y$, so that $Y$ is open in $X$.

Now, let $x \in X \backslash Y$, then there exists $U_{x} \in \mathcal{U}$ such that $x \in U_{x}$. Suppose $U_{x} \cap Y \neq \phi$, then there exists $y' \in U_{x} \cap Y$, which implies that $x \in Y$ by the definition of $Y$, a contradiction. Hence we have $U_{x} \cap Y = \phi$, so that $X \backslash Y$ is open in $X$.

Suppose that $Y$ has the right end point $y_r$ and $X \backslash Y$ has the left end point $x_{l}$. But $y_r \in Y$ implies $x_l \in Y$ by the definition of $Y$, a contradiction. 

Therefore, we can conclude that $(Y, X \backslash Y)$ forms a gap or pseudo gap. Thus $Y = X$ and hence it follows that $\mathcal{U}$ has a finite subcover.
\end{proof}

\begin{thm}
  Let $(X,<)$ be a linearly ordered set, $\lambda(<)$ and $\tau$ the linearly ordered topology and a GO-space topology defined on $X$ respectively. If $(X,<,\tau)$ is compact then $\lambda(<) = \tau$.
\end{thm}
\begin{proof}
  That $\lambda(<) \subseteq \tau$ follows from the definition of GO-space topology.

  Since $\lambda(<) \subseteq \tau$, the identity map $id: (X,\tau) \to (X,\lambda(<))$ is a continuous bijection and since $\tau$ is compact, it follows that $id$ is a homeomorphism. Consequently, $\tau = \lambda(<)$.
\end{proof}

\chapter{Lindel\"{o}f Spaces}
\section{Preliminaries}
\begin{defn}\label{lindelof:def:1}
  A space $X$ is said to be a Lindel\"{o}f space if it is regular and every open cover of $X$ has a countable subcover.
\end{defn}

\begin{defn}\label{lindelof:def:2}
A space is said to be second countable if it has a countable basis.
\end{defn}

\begin{thm}\label{lindelof:thm:3}
  Every regular second countable space is a Lindel\"{o}f space.
\end{thm}
\begin{proof}
  Let $X$ be a regular second countable space, $\mathcal{U} = \{U_{\alpha}\}_{\alpha \in \mathcal{A}}$ an open cover of $X$. Let $\mathcal{B} = \{B_i\}_{i \in \mathbb{N}}$ be a basis for $X$. Whenever possible, for each $i$, choose $\alpha_i$ such that $B_i \subseteq U_{\alpha_i}$. Then $\{U_{\alpha_i} \; | \; \text{for defined} \; \alpha_i\}$ is a countable subcover. Indeed, given $x \in X$, there exists $\alpha \in \mathcal{A}$ such that $x \in U_{\alpha}$ and there exists $i \in \mathbb{N}$ such that $x \in B_i \subseteq U_{\alpha}$. Thus $\alpha_i$ is defined and $x \in U_{\alpha_i}$.
\end{proof}

\begin{lem}\label{lindelof:lem:4}
  Let $X$ be a $T_1$ space and for every closed subset $F$ of $X$ and open subset $W$ such that $F \subseteq W$, there exists a sequence $W_1,W_2, \ldots$ of open subsets of $X$ such that $F \subseteq \bigcup_{i = 1}^{\infty}W_i$ and $\overline{W_i} \subseteq W$ for all $i \in \mathbb{N}$. Then $X$ is normal.
\end{lem}
\begin{proof}
  Let $A,B$ be disjoint closed subsets of $X$, take $F = A$ and $W = X \backslash B$, then $F \subseteq W$. By hypothesis, we can obtain a sequence $W_1,W_2, \ldots$ such that $W_i$ are open in $X$ and $\overline{W}_i \subseteq W$ for all $i = 1,2, \ldots$, and $A = F \subseteq \bigcup_{i = 1}^{\infty}W_i$. Therefore, $B \cap \overline{W_i} = \phi$ for all  $i = 1,2, \ldots$.

  Similarly, let $V = X \backslash A$, we can obtain a sequence $V_1, V_2, \ldots$ of open subsets of $X$ such that $B \subseteq \bigcup_{i = 1}^{\infty}V_i$ and $\overline{V}_i \subseteq V$ for all  $i = 1,2, \ldots$, so that $A \cap \overline{V_i} = \phi$ for all  $i = 1,2, \ldots$.

  Now, we define $$G_i = W_i \backslash \bigcup_{j \leq i}\overline{V_j} \; \text{and} \; H_i = V_i \backslash \bigcup_{j \leq i}\overline{W_j}$$. Since $W_i$ is open and $\bigcup_{j \leq i}\overline{V_j}$ is closed, we have $G_i = W_i \cap (X \backslash \bigcup_{j \leq i}\overline{V_j})$ is open in $X$. Similarly $H_i$ is open for $i = 1,2, \ldots$.

  Clearly, $A \subseteq G = \bigcup_{i = 1}^{\infty}G_i$ and $B \subseteq H = \bigcup_{i = 1}^{\infty}H_i$. Now we show $G \cap H = \phi$.

  $G_i \cap V_j = \phi$ for $j \leq i$ and $H_j \cap W_i = \phi$ for $i \leq j$, therefore we have $G_i \cap H_j = \phi$ for all $i ,j \in \mathbb{N}$. Consequently, $U \cap V = \phi$. 
\end{proof}

\begin{thm}\label{lindelof:thm:5}
  Every Lindel\"{o}f space is normal.
\end{thm}
\begin{proof}
  Let $F$ be a closed subset of $X$, $W$ an open subset of $X$ such that $F \subseteq W$. Since $X$ is regular, for $x \in F$, there exists an open neighbourhood $U_x$ of $x$ such that $x \in U_x \subseteq \overline{U_x} \subseteq W$. Then the open cover $\{X \backslash F \} \cup \{U_x \}_{x \in F}$ has a countable subcover $\{X \backslash F \} \cup \{U_{x_i}\}_{i = 1}^{\infty}$. Then we have $F \subseteq \bigcup_{i = 1}^{\infty}U_{x_i}$ and $\overline{U_{x_i}} \subseteq W$ for  $i = 1,2, \ldots$. Therefore by Lemma \ref{lindelof:lem:4}, $X$ is normal.
\end{proof}

\begin{defn}\label{lindelof:def:6}
  Let $X$ be a space, then a family $\mathcal{F} = \{F_{\alpha}\}_{\alpha \in \mathcal{A}}$ of subsets of $X$ has the countable intersection property if $\mathcal{F} \neq \phi$ and $\bigcap_{\alpha \in \mathcal{A}_0}F_{\alpha} \neq \phi$ for every countable $\mathcal{A}_0 \subseteq \mathcal{A}$.
\end{defn}

\begin{thm}\label{lindelof:thm:7}
  A regular space $X$ is Lindel\"{o}f if and only if every family of closed subsets of $X$ which has the countable intersection property has non-empty intersection.
\end{thm}
\begin{proof}
  Let $X$ be a Lindel\"{o}f space. Let $\mathcal{F} = \{F_{\alpha} \}_{\alpha \in \mathcal{A}}$ be a family of closed subsets of $X$ having the countable intersection property. Suppose $\bigcap \mathcal{F} = \phi$, then $X \backslash \mathcal{F} = \bigcup_{\alpha \in \mathcal{A}} X \backslash F_{\alpha} = X$. Therefore $\{X \backslash F\}_{\alpha \in \mathcal{A}}$ is an open cover for $X$, by Lindel\"{o}fness, it has a countable subcover $\{X \backslash F_i\}_{i \in \mathbb{N}}$, then we have $\bigcap_{i \in \mathbb{N}}F_i = \phi$, contradiction.

  Conversely, let $\mathcal{U} = \{U_{\alpha}\}_{\alpha \in \mathcal{A}}$ be an open cover of $X$. Define $F_{\alpha} = X \backslash U_{\alpha}$ and $\mathcal{F} = \{F_{\alpha}\}_{\alpha \in \mathcal{A}}$, then we have $\bigcap_{\alpha \in \mathcal{A}}F_{\alpha} = \phi$. By hypothesis, $\mathcal{F}$ does not have the countable intersection property, i.e. there exists a countable $\mathcal{A}_0 \subseteq \mathcal{A}$ such that $\bigcap_{\alpha \in \mathcal{A}_0}F_{\alpha} = \bigcap_{\alpha \in \mathcal{A}}X\backslash U_{\alpha} = \phi$. Hence $\{U_{\alpha}\}_{\alpha \in \mathcal{A}_0}$ is a countable subcover for $\mathcal{U}$. Therefore $X$ is Lindel\"{o}f.
\end{proof}

\begin{thm}\label{lindelof:thm:8}
  Every closed subspace of a Lindel\"{o}f space is a Lindel\"{o}f space.
\end{thm}
\begin{proof}
  Let $X$ be a Lindel\"{o}f space, $F$ a closed subspace of $X$. Let $\mathcal{U} = \{U_{\alpha}\}_{\alpha \in \mathcal{A}}$ be a cover of sets open in $F$, then there exists a family $\mathcal{U}' = \{U_{\alpha}'\}_{\alpha \in \mathcal{A}}$ of open subsets of $X$ such that $U_{\alpha} = F \cap U_{\alpha}'$. Consider the open cover $\{U_{\alpha}'\}_{\alpha \in \mathcal{A}} \cup \{ X \backslash F\}$ of $X$, since $X$ is Lindel\"{o}f, it has a countable subcover $\{U_{\alpha_i}'\}_{i \in \mathbb{N}} \cup \{ X \backslash F\}$, then $\{U_{\alpha_i}\}_{i \in \mathbb{N}}$ is a countable subcover of $\{U_{\alpha}\}_{\alpha \in \mathcal{A}}$. Therefore, $F$ is also Lindel\"{o}f. 
\end{proof}

\begin{thm}\label{lindelof:thm:9}
  Every regular space which can be represented as a countable union of Lindel\"{o}f subspaces is Lindel\"{o}f.
\end{thm}
\begin{proof}
  Let $X$ be a regular space satisfying the condition. Then let $\{C_i\}_{i \in \mathbb{N}}$ be a family of Lindel\"{o}f subspaces of $X$ such that $X = \bigcup_{i \in \mathbb{N}}C_i$. Let $\mathcal{U} = \{U_{\alpha} \}_{\alpha \in \mathcal{A}}$ be an open cover of $X$. For every $C_i$, consider the open cover $\mathcal{U}_i = \{C_i \cap U_{\alpha}\}_{\alpha \in \mathcal{A}}$, $\mathcal{U}_i$ has a countable subcover $\{C_i \cap U_{ij}\}_{j \in \mathbb{N}}$. Therefore $\{U_{ij}\}_{j \in \mathbb{N}, \, i \in \mathbb{N}}$ is a countable subcover for $\mathcal{U}$. Hence $X$ is Lindel\"{o}f. 
\end{proof}

\begin{thm}\label{lindelof:thm:10}
  Every open cover of a Lindel\"{o}f space has a locally finite open refinement.
\end{thm}
\begin{proof}
  Let $X$ be a Lindel\"{o}f space, $\mathcal{U}$ be an open cover of $X$. For $x \in X$, there exists an open set $V_x \in \mathcal{U}$ such that $x \in V_x$. Since $X$ is regular, there exists an open neighbourhood $U_x$ of $x$ such that $x \in U_x \subseteq \overline{U_x} \subseteq V_x$. Now, consider the open cover $\mathcal{V} = \{U_x\}_{x \in X}$, by Lindel\"{o}fness, $\mathcal{V}$ has a countable subcover $\{U_{x_i}\}_{i = 1}^{\infty}$.

  Define $$W_i = V_{x_i} \backslash \bigcup_{j = 1}^{i - 1}\overline{U_{x_j}} \; \text{for} \; i > 1 \; \text{and} \; W_1 = V_{x_1}.$$ Then $W_i$ is open in $X$ for all $i \in \mathbb{N}$, and the set $\{W_i\}_{i \in \mathbb{N}}$ is an open cover for $X$.

  Now, we show that the open cover is locally finite. For any $x \in X$, there exists $U_{x_i}$ such that $x \in U_{x_i}$. Then $U_{x_i} \cap W_{j} = \phi$ for $i < j$. 
\end{proof}

\begin{thm}\label{lindelof:thm:11}
  Let $X$ be a Lindel\"{o}f space, Y a regular space. If there exists a continuous map $f: X \to Y$ from $X$ onto $Y$, then $Y$ is also a Lindel\"{o}f space.
\end{thm}
\begin{proof}
  Let $\mathcal{V} = \{V_{\alpha}\}_{\alpha \in \mathcal{A}}$ be an open cover for $Y$. Then by the continuity of $f$, $\{f^{-1}(V_{\alpha})\}_{\alpha \in \mathcal{A}}$ is an open cover for $X$. Since $X$ is a Lindel\"{o}f space, it has a countable subcover $\{f^{-1}(V_{\alpha_i})\}_{i \in \mathbb{N}}$. Because $f$ is onto, we have $\{V_{\alpha_i}\}_{i \in \mathbb{N}}$ is a countable subcover of $\mathcal{V}$. 
\end{proof}

Recall that a map $f:X \to Y$ between two spaces $X$ and $Y$ is said to be closed if it is continuous and $f(F)$ is closed in $Y$ for every closed subset $F$ of $X$.
\begin{thm}\label{lindelof:thm:12}
  Let $X$ be a regular space, $Y$ be a space. If $f: X \to Y$ is a closed map and all fibres $f^{-1}(y)$ are  Lindel\"{o}f subspaces of $X$, then for every subspace $Z \subseteq Y$ that has the Lindel\"{o}f property, then $f^{-1}(Z)$ is a Lindel\"{o}f subspace of $X$.
\end{thm}

\begin{proof}
  Let $U_{\alpha}$ be open in $X$ for $\alpha \in \mathcal{A}$ such that $f^{-1}(Z) \subseteq \bigcup_{\alpha \in \mathcal{A}}U_{\alpha}$. For each $z \in Z$, the fibre $f^{-1}(z) \subseteq \bigcup_{\alpha \in \mathcal{A}_z}U_{\alpha}$ where $\mathcal{A}_z$ is a countable subset of $\mathcal{A}$, because every fibre $f^{-1}(z)$ has the Lindel\"{o}f property. Then we have $z \in Y \backslash f(X \backslash \bigcup_{\alpha \in \mathcal{A}_z}U_{\alpha})$, hence $Z \subseteq \bigcup_{z \in Z}(Y \backslash f(X \backslash \bigcup_{\alpha \in \mathcal{A}_z}U_{\alpha}))$. Since $f$ is a closed map, $Y \backslash f(X \backslash \bigcup_{\alpha \in \mathcal{A}_z}U_{\alpha})$ is open for all $z \in Z$. Therefore, by  Lindel\"{o}fness of $Z$, there exist a countable number of $z_i$'s such that $Z \subseteq \bigcup_{i = 1}^{\infty}(Y \backslash f(X \backslash \bigcup_{\alpha \in \mathcal{A}_{z_i}}U_{\alpha}))$.

  Hence, we have
  \begin{align*} f^{-1}(Z) \subseteq & \bigcup_{i = 1}^{\infty}f^{-1}(Y \backslash f(X \backslash \bigcup_{s \in S_{z_i}}U_{s})) \\
    & = \bigcup_{i = 1}^{\infty}(X \backslash  f^{-1}f(X \backslash \bigcup_{\alpha \in \mathcal{A}_{z_i}}U_{\alpha})) \subseteq  \bigcup_{i = 1}^{\infty}\bigcup_{\alpha \in \mathcal{A}_{z_i}}U_{\alpha}.
  \end{align*}
Then the set $\{U_{\alpha} \; | \; \alpha \in \mathcal{A}_{z_i} \}_{i \in \mathbb{N}}$ is a countable subcover of $\{U_{\alpha}\}_{\alpha \in \mathcal{A}}$. Hence, $f^{-1}(Z)$ has the Lindel\"{o}f property.
\end{proof}

\section{ Lindel\"{o}f property in GO-spaces}
\begin{thm}\label{lindelof:thm:13}
  Let $X$ be a GO-spaces. Then $X$ is a  Lindel\"{o}f space if and only if the following two conditions are satisfied:
  \begin{enumerate}
  \item Each discrete subset of $X$ contains at most countably many points.
  \item Each gap is a $Q-$gap and each pseudo gap is a $Q-$pseudo gap.
  \end{enumerate}
\end{thm}
\begin{proof}
  ($\Rightarrow$) Say $X$ is a  Lindel\"{o}f GO-space. Let $F$ be a discrete subset of $X$ and consider $x \in X \backslash F$, then there exists a convex open neighbourhood $U_x$ of $x$ such that $U_x \cap F = \phi$. Therefore, $F$ is closed, and since Lindel\"{o}f property is hereditary under closed subsets, $F$ is a Lindel\"{o}f subspace. For each $x \in F$, there exists an open neighbourhood $U_x$ such that $U_x \cap F = \{x\}$. Then $\mathcal{U} = \{\{x\}\}_{x \in F}$ is an open (in $F$) cover for $F$, hence it has a countable subcover. Therefore $F$ has at most countably many points.
  By Theorem \ref{lindelof:thm:10}, every Lindel\"{o}f space is paracompact (see Definition \ref{def:paracompactness}). Therefore by Theorem \ref{paracompact:go:2}, each gap of $X$ is a $Q-$gap and each pseudo gap of $X$ is a $Q-$pseudo gap.


  ($\Leftarrow$) By 2), we have $X$ is paracompact (see Theorem \ref{paracompact:go:2}). Consider an open cover $\mathcal{V}$ of $X$ which is locally finite. Choose any $x_0 \in X$, there exists an open neighbourhood $U_{x_0}$ of $x_0$ such that $U_{x_0}$ intersects finitely many elements $\{V(x_0)_{1},V(x_0)_{2}, \ldots , V(x_0)_{k_0} \}$ of $\mathcal{V}$ and define $U_{0} = \bigcup_{i = 1}^{k_0}V(x_0)_{i}$. Choose $x_1 \in X \backslash U_0$, then there exists an open neighbourhood $U_{x_1}$ of $x_1$ such that $U_{x_1}$ intersects finitely many elements  $V(x_1)_{1},V(x_1)_{2}, \ldots , V(x_1)_{k_1}$ of $\mathcal{V}$, and define $U_1 = \bigcup_{i = 1}^{k_1}V(x_1)_{i}$. By transfinite induction, for any ordinal number $\beta$, choose $x_{\beta} \in X \backslash \bigcup_{\alpha < \beta}U_{\alpha}$, then there exists an open neighbourhood $U_{x_{\beta}}$ of $x_{\beta}$ such that $U_{x_{\beta}}$ intersects finitely many elements $V(x_{\beta})_{1},V(x_{\beta})_{2}$, $\ldots$ ,$V(x_{\beta})_{k_{\beta}}$ of $\mathcal{V}$ and define $U_{\beta} = \bigcup_{i = 1}^{k_{\beta}}V(x_{\beta})_{i}$. We continue this process until an ordinal $\mu$ such that $X = \bigcup_{\alpha < \mu}U_{\alpha}$ and we obtain a set $D = \{x_{\alpha}\}_{\alpha < \mu}$.
 
  Now, we show that the set $D$ is discrete. Consider $x_{\alpha} \in D$, if $\alpha = 0$ then $U_{0}$ is the required neighbourhood of $x_0$, i.e. $U_0 \cap D = \{x_0\}$. For $\beta > 0$, there exists $V \in \mathcal{V}$ such that $x_{\beta} \in V$, clearly $V$ does not contain any $x_{\alpha}$ for $\alpha > \beta$. Suppose $x_{\alpha'} \in V$ where $\alpha' < \beta$, then by the construction we have $V \subseteq U_{\alpha'}$, contradiction. Therefore $V \cap D = \{x_{\beta}\}$.

  Now suppose $x \in X \backslash D$ and let $V \in \mathcal{V}$ be such that $x \in V$. Suppose $V \cap D \neq \phi$ and $x_{\beta} \in V \cap D$. As shown above, since $x_{\beta} \in V$, we conclude that $V \cap D = \{x_{\beta}\}$. 
%%%%%%%%%%% A valuable proof, but not used here  %%%%%%%%%%%%%%%
  % For $x \in X$ and $x \notin D$, if for all $V_x \in \mathcal{V}$, we have $V_x \cap D = \phi$, then the proof is done. If there exists an open subset $V_x \in \mathcal{V}$ such that there exists $\beta < \mu$ such that $x_{\beta}$ and $x$ are both in $V_x$, then $V_x \cap D = \{x_{\beta}\}$, 

   Hence $D$ is discrete. By hypothesis, $D$ contains at most countably many points. Therefore the family $\{V(x_{j})_i \; | \; x_j \in D \text{ and } i \leq k_j \text{ for each } k_j\}$ is a countable subcover for $\mathcal{V}$.

  Since $X$ is paracompact, we have that every open cover has an open locally finite refinement which covers $X$. Hence every open cover has a countable subcover, and $X$ is Lindel\"{o}f. 
\end{proof}


\begin{defn}\label{lindelof:def:13}
 Let $X$ be a space and $Y$ a subset of $X$. Then $Y$ is said to be relatively discrete if the induced topology on $Y$ is the discrete topology.
\end{defn}
\begin{lem}\label{lindelof:lem:13-1}
  In every GO-space $X$, one can find a relatively discrete cofinal subspace and a relatively discrete coinitial subspace.
\end{lem}
\begin{proof}
We give a proof for the cofinal case. If $X$ has a maximal point, then there is nothing to prove. Otherwise, by transfinite induction, one can construct a cofinal subset $\{x_{\alpha}\}_{\alpha < \mu}$ such that $x_{\alpha} < x_{\beta}$ for $\alpha < \beta < \mu$. Now, let $L = \{x_{\alpha} \; | \; \alpha < \mu, \; \alpha \text{ is a non-limit ordinal } \}$. Then $L$ is the required relatively discrete cofinal subset of $X$. 
\end{proof}

\begin{thm}\label{lindelof:thm:14}
  Let $X$ be a GO-space. Then $X$ is hereditarily Lindel\"{o}f if and only if each relatively discrete subset of $X$ contains at most countably many points.
\end{thm}
\begin{proof}
  ($\Rightarrow$) Evident, since each relatively discrete subset is a discrete space in the subspace topology.

  ($\Leftarrow$) Follows from Lemma \ref{lindelof:lem:13-1} and that every relatively discrete subset of $A \subseteq X$ is a relatively discrete subset of $X$.  
\end{proof}

\chapter{Countable Compactness}

\section{Preliminaries}
\begin{defn}
  A space $X$ is said to be countably compact if $X$ is Hausdorff and every countable open cover of $X$ has a finite subcover.
\end{defn}

\begin{thm}
  A space $X$ is compact if and only if it is a countably compact space with Lindel\"{o}f property.
\end{thm}
\begin{proof}
  ($\Rightarrow$) It follows directly from the definition of compactness.

  Conversely, say $X$ is a countably compact space with Lindel\"{o}f property. Then by  Lindel\"{o}f property, every open cover $\mathcal{U}$ of $X$ has a countable subcover, hence $\mathcal{U}$ has a finite subcover by countable compactness.
\end{proof}

\begin{thm}
  Every closed subspace of a countably compact space is countably compact.
\end{thm}
\begin{proof}
  Let $X$ be a countably compact space, $F$ a closed subspace of $X$. Let $\{V_{n}\}_{n \in \mathbb{N}}$ be a cover of $F$. For every $n \in \mathbb{N}$, there exists an open subset $U_n$ of $X$ such that $V_n = U_n \cap F$. Then $\{U_{n} \; | \; n \in \mathbb{N} \} \cup \{X \backslash F \}$ is a countable open cover of $X$. Therefore, it has a finite subcover $\{U_{n_i}\}_{i = 1}^{N} \cup \{X \backslash F \}$. Then  $\{U_{n_i} \cap F \}_{i = 1}^{N} = \{V_{n_i}\}_{i=1}^N$ is a finite subcover of $F$. Thus $F$ is a countably compact subspace.
\end{proof}

\begin{thm}
  For every Hausdorff space $X$, the following conditions are equivalent:
  \begin{enumerate}
  \item $X$ is countably compact.
  \item Every countable family of closed subsets of $X$ which has the finite intersection property has non-empty intersection.
  \item For every decreasing sequence $F_1 \supseteq F_2 \supseteq \ldots$ of non-empty closed subsets of $X$, the intersection $\bigcap_{n = 1}^{\infty}F_{i}$ is non-empty.
  \end{enumerate}
\end{thm}
\begin{proof}
  Suppose $X$ is countably compact and $\mathcal{F} = \{F_{i}\}_{i = 1}^{\infty}$  a family of closed subsets with f.i.p. Assume that $\bigcap \mathcal{F} = \phi$, then $X \backslash \bigcap \mathcal{F} = X$, thus $\{X \backslash F_i \}_{i = 1}^{\infty}$ is a countable open cover of $X$. By countable compactness, it has a finite subcover $\{X \backslash F_{i_j}\}_{j = 1}^{N}$, then we have $X \backslash \bigcap_{j = 1}^{N}F_{i_j} = X$, hence $\bigcap_{j = 1}^{N}F_{i_j} = \phi$, contradicting that $\mathcal{F}$ has the finite intersection property.

  $3)$ follows directly from $2)$.

  We now show $3)$ implies $1)$. Let $\mathcal{U} = \{U_i\}_{i = 1}^{\infty}$ be a countable open cover of $X$. Without loss of generality, assume that  for all $i \in \mathbb{N}$, $X \backslash U_i \neq \phi$. Define $F_i = X \backslash \bigcup_{j \leq i}U_j$ for $i \in \mathbb{N}$. If for all $i \in \mathbb{N}$, we have $F_i \neq \phi$, then by assumption we have $\bigcap_{i = 1}^{\infty}F_i  = X \backslash \bigcup_{i = 1}^{\infty}U_i \neq \phi$, contradicting that $\mathcal{U}$ is an open cover of $X$. Therefore there exists $N \in \mathbb{N}$ such that $F_N$ is empty, i.e. $\bigcup_{j \leq N}U_j = X$. Hence, $X$ is countably compact.  
\end{proof}

\begin{defn}
  Let $X$ be a space, then a collection $\mathcal{C}$ of subsets of $X$ is called locally finite if for all $x \in X$, there exists an open neighbourhood $U_{x}$ of $x$ such that $U_{x}$ intersects only finitely many elements of $\mathcal{C}$.
\end{defn}

\begin{thm}
  For every Hausdorff space $X$, the following conditions are equivalent:
  \begin{enumerate}
  \item $X$ is countably compact.
  \item Every locally finite family of non-empty subsets of $X$ is finite.
  \item Every locally finite family of one-point subsets of $X$ is finite.
  \item Every infinite subset of $X$ has an accumulation point.
  \item Every countably infinite subset of $X$ has an accumulation point.
  \end{enumerate}
\end{thm}
\begin{proof}
  1) implies 2). Suppose not, then there exists a locally finite family $\mathcal{L} = \{A_{s}\}_{s \in S}$ of non-empty subsets of $X$ such that $\mathcal{L}$ is infinite. Let $\mathcal{L}' = \{A_i\}_{i \in \mathbb{N}}$ be a subfamily of $\mathcal{L}$, hence $\mathcal{L}'$ is also locally finite. Now, define $\{F_i\}_{i = 1}^{\infty}$ such that $F_i = \bigcup_{j = i}^{\infty}\overline{A_j}$ for all $i = 1,2, \ldots$ . Since $\mathcal{L}'$ is a locally finite family, $F_i$ is closed for all $i \in \mathbb{N}$. Because $\mathcal{L}'$ is locally finite, for any $x \in X$, there exists an open neighbourhood $U_x$ of $x$ such that $U_x$ intersects at most finitely many elements $A_{n_1}$, $A_{n_2}$, $\ldots$, $A_{n_k}$ in $\mathcal{L}'$. Then let $n = \max{\{n_1,n_2,\ldots,n_k\}}$, thus we have $x \notin F_i$ for $i > n$. Then $\bigcap_{i = 1}^{\infty}F_i = \phi$, contradiction.

  2) implies 3) is obvious. We now show that 3) implies 4). Suppose not, let $A$ be an infinite subset of $X$ with no accumulation points, then for every $x \in X$, there exists an open neighbourhood $U_x$ of $x$ such that $(U_x\backslash \{x\}) \cap A = \phi$. Then the family $\{\{x\} \; | \; x \in A\}$ of sets is a locally finite family of one-point subsets of $X$. Contradiction. 4) implies 5) is obvious.

  Finally we show 5) implies 1). Suppose not, then there exists a decreasing sequence $F_i \supseteq F_2 \supseteq F_3 \cdots $ of non-empty closed subsets of $X$ such that $\bigcap_{i = 1}^{\infty}F_i = \phi$. Choose $x_i \in F_i$ and let $A = \{x_i\}_{i = 1}^{\infty}$. Then $A$ is infinite, otherwise, there exists a point $x_k \in A$ such that $x_k$ belongs to infinitely many $F_i$'s. Since the sequence of $F_i$'s is decreasing, $x_k \in \bigcap_{i = 1}^{\infty}F_i$. Now, we show that $A$ has no accumulation points. For $x \in X$, there exists $i$ such that $x \notin F_i$. Let $U = (X \backslash F_i) \backslash (\{x_j\}_{j = 1}^{i - 1} \backslash \{x\})$, then $U \backslash \{x\} \cap A = \phi$.    
\end{proof}


\section{Countable compactness of GO-spaces}
\begin{defn}
  A gap (or pseudo gap) $(A,B)$ is said to be countable if there is a strictly increasing sequence cofinal in $A$ or a strictly decreasing sequence coinitial in $B$.
\end{defn}
Let us begin by showing that from any sequence in a linearly ordered set, one can extract a monotonically increasing or decreasing subsequence. The proof is taken from \cite{HansWeber}.
\begin{lem}
  Every sequence in a linearly ordered set has a monotone subsequence.
\end{lem}
\begin{proof}
  Let $\{a_n\}_{n \in \mathbb{N}}$ be a sequence in a linearly ordered set $X$. If $C = \{a_n \;|\; n \in \mathbb{N}\}$ is finite, then $\{a_n\}_{n \in \mathbb{N}}$ has a subsequence which is constant. If $C$ is infinite and well-ordered, then one can inductively define natural numbers $k_1 < k_2 < \ldots$ by $a_{k_1} = \min \{C\}$ and $a_{k_{n+1}} = \min{\{C \backslash \{a_i \;|\; i \leq k_n \} \}}$, the subsequence $\{a_{k_n}\}_{n \in \mathbb{N}}$ is  increasing. If $C$ is not well-ordered and $M$ a subset of $C$ without minimal element, then one can inductively choose natural numbers $k_1 < k_2 < \ldots$ such that $a_{k_n} \in M$ and $a_{k_{n+1}} < \min{\{a_i \;|\; a_i \in M, i \leq k_n \}}$. The subsequence  $\{a_{k_n}\}_{n \in \mathbb{N}}$ is decreasing.
\end{proof}
\begin{thm}
  A GO-space is countably compact if and only if it has no countable gaps and no countable pseudo gaps
\end{thm}
\begin{proof}
  Let $X$ be a countably compact GO-space, suppose that $(A,B)$ is a countable pseudo gap in $X$. Without loss of generality, say there exists a strictly increasing sequence $\{a_{n}\}_{n \in \mathbb{N}}$ that is cofinal in $A$, hence $A$ has no right end point. Then the open cover $\{\,]\leftarrow, a_{n}[ \; | \; n \in \mathbb{N}\} \cup \{B\}$ has no finite subcover, contradicting that $X$ is countably compact.

  Conversely, suppose that $X$ has no countable gap or countable pseudo gap. By contradiction, say $X$ is not countably compact. Then there exists a countable open cover $\mathcal{U} = \{U_{i}\}_{i \in \mathbb{N}}$ of $X$ that has no finite subcover. Therefore, there is a sequence $\{x_k\}_{k \in \mathbb{N}}$ in $X$ such that $x_{k} \notin \bigcup_{i < k}U_{i}$ and $x_k$ is not an end point. One can assume that for all $k \in \mathbb{N}$, $U_k \nsubseteq \bigcup_{i < k }U_i$, we now extract from the sequence $\{x_k\}_{k \in \mathbb{N}}$ a monotonically increasing or decreasing sequence $\{x_{k_i}\}_{i \in \mathbb{N}}$. We may assume that $\{x_{k_i}\}_{i \in \mathbb{N}}$ is increasing. Then:
%Define $$Y = \{x_{k} \in X \; | \; x_{k} \text{ is not an end point } x_{k} \in U_{k} \text{ and }  x_{k} \notin \bigcup_{i < k}U_{i} \}.$$ One can assume that for all $k \in \mathbb{N}$, $U_k \nsubseteq \bigcup_{i < k }U_i$, we now extract from $Y$ a monotonically increasing or descreasing sequence $\{x_{k_i}\}_{i \in \mathbb{N}}$, then:
  \begin{enumerate}
  \item If it has no upper bound in $X$, then $(X, \phi)$ is an end gap and $x_{i}$ is strictly increasing and cofinal in $X$. Therefore the right end gap is countable. Contradiction. 
  \item If the sequence has an upper bound, then let $A = \bigcup_{i \in \mathbb{N}}\, ]\leftarrow, x_{i}[$ and $B = \{ x \in X \; | \; x \; \text{is an upper bound for} \; \{x_{k_i}\}_{i \in \mathbb{N}} \}$. Then we have $B = X \backslash A$. If $B$ has no left end point, then $B$ is open, and hence $(A,B)$ forms a countable gap. Contradiction.

    If $B$ has a left end point $b$ and $[b, \rightarrow [$ is open in $X$, then $(A,B)$ is a countable pseudo gap. Contradiction.

    If $B$ has a left end point $b$ and $[b, \rightarrow [$ is not open in $X$, then for every $U_{j}$ containing $b$, there exists a convex component $C_{j} \subseteq U_{j}$ such that $b \in C_{j}$. Then it is not difficult to see that for every $x \in C_{j}$ and every $i \in \mathbb{N}$,  $x > x_{k_i}$, contradicting that $b$ is the supremum.
    %However, by the definition of $\{x_{i}\}$, we have for all $x \in C_{j}$ and for all $i \in \mathbb{N}$,  $x > x_{i}$, contradicting that $b$ is the superemum.
  \end{enumerate}

  Therefore, we can find a countable gap in $X$, contradicting our assumption. The same argument applies for the case that $\{x_{k_i}\}_{i \in \mathbb{N}}$ is decreasing.
\end{proof}

\chapter{Paracompactness}

\section{Preliminaries}

Remember that for a space $X$, a collection $\mathcal{C}$ of subsets of $X$ is called locally finite if for all $x \in X$, there exists an open neighbourhood $U_{x}$ of $x$ such that $U_{x}$ intersects only finitely many elements of $\mathcal{C}$.


\begin{defn}\label{def:paracompactness}
  Let $X$ be a space, then $X$ is said to be paracompact if $X$ is Hausdorff and every open cover of $X$ has a locally finite open refinement that covers $X$.
\end{defn}

\begin{thm} \label{thm:paracompact:1}
  Let $X$ be a paracompact space and $A$, $B$ a pair of closed subsets of $X$. If for all $x  \in  B$, there exist open subsets $U_{x}, \, V_{x}$ of $X$ such that $A \subseteq  U_{x}$, $x  \in  V_{x}$ and $U_{x} \cap V_{x} = \phi$, then there exist disjoint open subsets $U, \, V$ of $X$ such that $A  \subseteq  U$, $B  \subseteq  V$.
\end{thm}
\begin{proof}
  For $x  \in  B$, let $V_{x}, U_x$ be open subsets of $X$ such that $A  \subseteq  U_{x}$, $x  \in  V_{x}$ and $U_{x} \cap V_{x}  =  \phi$. Then $\mathcal{C}  =  \{V_{x}\; | \;  x  \in  B\} \cup \{X \backslash B \}$ is an open cover for $X$. Since $X$ is paracompact, $\mathcal{C}$ has a locally finite open refinement $\mathcal{W}  =  \{W_{s}\}_{s  \in  \mathcal{S}}$.

  Define $\mathcal{S}_{0}  =  \{s  \in  \mathcal{S}\; | \; W_{s} \cap B  \neq  \phi \}$. For each $s \in \mathcal{S}_0$, there exists $V_x \in \mathcal{C}$ such that $W_s \subseteq V_x$ and $V_{x} \cap U_{x} = \phi$, hence $\overline{W_{s}}  \subseteq  \overline{V_{x}}$, thus $\overline{V_{x}} \cap U_{x}  =  \phi$ and we have $\overline{V_{x}} \cap A  =  \phi$.

  Hence, we have $A \cap \overline{W_{s}}  =  \phi$, for all $s  \in  \mathcal{S}_0$, and $B  \subseteq  \bigcup_{s  \in  \mathcal{S}_0}W_s$.

  $\mathcal{W}$ is locally finite implies $\overline{\bigcup_{s  \in  S_{0}}W_{s}}  =  \bigcup_{s \, \in \, S_{0}}\overline{W_{s}}$. Therefore $A \subseteq  X \backslash \bigcup_{s \, \in \, S_{0}}\overline{W_{s}}$ which is open in $X$.

    Define $U =  X \backslash \bigcup_{s \, \in \, S_{0}}\overline{W_{s}}$, $V =  \bigcup_{s \, \in \, S_{0}}W_{s}$. Then $A \subseteq U$, $B \subseteq V$ and $U, \, V$ are disjoint and open in $X$.
\end{proof}

\begin{thm} \label{thm:paracompact:2}
  Every paracompact space is normal.
\end{thm}
\begin{proof}
  Let $X$ be a paracompact space, then by definition, it is Hausdorff. Let $B$ be a closed subset of $X$ and $x  \notin  B$, then $\{x\}$ is closed in $X$ and for all $b  \in  B$, there exist open subsets $U_{b}, \, V_{b}$ of $X$ such that $\{x\}  \subseteq  U_{b}$, $b  \in V_{b}$ and $U_{b} \cap V_{b}  =  \phi$. Hence, by Theorem \ref{thm:paracompact:1}, there exist disjoint open subsets $U'$, $V'$ of $X$ such that $\{x\}  \subseteq U'$ and $B  \subseteq  V'$, and therefore $X$ is regular. Let $A$, $B$ be disjoint closed subsets of $X$, then $X$ is regular implies for all $x  \in  B$, there exist disjoint open subsets $U_{x}, \, V_{x}$ of $X$ such that $A  \subseteq  U_{x}$, $x  \in  V_{x}$. Then again by Theorem \ref{thm:paracompact:1}, there exist disjoint open subsets $U$, $V$ of $X$ such that $A  \subseteq  U$, $B  \subseteq  V$.
\end{proof}

\begin{defn}
  A family $\{f_{s}\}_{s  \in  \mathcal{S}}$ of continuous functions from a space $X$ into the closed unit interval $I$ is called a partition of unity on the space $X$ if $$\sum_{s  \in  \mathcal{S}} f_{s}(x)  =  1 \;\;\;\; \text{for all } x  \in X.$$
\end{defn}


\begin{defn}
  A partition of unity $\{f_{s}\}_{s \in \mathcal{S}}$ on a space $X$ is locally finite if the open cover $\{f_{s}^{-1}(\,]0,1])\}_{s \in \mathcal{S}}$ of $X$ is locally finite.
\end{defn}

\begin{defn}
  A partition of unity $\{f_{s}\}_{s \in \mathcal{S}}$ on a space $X$ is subordinated to a cover $\mathcal{C}$ of $X$ if the cover $\{f_{s}^{-1}(\,]0,1])\}_{s \in \mathcal{S}}$ of $X$ is a refinement of $\mathcal{C}$. 
\end{defn}

\begin{lem} \label{lem:paracompact:3}
  If every open cover of a regular space $X$ has a locally finite refinement (not necessarily open), then for every open cover $\{U_{s}\}_{s \in \mathcal{S}}$ of the space $X$, there exists a closed locally finite cover $\{F_{s}\}_{s \in \mathcal{S}}$ of $X$ such that for all $s \in \mathcal{S}$, $F_{s} \subseteq U_{s}$. 
\end{lem}
\begin{proof}
  Let $X$ be regular and $\mathcal{U} = \{U_{s}\}_{s \in \mathcal{S}}$ an open cover of $X$.

  For every $x \in X$, fix $U_s$ such that $x \in U_s$. Since $X$ is regular, there exists an open subset $W_x$ of $X$ such that $x \in W_x \subseteq \overline{W_x} \subseteq U_{s}$.

  Let $\mathcal{W} = \{W_x\}_{x \in X}$, then $\mathcal{W}$ is an open cover. Define $\overline{\mathcal{W}} = \{\overline{W} \; | \; W \in \mathcal{W} \}$, then $\overline{\mathcal{W}}$ is a refinement for $\mathcal{U}$. By hypothesis, we can take a locally finite refinement $\mathcal{C} = \{A_{t}\}_{t \in T}$ for $\mathcal{W}$ such that $\mathcal{C}$ covers $X$, i.e. for all $t \in T$, there exists $W \in \mathcal{W}$ such that $A_{t} \subseteq W$. Therefore, $\overline{A_{t}} \subseteq \overline{W} \subseteq U_{s(t)}$ for some $s(t) \in \mathcal{S}$.

  Now, let $F_{s} = \bigcup_{s(t) = s} \overline{A_{t}}$, since $\{A_{t}\}_{t \in T}$ is locally finite, we have
  \begin{enumerate}
  \item $\{\overline{A_{t}}\}_{t \in T}$ is locally finite, and
  \item $\bigcup_{s(t) = s}\overline{A_{s}} = \overline{\bigcup_{s(t) = s}A_{t}}$ is closed in $X$.
  \end{enumerate}

  Therefore, $\{F_{s}\}_{s \in \mathcal{S}}$ is a locally finite family of closed subsets of $X$. For $t \in T$ such that $s(t) = s$, we have $\overline{A_{t}} \subseteq U_{s(t)} = U_{s}$. Hence, $F_{s} \subseteq U_{s}$.

  Since $\mathcal{C}$ is a cover of $X$, $\{F_{s}\}_{s \in \mathcal{S}}$ covers $X$ as well. Therefore, $\{F_{s}\}_{s \in \mathcal{S}}$ is a closed locally finite refinement for $\{U_{s}\}_{s \in \mathcal{S}}$.
\end{proof}

\begin{lem} \label{lem:paracompact:4}
  If every open cover of a space $X$ has a locally finite closed refinement, then every open cover of $X$ also has a locally finite open refinement.
\end{lem}
\begin{proof}
  Let $\mathcal{U}$ be an open cover of $X$, and take a locally finite refinement $\mathcal{U}' = \{A_s\}_{s \in S}$ of $\mathcal{U}$. Then for every $x \in X$, choose an open neighbourhood $V_x$ of $x$ which intersects finitely many elements of $\mathcal{U'}$. Let $\mathcal{F}$ be a locally finite closed refinement of the open cover $\{V_x\}_{x \in X}$. For every $s \in S$, define $$ W_s = X \backslash \bigcup\{F \in \mathcal{F} \; | \; F \cap A_s = \phi\}.$$ Then the set $W_s$ is open and $A_s \subseteq W_s$. Hence for every $s \in S$ and $F \in \mathcal{F}$, we have $$W_s \cap F = \phi \; \text{if and only if} \; A_s \cap F = \phi . $$ Now, for every $s \in S$, take $U(s) \in \mathcal{U}$ such that $A_s \subseteq U(s)$. Let $V_s = W_s \cap U(s)$. Then the family $\{V_s\}_{s \in S}$ is an open refinement of the cover $\mathcal{U}$. Since for every $x \in X$, there exists an open neighbourhood which intersects only finitely many elements of $\mathcal{F}$ and each $F \in \mathcal{F}$ intersects finitely many elements of $\mathcal{U'}$, we have $\{V_s\}_{s \in S}$ is locally finite.
\end{proof}

\begin{lem} \label{lem:paracompact:5}
  Let $g: X \to I$ be an arbitrary continuous function from a space into the closed unit interval $I$ and $\{f_s\}_{s \in S}$ a partition of unity on $X$, then for any $x_{0} \in X$ such that $g(x_{0}) > 0$, there exists an open neighbourhood $U_{0}$ of $x_{0}$ and a finite set $S_{0} \subseteq S$ such that for all $x \in U_{0}$  and $s \in S \backslash S_{0}$, $f_{s}(x) < g(x)$.
\end{lem}
\begin{proof}
  Fix $S_{0} = \{s_{1}, \, s_{2}, \, \ldots \, , s_{k}\}$ such that $$1 - \sum_{i=1}^{k}f_{s_{i}}(x_{0}) < g(x_{0})$$
  Suppose $S_{0} = \phi$, then we have $1 < g(x_{0})$, contradiction. Therefore, $S_{0} \neq \phi$.

  Let $U_{0} = \{x \in X \; | \; 1 - \sum_{i = 1}^{k}f_{s_{i}}(x) < g(x) \}$, clearly, $x_{0} \in U_{0}$. If $x \in U_{0}$, then $1 - \sum_{i = 1}^{k}f_{s_{i}}(x) < g(x)$ so that $\sum_{s \in S}f_{s}(x) - \sum_{i = 1}^{k}f_{s_{i}}(x) < g(x)$ which implies $\sum_{s \in S \backslash S_{0}}f_{s}(x) < g(x)$. But $f_{s}(x) \geq 0$, hence we have for all $s \in S \backslash S_{0}$ and $x \in U_{0}$, $f_{s}(x) < g(x)$. That $U_{0}$ is open in $X$ follows from the continuity of $f_x$ for $s \in S$ and the continuity of $g$. 
\end{proof}

\begin{lem} \label{lem:paracompact:6}
  If for an open cover $\mathcal{U}$ of a space $X$, there exists a partition of unity $\{f_{s}\}_{s \in \mathcal{S}}$ subordinated to it, then $\mathcal{U}$ has an open locally finite refinement.
\end{lem}
\begin{proof}
  Let $x \in X$, there exists $s(x) \in S$ such that $f_{s(x)} > 0$. Define $g = f_{s(x)}$, then $f(x) = \sup{\{f_{s}(x)\}_{s \in S}}$ is a continuous function from $X$ to $I$. For all $s \in S$, the set $V_{s} = \{x \in X \; | \; f_{s}(x) > \frac{1}{2}f(x)\}$ is open in $X$. In addition, $V_{s} \subseteq f_{s}^{-1}(]0,1]) \subseteq U_{s}$, therefore, we have $\mathcal{V} = \{V_{s}\}_{s \in S}$ is a refinement of $\mathcal{U}$.

  Now, we show that $\mathcal{V}$ is locally finite cover.

  Let $x_{0} \in X$, consider $U_{0} = \{x \in X \; | \; 1 - \sum_{i = 1}^{k}f_{s_{i}}(x) < \frac{1}{2}f(x) \}$ which is open in $X$, where $f_{s}(x) < \frac{1}{2}f(x)$ for $x \in U_{0}$ and $s \in S \backslash S_{0}$, $S_{0} = \{s_{1}, \, \ldots \, ,s_{k}\}$ as in Lemma \ref{lem:paracompact:5}. Then $U_{0}$ intersects finitely many $V_{s}$, where $s \in S_{0}$. Hence, $\mathcal{V}$ is locally finite.
  
  Finally we show that $\mathcal{V}$ is a cover for $X$. Suppose it is not a cover, then there exists $x \in X$ such that for all  $s \in S$, $x \notin V_{s}$, i.e. for all $s \in S$ $f_{s}(x) \leq \frac{1}{2}f(x)$. But for all $s \in S \backslash S_{0}$, we have $f_{s}(x) < g(x)$. Therefore, for $s \in S_{0}$, we have $f_{s}(x) = \frac{1}{2}f(x) = \frac{1}{2} \sup{\{f_{s}(x)\}_{s \in S}}$, so that $f_{s}(x) = 0$, a contradiction. Hence, $\mathcal{V}$ is a cover.
\end{proof}

\begin{thm} \label{thm:paracompact:7}
  Let $X$ be a $T_{1}$ space. Then the following conditions are equivalent:
  \begin{enumerate}
  \item $X$ is paracompact.
  \item Every open cover of $X$ has a locally finite partition of unity subordinated to it.
  \item Every open cover of $X$ has a partition of unity subordinated to it.
  \end{enumerate}
\end{thm}
\begin{proof}

  Say $X$ is paracompact, let $\mathcal{U} = \{U_{s}\}_{s \in S}$ be a locally finite open cover of $X$. Since $X$ is paracompact, $X$ is normal, hence is regular. Therefore, there exists a locally finite closed cover $\{F_{s}\}_{s \in S}$ of $X$ which refines $\mathcal{U}$.

  Let $F_{s} \in \{F_{s}\}_{s \in S}$, then there exists $U_{s} \in \mathcal{U}$ such that $F_{s} \subseteq U_{s}$. Consider the closed subsets $F_{s}$ and $X \backslash U_{s}$ of $X$. By Urysohn lemma, there exists a continuous function $g_{s}:X \to I$ such that for all $x \in F_{s}$, $g_{s}(x) = 1$ and for all $x \in X \backslash U_{s}$, $g_{s}(x) = 0$. Let $x \in F_{s}$, since $\{F_{s}\}_{s \in S}$ and $\mathcal{U}$ are locally finite, there exists an open neighbourhood $V_{x}$ of $x$ such that $V_{x}$ intersects finitely many $F_{s}$'s and $U_s$'s. Therefore, $x$ is contained only in finitely many elements of $\{U_{s}\}_{s \in S}$. Hence, $g(x) = \sum_{s \in S} g_{s}(x) = \sum_{i = 1}^{k}g_{s_{i}}(x)$, where $k \in \mathbb{N}$. Since $g_{s}$ is continuous for all $s \in S$, we have $g(x)$ is continuous at each point $x$ of $X$. Let $f_{s}(x) = \frac{g_{s}}{g}(x)$, the continuity of $g_{s}$ and $g$ implies $f_{s}$ is continuous. Hence, we have
  $$\sum_{s \in S}f_{s}(x) = \sum_{s \in S}\frac{g_{s}(x)}{g(x)} = \frac{\sum_{s \in S}g_{s}(x)}{g(x)} = 1$$
  Therefore, $\{f_{s}\}_{s \in S}$ is a partition of unity on $X$.

  Consider $f_{s}^{-1}(\,]0, 1]) = \{x \in X \; | \; f_{s}(x) \in \, ]0, 1] \}$. Let $x \in f_{s}^{-1}(\,]0, 1])$. Suppose $x \in X \backslash U_{s}$, then $g_{s}(x) = 0$ implies $f_{s}(x) = \frac{g_{s}(x)}{g(x)} = 0$, hence we have $x \notin f_{s}^{-1}(]0, 1])$, contradicting our assumption. Therefore, $x \in U_{s}$.  Thus, for all $s \in S$, we have $f_{s}^{-1}(]0,1]) \subseteq U_{s}$. Therefore, $\{f_{s}^{-1}(\,]0,1])\}_{s \in S}$ is a refinement of $\mathcal{U}$. Hence, it is subordinated to $\mathcal{U}$.

  2) implies 3) is obvious.
  
  3) implies 1). In this case, because of Lemma \ref{lem:paracompact:6}, we only need to show that $X$ is Hausdorff. Consider $x_{1}, \; x_{2} \in X$ and $x_{1} \neq x_{2}$. The open cover $\mathcal{U} = \{X \backslash \{x_{1}\}, \; X \backslash \{x_{2}\} \}$ has a partition of unity $\{f_{s}\}_{s \in S}$ subordinated to it. Let $s_{0} \in S$ and say $x_1 \in f_{s_{0}}^{-1}(]0,1]) \subseteq X \backslash\{x_{2}\}$, then $f_{s_{0}}(x_{1}) = a > 0$. Now, consider $f_{s_{0}}(x_{2}) = 0$, and we define
  \begin{enumerate}
  \item $U_{1} = f_{s_{0}}^{-1}(\,]\frac{a}{2}, 1])$
  \item $U_{2} = f_{s_{0}}^{-1}([0, \frac{a}{2}[\,)$
  \end{enumerate}
  The intervals $] \frac{a}{2}, 1]$ and $[0, \frac{a}{2}[$ are open in $[0,1]$ in the order topology, and $f_{s_{0}}$ is continuous, thus $U_{1}$, $U_{2}$ are open and disjoint in $X$. Also we have $x_{1} \in U_{1}$ and $x_{2} \in U_{2}$. Therefore $X$ is paracompact.  
\end{proof}

A family of subsets of a space $X$ is said to be $\sigma-$locally finite ($\sigma-$discrete) if it can be represented as a countable union of locally finite (discrete) families of subsets of $X$.

\begin{thm} \label{thm:paracompact:8}
  Every open $\sigma-$locally finite cover $\mathcal{V}$ of a space $X$ has a locally finite refinement.
\end{thm}
\begin{proof}
  Let $\mathcal{V} = \bigcup_{i = 1}^{\infty}\mathcal{V}_{i}$, where $\mathcal{V}_{i} = \{V_{s}\}_{s \in S_{i}}$ is a locally finite family of open subsets of $X$ which are pairwise disjoint for every $i \in \mathbb{N}$. Then for every $s \in S_{i}$, define $A_{s} = V_{s} \backslash \bigcup_{k < i} \bigcup_{s \in S_{k}} V_{s}$.

  The family $\mathcal{C} = \{A_{s}\}_{s \in S}$ where $S = \bigcup_{i = 1}^{\infty}S_{i}$ covers $X$ and $\mathcal{C}$ is a refinement of $\mathcal{V}$.

  Now, we show that $\mathcal{C}$ is locally finite.

  Let $x \in X$, $k$ be the smallest integer such that $x \in \bigcup_{s \in S_{k}} V_{s}$, then there exists $s_{0} \in S_{k}$ such that $x \in V_{s_{0}}$. Consider $A_{s}$ with $s \in \bigcup_{i > k}S_{i}$, then $V_{s_{0}} \cap A_{s} = \phi$, i.e. $V_{s_{0}}$ is a neighbourhood of $x$ that is disjoint from all $A_s$ with $s \in \bigcup_{k < i}S_{i}$.

  Now, consider the sets $A_{s}$ with $s \in \bigcup_{i \leq k}S_{i}$. Since for all $i \in \mathbb{N}$, $\mathcal{V}_{i}$ is locally finite, we have for $i \leq k$, there exists an open neighbourhood $U_{i}$ of $x$ such that $U_{i}$ intersects only finitely many $V_{s}$ in $\mathcal{V}_{i}$.

  Consider $U = U_{1} \cap U_{2} \cap \, \cdots \, \cap U_{k} \cap V_{s_{0}}$. Then $x \in U$ implies $U \neq \phi$, $U_{i}$ is open for $i \leq k$ and $V_{s_{0}}$ is open implies $U$ is open. Then we have $U \cap A_{s} = \phi$ for $s \in \bigcup_{k < i}S_{i}$ because $U_{i}$ intersects finitely many $A_{s}$ for $s \in S_{i}$ for each $i \leq k$. Therefore, $U$ meets only finitely many members of $\mathcal{C}$.
\end{proof}

\begin{thm} \label{thm:paracompact:9}
  For every regular space $X$, the following conditions are equivalent:
  \begin{enumerate}
  \item $X$ is paracompact.
  \item Every open cover of $X$ has an open $\sigma-$locally finite refinement.
  \item Every open cover of $X$ has a locally finite refinement.
  \item Every open cover of $X$ has a closed locally finite refinement.
  \end{enumerate}
\end{thm}
\begin{proof}
  This follows from Lemma \ref{lem:paracompact:3}, Lemma \ref{lem:paracompact:4} and Theorem \ref{thm:paracompact:8}.
\end{proof}

\begin{defn}
  Let $X$ be a set, $\mathcal{C} = \{A_{s}\}_{s \in S}$ a cover of $X$. The star of $M \subseteq X$ with respect to $\mathcal{C}$ is defined as
  $$St(M,\mathcal{C}) = \bigcup\{A_{s} \; | \; M \cap A_{s} \neq \phi \}$$

  The star of a one-point set $\{x\}$ with respect to a cover $\mathcal{C}$ is called the star of the point $x$ with respect to $\mathcal{C}$, denoted by $St(x,\mathcal{C})$.
\end{defn}

\begin{defn}
  Let $X$ be a set, $\mathcal{C}_1 = \{A_{s}\}_{s \in S}$, $\mathcal{C}_2 = \{B_{t}\}_{t \in T}$ be covers of $X$, then $\mathcal{C}_2$ is a star refinement of $\mathcal{C}_1$ if for all $t \in T$, there exists $s \in S$ such that $St(B_{t},\mathcal{C}_2) \subseteq A_{s}$. If for every $x \in X$, there exists $s \in S$ such that $St(x,\mathcal{C}_2) \subseteq A_{s}$, we say that $\mathcal{C}_2$ is a barycentric refinement of $\mathcal{C}_1$.
\end{defn}

Clearly, every star refinement is a barycentric refinement and every barycentric refinement is a refinement.

\begin{lem} \label{lem:paracompact:10}
  If an open cover $\mathcal{U}$ of a space $X$ has a closed locally finite refinement, then $\mathcal{U}$ also has an open barycentric refinement. 
\end{lem}

\begin{proof}
  Let $\mathcal{U} = \{U_{s}\}_{s \in S}$ be an open cover for $X$, $\mathcal{F} = \{F_{s}\}_{t \in T}$ be a closed locally finite refinement of $\mathcal{U}$ which covers $X$. Then we have for all $t \in T$, there exists $s(t) \in S$ such that $F_{t} \subseteq U_{s(t)}$. Let $x \in X$, then there exists $t \in T$ such that $x \in F_{t}$. $\mathcal{F}$ is locally finite implies there exists an open neighbourhood $U_{x}$ of $x$ such that $U_{x}$ intersects finitely many $F_{t} \in \mathcal{F}$. Therefore, the set $T(x) = \{ t \in T \; | \; x \in F_{t} \}$ is also finite.

  Now, consider $$V_{x} = (\bigcap_{t \in T(x)} U_{s(t)}) \cap (X \backslash \bigcup_{t \notin T(x)} F_{t})$$
  then $x \in V_{x}$. Since $T(x)$ is a finite set, $\bigcap_{t \in T(x)}U_{s(t)}$ is open. $\mathcal{F}$ is locally finite implies $\{F_{t} \; | \; t \notin T(x)\} \subseteq \mathcal{F}$ is also locally finite. Therefore, $\bigcup_{t \notin T(x)} \overline{F_{t}} = \bigcup_{t \notin T(x)} F_{t} = \overline{\bigcup_{t \notin T(x)}F_{t}}$ is closed in $X$. Hence, $X \backslash \bigcup_{t \notin T(x)}F_{t}$ is open in $X$. Thus, $V_{x}$ is open in $X$.

  Now, $\mathcal{V} = \{V_{x}\}_{x \in X}$ is an open cover for $X$, and we show that $\mathcal{V}$ is an open barycentric refinement to $\mathcal{U}$.

  Let $x_{0} \in X$, consider $St(x_{0},\mathcal{V})$. Let $t_{0} \in T(x_{0})$, if $x_{0} \in V_{x}$, then $x_{0} \in X \backslash \bigcup_{t \notin T(x)}F_{t}$. By contradiction, suppose that $t_{0} \notin T(x)$, then $x_0 \in F_{t_{0}} \subseteq \bigcup_{t \notin T(x)}F_{t}$ implies $x_{0} \in \bigcup_{t \notin T(x)}F_{t}$, contradiction. Therefore, $t_{0} \in T(x)$ so that by definition, $V_{x} \subseteq U_{s(t_{0})}$. Hence, we have $St(x_{0},\mathcal{V}) \subseteq U_{s(t_{0})}$ and $\mathcal{V}$ is an open barycentric refinement of $\mathcal{U}$.
\end{proof}

\begin{remark}
  In fact, if the open cover $\mathcal{U}$ is locally finite, then its barycentric refinement $\mathcal{V}$ is also locally finite.
\end{remark}

\begin{lem} \label{lem:paracompact:11}
  If a cover $\mathcal{C}_1 = \{A_{s}\}_{s \in S}$ of a set $X$ is barycentric refinement of a cover $\mathcal{C}_2 = \{B_{t}\}_{t \in T}$ of $X$, and $\mathcal{C}_2$ is a barycentric refinement of a cover $\mathcal{C}_3 = \{C_z\}_{z \in Z}$ of $X$, then $\mathcal{C}_1$ is a star refinement of $\mathcal{C}_3$. 
\end{lem}

\begin{proof}
  Let $A_{s} \in \mathcal{C}_1$, consider $St(A_{s}, \mathcal{C}_1)$. Take $x \in St(A_{s},\mathcal{C}_1)$, then there exists $A_{s(x)} \in \mathcal{C}_1$ such that $x \in A_{s(x)}$ and $A_{s(x)} \cap A_{s} \neq \phi$. For any $x_{0} \in A_{s(x)} \cap A_{s}$, since $\mathcal{C}_1$ is a barycentric refinement of $\mathcal{C}_2$, we have there exists $B_{t(x_{0})} \in \mathcal{C}_2$ such that $A_s \subseteq St(x_{0}, \mathcal{C}_1) \subseteq B_{t(x_{0})}$. %Clearly for all $x_{0} \in A_{s}$, $A_{s} \subseteq St(x_{0}, \mathcal{C}_1) \subseteq B_{t(x_{0})}$.

  Now, let $x_{1}$ be any fixed point in $A_{s}$, then we have $\bigcup_{x_{0} \in A_{s}}B_{t(x_{0})} \subseteq St(x_{1}, \mathcal{C}_2) = \bigcup \{B_{t} \; | \; x_{1} \in B_{t} \}$. Since $\mathcal{C}_2$ is a barycentric refinement of $\mathcal{C}_3$, there exists $C_{z(x_{1})} \in \mathcal{C}_3$ such that $St(x_{1},\mathcal{C}_2) \subseteq C_{z(x_{1})}$. Then for any $A_{s(x)} \cap A_{s} \neq \phi$, there exists $x_{0} \in A_{s(x)} \cap A_{s}$ such that $A_{s(x)} \subseteq St(x_{0}, \mathcal{C}_1) \subseteq B_{t(x_{0})} \subseteq \bigcup_{x_{0} \in A_{s}}B_{t(x_{0})} \subseteq St(x_{1}, \mathcal{C}_2) \subseteq C_{z(x_{1})}$. Therefore, $St(A_{s},\mathcal{C}_1) \subseteq C_{z(x_{1})}$.
\end{proof}

\begin{lem} \label{lem:paracompact:12}
  If every open cover of a space $X$ has an open star refinement, then every open cover of $X$ has also an open $\sigma-$discrete refinement.
\end{lem}
\begin{proof}
Let $\mathcal{U} = \{U_{s}\}_{s \in S}$ be an open cover of $X$. Let $\mathcal{U}_{0} = \mathcal{U}$, and denote by $\mathcal{U}_0$, $\mathcal{U}_{1}$, $\mathcal{U}_{2}$, $\ldots$ a sequence of open covers of $X$ such that $\mathcal{U}_{i+1}$ is a star refinement of $\mathcal{U}_{i}$ for $i \in \mathbb{N}$. The sequence exists because of the hypothesis.

  For $s \in S$ and $i = 1,2,3, \ldots$, we define $$U_{s,i} = \{ x \in X \; | \; x \; \text{has an open neighbourhood V such that} \; St(V,\mathcal{U}_{i}) \subseteq U_{s}\}$$
  Step 1: $U_{s,i}$ is open in $X$.

  Let $x \in U_{s,i}$, then there exists $V$ open such that $x \in V \subseteq St(V,\mathcal{U}_{i}) \subseteq U_{s}$. Let $y \in V$, then $V$ is also a neighbourhood of $y$ and $St(V, \mathcal{U}) \subseteq U_{s}$. Therefore $y \in U_{s,i}$. Hence $V \subseteq U_{s,i}$. Thus we have $U_{s,i}$ is open in $X$.

  Now, we fix $i$ and consider the family $\{U_{s,i}\}_{s \in S}$. Then for all $s \in S$, $U_{s,i} \subseteq U_{s}$, therefore $\{U_{s,i}\}_{s \in S}$ is an open refinement of $\mathcal{U}$. In addition, for any fixed $i$, the family $\{U_{s,i}\}_{s \in \mathcal{S}}$ is an open cover for $X$. Because $\mathcal{U}_i$ is an open cover for $X$, then for any $x \in X$, there exists $U \in \mathcal{U}_i$ such that $x \in U$, then there exists $U_s \in \mathcal{U}$ such that $St(U, \mathcal{U}_i) \subseteq U_s$. Thus $x \in U_{s,i}$. 

  Step 2: if $x \in U_{s,i}$ and $y \notin U_{s,i+1}$, then there exists no $U \in \mathcal{U}_{i+1}$ such that $x,y \in U$.
  %if $U_{s,i} \cap U \neq \phi$ where $U \in \mathcal{U}_{i+1}$, then $U \subseteq U_{s, i+1}$.
  
  Suppose $U \in \mathcal{U}_{i+1}$ and $x \in U \cap U_{s, i}$, then there exists $W \in \mathcal{U}_{i}$ such that $St(U,\mathcal{U}_{i+1}) \subseteq W$. Therefore, we have $W \subseteq St(x,\mathcal{U}_{i})  \subseteq U_{s}$, but $St(U,\mathcal{U}_{i+1}) \subseteq W$ implies $St(U,\mathcal{U}_{i+1}) \subseteq U_{s}$. Therefore, $U \subseteq U_{s, i+1}$.  i.e. if  $x \in U_{s,i}$ and $y \notin U_{s,i+1}$, then there is no $U \in \mathcal{U}_{i+1}$ such that $x,y \in U$.


  Step 3:

  Take a well-ordering relation $<$ on the set $S$, define the set $$V_{s_{0},i} = U_{s_{0},i} \backslash \overline{\bigcup_{s < s_{0}}U_{s, i+1}}$$
  then the above set is open in $X$.

  Let $s_{1}$,$s_{2} \in S$ and $s_{1} \neq s_{2}$, then without loss of generality, take $s_{1} < s_{2}$. 
  Now, we show that $\{V_{s,i}\}_{s \in S}$ is discrete for a fixed $i$.

  Let $x \in V_{s_{1},i}$ and $y \in V_{s_{2},i}$, then $x \in U_{s_{1},i} \backslash \overline{\bigcup_{s < s_{1}} U_{s, i+1}}$, $y \in U_{s_{2},i} \backslash \overline{\bigcup_{s < s_{2}}U_{s, i+1}} = U_{s_{2},i} \backslash \overline{\bigcup_{s < s_{2}, s \neq s_{1}}(U_{s,i+1} \cup U_{s_{1},i+1})}$. Therefore $y \notin U_{s_{1}, i+1}$. By the conclusion of Step 2, there is no $U \in \mathcal{U}_{i+1}$ such that $x,y \in U$.

  $\mathcal{U}_{i+1}$ is an open cover for $X$ implies for all $x_{0} \in X$, there exists $U \in \mathcal{U}_{i+1}$ such that $x_{0} \in U$ and $U$ can intersect at most one $V_{s,i} \in \{V_{s,i}\}_{s \in S}$ for a fixed $i$. Therefore, $\{V_{s,i}\}_{s \in S}$ is an open discrete family for each $i = 1,2,3, \ldots \;$.

  Finally, we show that $\{V_{s,i}\}_{s \in S}$ is a cover of $X$.

  For any fixed $i$, $\{U_{s,i}\}_{s \in S}$ is an open cover of $X$, then there exists $s_{(x, i)} \in S$ such that $x \in U_{s_{(x,i)},i}$ and $s_{(x,i)}$ is the smallest element in $S$ such that $x \in U_{s,i}$. For each $i = 1,2,3, \ldots \;$, we can get a $s_{(x, i)}$. Since $S$ is well ordered, there exists a smallest element for the set $\{s_{(x,i)} \; | \; i = 1,2,3, \ldots \}$. We denote the element by $s(x)$. Then there exists an integer $k$ such that $x \in U_{s(x),k}$.

  Clearly, for $s < s(x)$ and any $i = 1,2,3, \ldots \;$, $x \notin U_{s,i}$, in particular, $x \notin \bigcup_{s < s(x)}U_{s, k+1}$.

  Now to show that $x \in V_{s(x),k}$, 
  %i.e. $U_{s(x),k} \subseteq V_{s(x),k}$. 
  we only need to show that $x \notin \overline{\bigcup_{s < s(x)}U_{s, k+1}}$.

  Let $y \in \bigcup_{s < s(x)}U_{s, k+1}$, then there exists $U_{s, k+1}$ such that $y \in U_{s, k+1}$. Since $x \notin U_{s,k+2}$, by the result from Step 2, there is no $U \in \mathcal{U}_{k+2}$ such that $x,y \in U$. Therefore, $St(x, \mathcal{U}_{k+2}) \cap \bigcup_{s<s(x)}U_{s,k+1} = \phi$. Since $St(x, \mathcal{U}_{k+2})$ is open in $X$, we have $x \notin \overline{\bigcup_{s < s(x)}U_{s,k+1}}$. Therefore, $x \in V_{s(x),k}$.
\end{proof}

\begin{thm} \label{thm:paracompact:13}
  Let $X$ be a $T_{1}$ space, then the following conditions are equivalent:
  \begin{enumerate}
  \item $X$ is paracompact.
  \item Every open cover of $X$ has an open barycentric refinement.
  \item Every open cover of $X$ has an open star refinement.
  \item $X$ is regular and every open cover of $X$ has an open $\sigma-$discrete refinement.
  \end{enumerate}
\end{thm}
\begin{proof}
  1) implies 2). Since $X$ is paracompact, we have $X$ is normal and every open cover has a locally finite refinement. Then every open cover has a closed locally finite refinement. Therefore, by Lemma \ref{lem:paracompact:10}, every open cover has an open barycentric refinement.

  2) implies 3). Let $\mathcal{U}$ be an open cover of $X$. By hypothesis, there exists $\mathcal{U}_{1}$ which is an open barycentric refinement of $\mathcal{U}$. Also, $\mathcal{U}_{1}$ has an open barycentric refinement $\mathcal{U}_{2}$. Therefore, $\mathcal{U}_{2}$ is an open star refinement by Lemma \ref{lem:paracompact:11}.

  3) implies 4). Because of Lemma \ref{lem:paracompact:12}, we only need to show that $X$ is regular.

  Let $x \in X$, $F \subseteq X$ closed in $X$ such that $x \notin F$. Define $\mathcal{U} = \{X \backslash\{x\}, X \backslash F \}$, then $\mathcal{U}$ is an open cover for $X$. Hence $\mathcal{U}$ has an open star refinement $\mathcal{U}_{1}$. Let $U \in \mathcal{U}_{1}$ such that $x \in U$, then $St(U,\mathcal{U}_{1}) \subseteq X \backslash F$. Consider $\overline{U} \cap F$. If $\overline{U} \cap F \neq \phi$, then there exists $x' \in \overline{U} \cap F$.

  Since $\mathcal{U}_{1}$ is a cover, there exists $V \in \mathcal{U}_{1}$ such that $x' \in V$, then $V \cap U \neq \phi$ and $V \cap F \neq \phi$. Therefore, $V \subseteq St(U, \mathcal{U}_{1})$, hence $St(U,\mathcal{U}_{1}) \cap F \neq \phi$, giving a contradiction. Therefore, $\overline{U} \cap F = \phi$ and $U,X \backslash \overline{U}$ are open disjoint sets of $X$ such that $x \in U$, $F \subseteq X \backslash \overline{U}$. Hence, $X$ is regular.

  4) implies 1). $X$ is regular and an open $\sigma-$discrete refinement of an open cover $\mathcal{U}$ is an open $\sigma$-locally finite refinement, hence the result follows from Theorem \ref{thm:paracompact:9}.
\end{proof}

\begin{thm}
  Paracompactness is hereditary with respect to $F_{\sigma}$ sets.
\end{thm}
\begin{proof}
  Let $X$ be a paracompact space and $M$ a subspace such that $M = \bigcup_{i = 1}^{\infty}F_{i}$, where $F_{i}$ are closed subsets of $X$ for all $i \in \mathbb{N}$. Let $\mathcal{U} = \{ U_{\alpha} \}_{\alpha \in \mathcal{A}}$ be an open cover of $M$, then there exists a collection $\mathcal{V} = \{V_{\alpha}\}_{\alpha \in \mathcal{A}}$ of open subsets of $X$ such that $U_{\alpha} = M \cap V_{\alpha}$ for every $\alpha \in \mathcal{A}$.

  Now, consider $\mathcal{C}_{i} = \mathcal{V} \cup \{X \backslash F_{i} \}$, clearly $\mathcal{C}_{i}$ is an open cover of $X$. Therefore it has a locally finite refinement $\mathcal{C}_{i}'$. Define $$\mathcal{B}_{i} = \{ M \cap U \; | \; U \in \mathcal{C}_{i}' \; and \; U \cap F_{i} \neq \phi \}.$$ Then $\mathcal{B} = \bigcup_{i = 1}^{\infty} \mathcal{B}_{i}$ is an open $\sigma -$locally finite refinement of $\mathcal{U}$. Since $X$ is normal, and hence regular, the result now follows from Theorem \ref{thm:paracompact:9}.
  %hence it has a locally finite refinement. Thus $\mathcal{U}$ has a locally finite open refinement. Therefore $M$ is paracompact.
\end{proof}

\begin{corollary}
  Closed subspaces of a paracompact space are also paracompact.
\end{corollary}

\section{Paracompactness of GO-spaces}

In this section, we are going to introduce two approaches for characterizing paracompact GO-spaces: the characterization by using $Q-$gaps and $Q-$pseudo gaps and the characterization by using \emph{stationary} sets.
\subsection{Characterization by $Q-$(pseudo) gaps}

\begin{defn}
  Let $X$ be a space,  $\mathcal{C} = \{ A_{\alpha} \subseteq X \; | \; \alpha \in \mathcal{A} \}$ a family of subsets of $X$. Then $\mathcal{C}$ is said to be point-finite if for all $x \in X$, the set $\{  A_{\alpha} \; | \; x \in A_{\alpha}  \}$ is finite.
\end{defn}

\begin{lem}\label{paracompact:go:1}
  Let $X$ be a LOTS, $\mathcal{U} = \{ \; U_i \; | \; i \in \mathbb{N} \; \}$ a countable open cover of $X$ such that for all $i \in \mathbb{N}$, $U_i \subseteq U_{i+1}$, then $\mathcal{U}$ has a locally finite refinement.
\end{lem}
\begin{proof}
  For $i \in \mathbb{N}$, decompose each $U_{i}$ into its convex components $V_{i \alpha}$ where $\alpha \in \mathcal{A}_{i}$. Define $\mathcal{V}_{i} = \{V_{i\alpha} \; | \; \alpha \in \mathcal{A}_{i} \}$ and $\mathcal{V} = \bigcup_{i \in \mathbb{N}}\mathcal{V}_{i}$.

  For each $V \in \mathcal{V}_{i}$, define $W(V) = \bigcup_{i \in \mathbb{N}} V_{i}$, where $V_1 = V$, $V_{j+1} \in \mathcal{V}_{j + i}$ and $V_{j} \subseteq V_{j + 1}$ for $j \in \mathbb{N}$. Note here that $W(V)$'s either coincide or disjoint, i.e. for $V \neq V'$ we have either $W(V) = W(V')$ or $W(V) \cap W(V') = \phi$. Consider $\{V_{i} \; | \; i \in \mathbb{N} \}$ to be an open cover of the subspace $W(V)$. Since $W(V)$ is convex in $X$, the subspace topology is still a linearly ordered topology. If it has a locally finite refinement in $W(V)$ for each $V$, then $\mathcal{V}$ , hence $\mathcal{U}$, also has a locally finite open refinement of $X$.

  Therefore, we can assume that $\mathcal{U}$ consists of open convex sets. Take $x_{0} \in X$, suppose that $[x_{0}, \rightarrow [ \, \nsubseteq U_{i}$ for any $i \in \mathbb{N}$. Take $x_{i} \in [x_{0},\rightarrow[ \, \backslash U_{i}$ such that $x_{0} < x_{1} < x_{2} < \cdots$. Then $D = \{ x_{i} \; | \; i \in \mathbb{N} \}$ is discrete cofinal in $X$. For $x_i$, there exists $U_{k_i} \in \mathcal{U}$ such that $x_i \in U_{k_i}$, let $U_{x_{i}} \subseteq U_{k_i}$ be a basis element containing $x_{i}$ and $C_{x_i}$ be a convex open set with $x_i \in C_{x_i}$ such that $C_{x_i} \cap D = \{x_{i}\}$. Define $P_{i}' = C_{x_i} \cap U_{x_i}$, then we have $P_{i}' \subseteq U_{k_i}$ and $P_{i}' \cap D = \{ x_{i}\}$.

  For $i = 0$, if $x_0$ and $x_1$ are not neighbours, then define $b_{0} =  x_{0}'$  where $x_0 < x_{0}' < x_1$ otherwise take $b_{0} = x_{1}$. Define $P_{0} = P_{0}' \cap \, ]\leftarrow, b_{0}[$.

  For $i > 0$, if $x_i$ and $x_{i + 1}$ are not neighbours in $X$, then take $x_i < b_i < x_{i+1}$, otherwise take $b_i = x_{i +1}$. If $b_{i - 1}$ and $x_i$ are not neighbours, then take $b_{i - 1} < a_i < x_i$, otherwise take $a_i = b_{i - 1}$.  
  %if there exists $x_{i - 1}''$ such that $x_{i -1 }' < x_{i - 1}'' < x_{i}$ then choose $a_{i} = x_{i - 1}''$, otherwise $a_{i} = x_{i - 1}'$. If there exists $x_{i}'$ such that $x_i < x_{i}' < x_{i+1}$, then choose $b_{i} = x_{i}'$ otherwise take $b_{i} = x_{i+1}$,
  Then define $P_i = P_{i}' \, \cap \, ]a_{i}, \rightarrow[ \, \cap \, ]\leftarrow, b_{i} [$. We can continue this process for all $i > 0$ and obtain a collection $\mathcal{P}$ of sets. It is clear that $\mathcal{P}$ is discrete. Consider $\mathcal{V} = \{\,]x_i, x_{i+1}[ \, , \; P_{i} \; | \; i \in \mathbb{N} \}$, it is obvious that $\mathcal{V}$ is an open locally finite refinement of $\mathcal{U}$ which covers $[x_{0}, \rightarrow[$ . Similarly, we can show that $]\leftarrow, x_{0}]$ is also covered by an open locally finite refinement of $\mathcal{U}$. Therefore $\mathcal{U}$ has a locally finite refinement which covers $X$.        
\end{proof}

\begin{thm}\label{paracompact:go:2}
  A GO-space $X$ is paracompact if and only if every gap of $X$ is a $Q-$gap and every pseudo-gap is a $Q-$pseudo gap.
\end{thm}
\begin{proof}
  ($\Rightarrow$) Say $X$ is paracompact, let $(A,B)$ be a gap in $X$ such that $A \neq \phi$. The family $\mathcal{U} = \{\; ]\leftarrow,a[ \; | \; a \in A \; \}$ is an open cover of $A$. Since $A$ is closed and open in $X$, $A$ is a paracompact subspace of $X$, therefore $\mathcal{U}$ has a locally finite open refinement $\mathcal{V}$. Now, we decompose each $V \in \mathcal{V}$ into its convex components. Let $\mathcal{W} = \{  W_{\alpha}  \}_{\alpha \in \mathcal{A}}$ be the open cover consisting of all of those convex components.

  Take $x \in A$, since $\mathcal{V}$ is locally finite, there exists an open neighbourhood $U_{x}$ of $x$ which intersects finitely many $V_{\beta} \in \mathcal{V}$. Therefore, there can be only finitely many elements of $\mathcal{V}$ which contain $x$, hence there are only finitely many $W_{\alpha}$ containing $x$, so that $\mathcal{W}$ is point-finite. Let $\mathcal{P}$ be a locally finite refinement of $\mathcal{W}$. Choose $x_{P} \in P$ for each $P \in \mathcal{P}$, and consider the set $S = \{ x_{P} \; | \; P \in \mathcal{P}  \}$. For any $a \in A$, there exists an open neighbourhood $U_{a}$ of $a$ such that $U_{a}$ intersects finitely many $P \in \mathcal{P}$, hence $U_{a}$ can contain at most finitely many $x_{p}$. By Hausdorffness, there exists a neighbourhood $U_a'$ of $a$ that contains at most one element of $\mathcal{P}$, so that $\mathcal{P}$ is discrete.
  %Without loss of generality, say $x_{P_1}, \, \ldots \, , x_{P_n}$ are in $U_{a}$ and $ x_{P_i} < a < x_{P_j}$, where $i,j \leq n$ are some fixed integers. Then the neighbourhood $U' = \, ]\leftarrow, x_{P_j}[ \, \cap  U_{a}  \cap \, ]x_{P_i}, \rightarrow[$ at most intersects $\mathcal{P}$ at one point only. Therefore $\mathcal{P}$ is discrete.

  Now, let $a \in A$. Choose $x_1 > a$. Suppose $x_1 \in P_1 \in \mathcal{P}$ and $P_1 \subseteq W_1 \in \mathcal{W}$. If $P_1 \cap \, ]\leftarrow,a[ \, \neq \phi$, then there exists $ p \in P_1$ such that $p < a$, therefore we have $a \in ]p,x_1[$ and since $W_1$ is convex, $]p,x_1[ \, \subseteq W_1$. Hence $a \in W_1$. Since $\mathcal{W}$ is the set of convex components of $\mathcal{V}$ which is a refinement of $\mathcal{U}$, there exists $a_1 \in A$ such that $W_1 \subseteq \, ]\leftarrow, a_1]$. Then select $x_2 \in A$ such that $x_2 > a_1$. Let $x_2 \in P_2 \in \mathcal{P}$, and $P_2 \subseteq W_2 \in \mathcal{W}$. If $P_2 \cap \, ]\leftarrow, a] \neq \phi$, then $a \in W_2$, and we can continue the process. Because $\mathcal{W}$ is point finite, there exists a number $n$ such that $P_n \cap \, ]\leftarrow, a] = \phi$, thus $x_{p_{n}} > a$. Hence the set $S = \{ \; x_{P} \; | \; P \in \mathcal{P} \; \}$ is cofinal in $A$. Similarly, we can show that $B$ has a discrete coinitial subset whenever $B \neq \phi$. Hence $(A,B)$ is a $Q-gap$.

  In a similarly way, we can show that every pseudo gap is a $Q-pseudo$ gap.

  ($\Leftarrow$) We first show that for a LOTS $X$, if every gap is a $Q-gap$, then $X$ is paracompact.

  Let $\mathcal{U}$ be an open cover of $X$, $F$ be the set of gaps which are not covered by $\mathcal{U}$. Now consider the Dedekind compactification $X^+$ of $X$, then $F$ is closed in $X^+$, hence $U = X^+ \backslash F$ is open in $X^+$. Now decompose $U$ into its convex components $C_{\alpha}$, where $\alpha \in \mathcal{A}$. Define $H_{\alpha} = C_{\alpha} \cap X$. Then clearly, $\mathcal{H} = \{ H_{\alpha} \; | \; \alpha \in \mathcal{A}  \}$ is a disjoint convex open cover of $X$. Consider $H_{\alpha}$ to be a LOTS covered by $\mathcal{U}$. Then $\mathcal{U}$ covers every gap of $H_{\alpha}$ except possible end gaps. Select a non-end point $a$ of $H_{\alpha}$. If $H_{\alpha}$ has the maximal point, then $H'_{\alpha} = \{ \; x \in H_{\alpha} \; | \; x \geq a \; \}$ is covered by finitely many elements of $\mathcal{U}$. If $(H_{\alpha}, \phi)$ is an end gap of $H_{\alpha}$, then it determines a gap in $X$, hence it determines a $Q-gap$ of $X$. Thus $H_{\alpha}$ has a discrete cofinal subset. Thus by Lemma \ref{paracompact:go:1}, $H_{\alpha}'$ is covered by a locally finite open refinement of $\mathcal{U}$. Similarly we can show that $H_{\alpha}'' = \{x \in H_{\alpha} \; | \; x \leq a \}$ is also covered by a locally finite open refinement of $\mathcal{U}$, and this is true for all $\alpha \in \mathcal{A}$. Therefore $X$ is a paracompact LOTS.

  For a GO-space $X$, we can extend it into the LOTS $X^*$ as defined in Definition \ref{X*}. In this case, every gap in $X^*$ is a $Q-gap$. So $X^*$ is paracompact, hence $X$ is also paracompact, being a closed subspace of $X^*$.
\end{proof}

\begin{defn}
  Let $X$ be a space. The set $\Delta = \{ (x,x) \; | \; x \in X \}$ is called a $G_{\delta}$ diagonal if it is a $G_{\delta}$ set in $X \times X$.
\end{defn}

\begin{lem}
  Let $X$ be a space, then it has a $G_{\delta}$ diagonal if and only if $X$ has a sequence $\{\mathcal{U}_{i}\}_{i = 1}^{\infty}$ of open covers such that $\bigcap_{i = 1}^{\infty}St(x, \mathcal{U}_{i}) = \{x\}$ for every $x \in X$.
\end{lem}
\begin{proof}
  Let $\Delta = \{ (x,x) \; | \; x \in X \}$, suppose $\Delta$ is a $G_{\delta}$ diagonal. Then there exists a sequence $\{W_{i}\}_{i = 1}^{\infty}$ of open subsets of $X \times X$ such that $\Delta = \bigcap_{i = 1}^{\infty}W_{i}$. Take $(x,x) \in W_{i}$, then there exists a basis element $U(x) \times V(x)$ of $X \times X$ such that  $U(x) \times V(x) \subseteq W_{i}$, where $U(x)$ and $V(x)$ are open neighbourhoods of $x$, then take $U_{i}(x) = U(x) \cap V(x)$, then $(x,x) \in U_{i}(x) \times U_{i}(x) \subseteq U(x) \times V(x) \subseteq W_{i}$. Now, define $ \mathcal{U}_{i} = \{ U_{i}(x) \; | \; x \in X \}$. Then for any $x \in X$, consider $A = \bigcap_{i = 1}^{\infty} St(x, \mathcal{U}_{i})$, suppose $x,y \in A$ and $x \neq y$. Then we have for all $i \in \mathbb{N}$, there exists $U_{i}' \in \mathcal{U}_{i}$ such that $x,y \in U_{i}'$. Therefore $(x,y) \in W_{i}$ for each $i \in \mathbb{N}$, hence $(x,y) \in \Delta$. Contradiction.

  Conversely, say $\{\mathcal{U}_{i}\}_{i = 1}^{\infty}$ is a sequence of open covers of $X$ satisfying the above condition. Define $W_{i} = \bigcup\{U \times U \; | \; U \in \mathcal{U}_{i} \}$. Suppose $(x,y) \in \bigcap_{i = 1}^{\infty}W_{i}$, then for each $\mathcal{U}_{i}$, there exists $U_{i} \in \mathcal{U}_{i}$ such that $x,y \in U_{i}$ which implies that $ \{x,y \} \subseteq \bigcap_{i = 1}^{\infty}St(x, \mathcal{U}_{i})$. Contradiction.
\end{proof}

\begin{lem}
  Let $X$ be a GO-space with a $G_{\delta}$ diagonal, then there exists a sequence $\{ \mathcal{U}_{i} \}_{i = 1}^{\infty}$ of open covers of $X$ such that
  \begin{enumerate}
  \item $\mathcal{U}_{i}$ consists of convex subsets of $X$ for every $i \in \mathbb{N}$.
  \item $\mathcal{U}_{i+1} \prec \mathcal{U}_{i}$ for every $i \in \mathbb{N}$.
  \item $\bigcap_{i = 1}^{\infty} St(x, \mathcal{U}_{i}) = \{ x \}$ for every $x \in X$.
  \end{enumerate}
\end{lem}

\begin{proof}
  $X$ has a $G_{\delta}$ diagonal implies there exists a sequence $\{ \mathcal{U}_{i} \}_{i = 1}^{\infty}$ of open covers of $X$ such that $$\bigcap_{i = 1}^{\infty}St(x, \mathcal{U}_{i}) = \{ x\},$$ for every $x \in X$. 

  Each element $U \in \mathcal{U}_i$ can be decomposed into its convex components and thus forming another open cover $\mathcal{U}_i' \prec \mathcal{U}_i$ consisting of convex sets. Define $\mathcal{U}_i'' = \mathcal{U}_i'$, and $\mathcal{U}_i'' = \mathcal{U}_i' \wedge \mathcal{U}_{i -1}'' = \{U \cap V \; | \; U \in \mathcal{U}_i', V \in \mathcal{U}_{i - 1}''\}$ for every $i > 1$. Evidently, $\mathcal{U}_j''$ is an open cover of $X$ consisting of convex subsets for every $j \in \mathbb{N}$ and we have $\mathcal{U}_{1}'' \succ \mathcal{U}_{2}'' \succ \mathcal{U}_{3}'' \succ \; \cdots$ .  Furthermore, $\phi \neq \bigcap_{i = 1}^{\infty}St(x, \mathcal{U}_{i}'') \subseteq \bigcap_{i = 1}^{\infty} St(x, \mathcal{U}_{i}) = \{ x \}$.
\end{proof}

\begin{thm}
  Let $X$ be a GO-space with a $G_{\delta}-diagonal$. Then $X$ is hereditarily paracompact.
\end{thm}
\begin{proof}
  Let $X$ be a GO-space with a $G_{\delta}$-diagonal. Suppose $X$ is not paracompact, then $X$ has a $non-Q$ gap or $non-Q-$pseudo gap. Say $(A,B)$ is a $non-Q$ gap, suppose $A$ has no discrete cofinal subset. Let $\{ \mathcal{U}_{n} \; | \; n \in \mathbb{N} \}$ be a sequence of open covers of $X$ consisting of convex sets such that $\mathcal{U}_{i} > \mathcal{U}_{i+1}$ for every $i \in \mathbb{N}$ and $\bigcap_{n = 1}^{\infty}St(x,\mathcal{U}_{n}) = \{x\}$ for each $x \in X$.

  Now, we show that for each n, there exists $x_{n} \in A$ such that $St(x_n,\mathcal{U}_{n}) \cap A$ is cofinal in $A$. Suppose not, by transfinite induction, we can construct an increasing sequence $\{x_{\alpha} \; | \; \alpha < \tau \}$ in $A$ such that $x_{\beta} \notin St(x_{\alpha},\mathcal{U}_{n})$ for $\alpha < \beta < \tau$ and is cofinal in $A$. Then $\{x_{\alpha}\}$ is also discrete, hence contradicting the assumption.

  Since $\{x_{n}\}$ is discrete, by our assumption, it cannot be cofinal in $A$, therefore there exists $a \in A$ such that $x_n < a$ for all $n$. Now, consider $St(a, \mathcal{U}_{n}) \cap A$, and we show that it is cofinal in $A$. Let $x \in A$ and $x > a$. $St(x_n, \mathcal{U}_{n}) \cap A$ is cofinal implies there exists $x' \in St(x_n, \mathcal{U}_{n}) \cap A$ such that $x < x'$, hence we have there exists a convex open set $U_{x_{n}} \in \mathcal{U}_{n}$ such that $x_n \in U_{x_{n}}$ and $x' \in U_{x_{n}}$. Since $x_{n} < a < x < x'$ and $U_{x_{n}}$ is convex, we have $a \in U_{x_{n}}$, which implies $x \in St(a, \mathcal{U}_{n})$. Therefore, $St(a, \mathcal{U}_{n}) \cap A$ is cofinal in $A$. Since it is also convex, we have $$ \{x \in A \; | \; x \geq a \} \subseteq St(a, \mathcal{U}_{n}) \; \; \forall n \in \mathbb{N},$$ which contradicts $\bigcap_{n =1}^{\infty}St(a,\mathcal{U}_{n}) = \{a\}$. Note that the set $\{x \in A \; | \; x \geq a \}$ cannot have only one point $a$ since $A$ has not maximal point in it. Finally, every subspace of $X$ is a GO-space with a $G_{\delta}$-diagonal and hence, is again paracompact.  
\end{proof}

\subsection{Characterization by Stationary sets}

We can also characterize paracompactness of GO-spaces in terms of \emph{stationary} sets in $\Omega$, where $\Omega$ is a limit ordinal number. In the following discussions, the topology defined on ordinal numbers is the linear order topology.

\begin{defn}
  Let $\Omega$ be a limit ordinal, $S \subseteq \Omega$, then $S$ is stationary in $\Omega$ if for each cofinal closed subset $F$ of $\Omega$, we have $F \cap S \neq \phi$.
\end{defn}

\begin{corollary}
  Let $\Omega$ be a limit ordinal with uncountable cofinality. If $S \subseteq \Omega$ contains a closed cofinal subset of $\Omega$, then $S$ is stationary.
\end{corollary}
\begin{proof}
 Let $C$ be a closed cofinal subset of $\Omega$, $S' \subseteq S$ a closed cofinal subset of $\Omega$. Since $S'$ is a cofinal subset of $\Omega$, for each $c \in C$, there exists $s \in S'$ and $c' \in C$ such that $c < s < c'$. Therefore we can construct a sequence $\{c_{n}\}_{n = 1}^{\infty}$ in $C$ and a sequence $\{s_{n}\}_{n = 1}^{\infty}$ in $S'$ such that $c_1 < s_1 < c_2 < s_2 < \; \cdots$ . Let $\alpha = \sup{\{c_{n} \; | \; n \in \mathbb{N}\}} = \sup{ \{s_{n} \; | \; n \in \mathbb{N} \}}$. Because $\Omega$ has uncountable cofinality, $\alpha \in \Omega$. Also, since both $C$ and $S'$ are closed, we have $\alpha \in C \cap S'$. Hence $S$ is stationary.  
\end{proof}

\begin{lem} \label{lem:paracompact:stationary:1}
Let $S$ be a stationary subset of an ordinal $\Omega$ which has uncountable cofinality. Let $T \subseteq S$ be a relatively closed cofinal subset of $S$. Then $T$ is also stationary in $\Omega$. In particular, the set of all non-isolated points of the subspace $S$ is stationary.
\end{lem}

\begin{proof}
  Suppose not, say $T$ is not stationary, then there exists $C \subseteq \Omega$ which is closed cofinal in $\Omega$ and $C \cap T = \phi$. Since $T$ is relatively closed in $S$, then there exists a closed subset $D$ of $\Omega$ such that $T = S \cap D$. Now $T$ is cofinal in $S$ implies $T$ is cofinal in $\Omega$ which in turn implies $D$ is cofinal in $\Omega$. Then $C \cap D$ is a closed cofinal subset of $\Omega$, but $(C \cap D) \cap S = C \cap T = \phi$, contradicting that $S$ is stationary. Therefore $T$ is stationary in $\Omega$.

  Now, suppose $S$ consists of only isolated points of $S$, then for all $s \in S$, there exists $\alpha_s < \Omega$ such that $]\alpha_s,s] \cap S = \{s\}$. Consider the set $D = \{\alpha_s \; | \; s \in S\}$, then $D$ is cofinal in $\Omega$. Let $C = \overline{D}$, then $C$ is also cofinal in $\Omega$. We now show that $C \cap S = \phi$. Suppose $s \in C \cap S$, then $s \in \overline{D}$, which implies for all $\alpha < s$, $]\alpha,s] \cap D \neq \phi$, in particular $]\alpha_s,s] \cap D \neq \phi$. Let $\alpha_{s'} \in \, ]\alpha_s,s]$. If $s < s'$, then $s \in \, ]\alpha_{s'},s']$. In this case, we have $]\alpha_{s'},s'] \cap S = \{s,s'\}$, contradiction. If $s' < s$, then $s' \in \, ]\alpha_s,s]$, again giving a contradiction.   
\end{proof}

\begin{lem} \label{lem:paracompact:stationary:2}
  Let $\Omega$ be an ordinal with uncountable cofinality and suppose $S$ is a stationary subset of $\Omega$. For each $s \in S$, let $I(s)$ be an open neighbourhood of $s$ and let $\mathcal{I} = \{I(s) \; | \; s \in S \}$. Then there exists a point $x \in S$ such that $[x, \rightarrow[ \, \subseteq St(x,\mathcal{I})$.
\end{lem}

\begin{proof}
  By contradiction, suppose it is not true. Let $\Gamma = cf(\Omega)$, then there exists a subset $S' = \{ s(\gamma) \; | \; \gamma < \Gamma \}$ of $S$ such that for $\alpha < \beta$, $s(\alpha) < s(\beta)$ and $S'$ is cofinal in $\Omega$. Take $x(0) = s(0)$, then by our assumption, there exists $y(0) \in S$ such that $x(0) < y(0)$ and $y(0) \notin St(x(0), \mathcal{I})$. Since $S'$ is cofinal in $\Omega$, there exists $\alpha_0 < \Gamma$ such that $y(0) < s(\alpha_0)$.

  We now use transfinite induction to construct a set $\{x(\alpha) \; | \; \alpha < \Gamma\}$ such that
  \begin{enumerate}
  \item if $\gamma < \delta < \Gamma$, then $x(\gamma) < y(\gamma) < x(\delta)$.
  \item if $\gamma < \Gamma$ then $y(\gamma) \notin St(x(\gamma),\mathcal{I})$.
  \item $\{x(\gamma) \; | \; \gamma < \Gamma \}$ is cofinal in $\Omega$.
  \end{enumerate}
  Let $\mu < \Gamma$, suppose that $x(\gamma)$ and $y(\gamma)$ are defined
  % (hence $y(\gamma)$ is also defined)
  for all $\gamma < \mu$. If $\mu$ is a non-limit ordinal, then there exists $\gamma' < \Gamma$ such that $\mu = \gamma' + 1$, since $y(\gamma')$ is defined, there exists $s(\alpha') \in S'$ such that $y(\gamma') < s(\alpha')$, one can assume that $\alpha' \geq \mu$, then we define $x(\mu) = s(\alpha')$. If $\mu$ is a limit ordinal, then consider $y = \sup{\{y(\gamma) \; | \; \gamma < \mu\}}$. Since $\mu < \Gamma = cf(\Omega)$, we have $y \in \Omega$, therefore, there exists $\mu \leq \alpha' < \Gamma$ such that $y < s(\alpha')$. Then we define $x(\mu) = s(\alpha')$. One can then define $y(\mu)$ in such a way to satisfy (2) and $y(\mu) > x(\mu)$ because $[x(\mu), \rightarrow[ \nsubseteq St(x(\mu),\mathcal{I})$.
  
  Let $C =  \overline{\{x(\gamma) \; | \; \gamma < \Gamma \}}^{\Omega} \cap  \overline{\{y(\gamma) \; | \; \gamma < \Gamma \}}^{\Omega}$. Now we show that $C$ is cofinal. Let $x \in \Omega$, then there exists $\gamma_1 < \Gamma$ such that $x < x(\gamma_1) < y(\gamma_1)$. Take the increasing sequences $\{x(\gamma_i)\}_{i = 1}^{\infty}$ and $\{y(\gamma_i)\}_{i = 1}^{\infty}$ for some increasing sequence $\gamma_1 < \gamma_2 < \, \ldots \, < \Gamma$. Because $cf(\Omega) > \omega_0$, we have $x' = \sup{\{x(\gamma) \; | \; \gamma < \Gamma \}} \in \Omega$ and $y' = \sup{\{y(\gamma) \; | \; \gamma < \Gamma \}} \in \Omega$. Since $x(\gamma_i) < y(\gamma_i) < x(\gamma_{i + 1})$, we have $x' = y'$. Therefore $x' \in C$ and $x < x'$. Hence $C$ is a closed cofinal subset of $\Omega$.   
  
  Suppose $s \in S \cap C$ and consider the open neighbourhood $I(s) \in \mathcal{I}$.% such that $s \in I(s)$.
  Without loss of generality, suppose $I(s)$ is convex in $\Omega$. Since $s \in \overline{\{x(\gamma) \; | \; \gamma < \Gamma \}}^{\Omega}$, there exists $x(\gamma) \in I(s)$. Suppose $y(\gamma) \in I(s)$, then $y(\gamma) \in St(x(\gamma), \mathcal{I})$. Contradiction. Therefore, we have $y(\gamma) \notin I(s)$

  If $x(\gamma) < s$, then because $I(s)$ is convex and $x(\gamma) < y(\gamma)$, we have $s < y(\gamma)$, otherwise $y(\gamma)$ is in between $x(\gamma)$ and $s$, hence it is in $I(s)$, which is a contradiction. In addition for $\alpha > \gamma$, we have $y(\alpha)$ and $x(\alpha)$ are not in $I(s)$. Now, consider $\alpha < \gamma$. If $x(\alpha) \in I(s)$, then by the convexity of $I(s)$, we have $y(\alpha) \in I(s)$, contradicting the construction of the sequences. Hence $x(\alpha) \notin I(s)$ for all $\alpha < \gamma$. Then the open neighbourhood $I = ]x(\gamma), \rightarrow[ \; \cap I(s) \cap \; ]\leftarrow, y(\gamma)[$ of $s$ does not intersect $\{x(\gamma) \; | \; \gamma < \Gamma \}$, so that $s \notin C$. Contradiction.

  If  $x(\gamma) > s$. Since $y(\gamma) \notin I(s)$, for all $ \alpha > \gamma$, $x(\alpha),y(\alpha) \notin I(s)$. Consider $\alpha < \gamma$, if $x(\alpha) \in I(s)$, then since $x(\alpha) < y(\alpha) < x(\gamma)$, we have $y(\alpha) \in I(s)$. Contradiction. Then $I = I(s) \cap \; ]\leftarrow, x(\gamma)[$ is an open neighbourhood of $s$ which does not intersect $\{x(\gamma) \; | \; \gamma < \Gamma \}$. Contradiction.

  If $x(\gamma) = s$, then for all $\alpha < \gamma$, we have $x(\alpha) \notin I(s)$. Otherwise we would have $x(\alpha), y(\alpha) \in I(s)$. Then $I(s) \cap \{y(\gamma) \; | \; \gamma < \Gamma \} = \phi$, contradicting $s \in C$.

  Therefore, $S \cap C = \phi$. But this contradicts that $S$ is a stationary set in $\Omega$.
\end{proof}

\begin{lem} \label{lem:paracompact:stationary:3}
  Let $S$ be a stationary subset of an ordinal $\Omega$ where $cf(\Omega) > \omega_{0}$. Then no (relatively) open cover of $S$ by bounded subsets of $\Omega$ can have a point-finite (relatively) open refinement.
\end{lem}

\begin{proof}
  Suppose $\mathcal{U}$ is a relatively open cover of $S$ by bounded subsets of $\Omega$ such that it has a point-finite relatively open refinement $\mathcal{V}$. Then without loss of generality, let $\mathcal{V} = \{V(i) \; | \; i \in \mathcal{I}\}$, where $V(i)$ is a bounded open subset in $\Omega$. Since $\mathcal{V}$ is point-finite at each point of $S$, we have $St(s,\mathcal{V})$ is also bounded. However, by the previous lemma, there exists $x \in S$ such that $[x,\rightarrow[ \subseteq St(x,\mathcal{V})$. Contradiction.
\end{proof}

\begin{lem} \label{lem:paracompact:stationary:4}
  If $X$ is a LOTS which is not paracompact, then there exists a closed subspace of $X$ that is homeomorphic to a stationary subset of an ordinal $\Omega$, where $\Omega = cf(\Omega) > \omega_{0}$.
\end{lem}

\begin{proof}
  Consider the Dedekind compactification $X^+$ of $X$, since $X$ is not paracompact, it has non $Q-$gaps, i.e. there exists $c \in X^+ \backslash X$ such that $c$ is not a $Q-$gap. Let $U = \{ x \in X \; | \; x < c \}$ such that $U$ has no discrete cofinal subset.

  Suppose $V = \{ x(\alpha) \in X \; | \; \alpha < \Omega, \; x(\alpha) < c \}$ is a strictly increasing cofinal subset of $U$ and $cf(\Omega) = \Omega$. We first show that $cf(\Omega) > \omega_{0}$. Suppose not, say $\Omega$ has countable cofinality, then let $\{x(n)\}_{n = 1}^{\infty}$ be a increasing countable cofinal sequence in $V$, then it is also countable cofinal in $U$, but this means it is a discrete cofinal subset of $U$, contradiction.

  Define
  \begin{align*}
    S = \{\lambda < \Omega \; | \; & \lambda \text{ is a limit ordinal} \\
     & \text{ and } \sup{\{x(\alpha) \; | \; \alpha < \lambda \}} = x(\lambda) \in X \},
  \end{align*}
  where the supremum is taken in $X^+$. We now show that $S$ is stationary in $\Omega$. Suppose not, let $C$ be a closed cofinal subset of $\Omega$ and $C \cap S = \phi$, then consider $T = \{ x(\alpha) \; | \; \alpha \in C \} \subseteq V$. For each limit ordinal $\lambda \in C$, we have $\sup{\{x(\alpha) \; | \; \alpha < \lambda \}} = c(\lambda) \in X^+ \backslash X$ and $c(\lambda) < x(\lambda)$. Since $c(\lambda)$ is a gap, then there exists $x \in X$ such that $c(\lambda) < x < x(\lambda)$. It is clear that $T$ is cofinal in $U$, now we show that it is discrete. Let $u \in U$, then we have one of the following cases:
    \begin{enumerate}
    \item there exists a non-limit ordinal $\alpha_{0}$ such that $ x(\alpha_{0}) < u < x(\alpha_{0}+1)$.
    \item $u = x(\alpha_{0})$ where $\alpha_{0}$ is a non-limit ordinal.
    \item there exists a limit ordinal $\lambda$ such that $c(\lambda) < u < x(\lambda)$.
    \item there exists a limit ordinal $\lambda$ such that $u = x(\lambda)$.
    \end{enumerate}
    In any case, it is simple to construct a open convex neighbourhood of $u$ which at most intersects with $T$ at one point only. Therefore, $T$ is discrete. Then $c$ is a $Q-$gap from below, contradicting our assumption.

    We have also just proved that $S \neq \phi$, otherwise it cannot be stationary. Therefore, for any strictly increasing transfinite sequence $\{x(\alpha) \; | \; \alpha < \Omega \}$ in $X$ whose supremum in $X^+$ is $c$, there exists a limit ordinal $\lambda < \Omega$ such that $\sup{\{x(\alpha) \; | \; \alpha < \lambda \}} \in X$, where the supremum is taken in $X^+$.

    Consider the function $f: S \to X$ with $f(\lambda) = x(\lambda)$ for all limit ordinal $\lambda$ in $S$. Then $f$ is a homeomorphism from $S$ onto $f(S)$ which is closed in $X$.
\end{proof}

\begin{thm}
  Let $X$ be a GO-space, then the following conditions are equivalent:
  \begin{enumerate}
  \item $X$ is not paracompact.
  \item For some ordinal number $\Omega$ with $cf(\Omega) = \Omega > \omega_0$, there exists a stationary subset $S$ of $\Omega$ which is homeomorphic to a closed subspace of $X$.
  \item  For some ordinal number $\Omega$ with $cf(\Omega) > \omega_0$, there exists a stationary subset $S$ of $\Omega$ which is homeomorphic to a closed subspace of $X$.
  \end{enumerate}
\end{thm}
\begin{proof}
  
  2) implies 3) is obvious.

  3) implies 1). Let $S$ be a stationary subset of $\Omega$ such that $S$ is homeomorphic to a closed subspace $F$ of $X$. Then there exists a homeomorphism $f:S \to F$ from $S$ onto $F$. If $X$ is paracompact then so is $F$, as a closed subspace of $X$. But then would be $S$, being homeomorphic to $F$, which is impossible due to Lemma \ref{lem:paracompact:stationary:3}. 
  %Let $\mathcal{U}_X$ be an cover of $F$ by sets open in $X$, then $\mathcal{U} = \{U(i) \; | \; i \in \mathcal{I} \; and \; U_i =  U(i)_x \cap F, U(i)_x \in \mathcal{U}_X \}$ is an open cover of $F$. Then $\mathcal{V} = \{V(i) = f^{-1}(U(i))\}_{i \in \mathcal{I}}$ is an open cover of $S$. By the lemma \ref{lem:paracompact:stationary:3}, $\mathcal{V}$ has no point-finite refinement, hence it has no locally-finite refinement, which implies $\mathcal{U}$ has no locally finite refinement. Therefore, $F$ is not paracompact. Because $F$ is closed in $X$, we have $X$ is not paracompact.

  1) implies 2). If $X$ is a LOTS, then we can just apply Lemma \ref{lem:paracompact:stationary:4}.

  Suppose $X$ is a GO-space but not a LOTS. Now, consider the LOTS $X^{*}$. Then it is clear that every point in $X^* \backslash X$ is isolated in $X^*$. Also, since $X$ is closed in $X^*$, $X$ is not paracompact implies $X^*$ is not paracompact. Therefore, by Lemma \ref{lem:paracompact:stationary:4}, there exists a closed subspace $F$ of $X^*$ such that $F$ is homeomorphic to a stationary set $S$ of $\Omega$ where $cf(\Omega) = \Omega > \omega_0 $. Let $h: S \to F$ be a homeomorphism from $S$ onto $F$. Let $T$ be the set of non-isolated points of the subspace $S$, then $T$ is stationary in $\Omega$ by Lemma \ref{lem:paracompact:stationary:1}. Also $h(T)$ is closed in $X^*$. Since all points of $X^* \backslash X$ is isolated, we have $h(T) \subseteq X$. Hence $h(T)$ is also closed in $X$. 
\end{proof}

\begin{corollary}
  Let $X$ be a GO-space, then the following conditions are equivalent:
  \begin{enumerate}
  \item $X$ is NOT hereditarily paracompact.
  \item There exists a subspace $C$ of $X$ such that $C$ is homeomorphic to a stationary subset of an ordinal $\Omega$ with $cf(\Omega) = \Omega > \omega_0$.
  \item There exists a subspace $C$ of $X$ such that $C$ is homeomorphic to a stationary subset of an ordinal $\Omega$ with $cf(\Omega) > \omega_0$.
  \end{enumerate}
\end{corollary}

\chapter{Connectedness}
\section{Preliminaries}
\begin{defn}\label{connectedness:1}
  Let $X$ be a space, $A, B$ be open disjoint subsets of $X$. If $X = A \cup B$, then the pair $A,B$ form a separation for $X$. $X$ is said to be connected if it has no separations.
\end{defn}
\begin{thm}\label{connectedness:2}
  For every space $X$, the following conditions are equivalent:
  \begin{enumerate}
  \item $X$ is connected.
  \item $X$ and $\phi$ are the only closed and open sets in $X$.
  \item If $X = X_1 \cup X_2$, $X_1$ and $X_2$ are separated, then one of them is empty.
  \item Every continuous function $f: X \to D$ is a constant function, where $D = \{0,1\}$ with the discrete topology.
  \end{enumerate}
\end{thm}
\begin{proof}
  1) implies 2). Suppose $U$ is a closed and open subset of $X$ such that $\phi \neq U \neq X$. Then $(U,X \backslash U)$ is a separation for $X$. Hence $X$ is disconnected, contradicting our assumption.

  2) implies 3). Suppose $X_1$ and $X_2$ are non-empty separated subsets of $X$, then $X_1 \cap \overline{X_2} = \phi = X_2 \cap \overline{X_1}$. Since $X = X_1 \cup X_2$, $X_1$ and $X_2$ are closed and open in $X$. Contradiction.

  3) implies 4). Suppose not, then there exists a continuous function $f: \, X \to D$. Let $X_1 = f^{-1}(0)$ and $X_2 = f^{-1}(1)$. Since $\{0\}$ and $\{1\}$ are closed and open in $D$, $X_1$ and $X_2$ are non-empty closed and open subsets in $X$ and $X = X_1 \cup X_2$. Therefore $X_1$ and $X_2$ are separated. Contradiction.

  4) implies 1). By contrapositive, suppose $X$ is disconnected, then let $(X_1,X_2)$ be a separation for $X$. Define a function $f: \, X \to D$ such that $f(X_1) \subseteq \{0\}$ and $f(X_2) \subseteq \{1\}$. Then $f$ is a non-constant continuous function.
\end{proof}
\begin{corollary}\label{connectedness:3}
  A space $X$ is connected if and only if it cannot be represented as the union $X_1 \cup X_2$ of two non-empty disjoint open (or closed) subsets.
\end{corollary}

\begin{thm}\label{connectedness:4}
  Let $X$ be a connected space and $Y$ a space. Let $f:  X \to Y$ be a continuous map from $X$ to $Y$. Then $f(X)$ is a connected subspace of $Y$.
\end{thm}
\begin{proof}
  Suppose not, let $U_1$ and $U_2$ be non-empty disjoint open subsets of the subspace $f(X)$ such that $f(X) = U_1 \cup U_2$, then there exist $U_1'$ and $U_2'$ open subsets of $Y$ such that $U_1 = U_1' \cap f(X)$ and $U_2 = U_2' \cap f(X)$. Then we have $X_1 = f^{-1}(U_1') = f^{-1}(U_1)$ and  $X_2 = f^{-1}(U_2') = f^{-1}(U_2)$. Therefore $X = X_1 \cup X_2$ and $X_1$,$X_2$ are disjoint open in $X$, contradicting that $X$ is connected.
\end{proof}

\begin{thm}\label{connectedness:5}
  Let $X$ be a space, then a subspace $C$ of $X$ is connected if and only if for every pair $X_1$, $X_2$ of separated subsets of $X$ such that $C = X_1 \cup X_2$, we have $X_1 = \phi$ or $X_2 = \phi$.
\end{thm}
\begin{proof}
  Suppose that $C = X_1 \cup X_2$ be a connected subspace of $X$, where $X_1,X_2$ are separated subsets of $X$, then they are also separated in $C$. Therefore we have either $X_1 = \phi$ or $X_2 = \phi$.

  Conversely, Suppose $C$ is not connected, then there exists a pair $X_1'$, $X_2'$ of non-empty disjoint closed subsets of $C$ such that $C = X_1' \cup X_2'$, they are separated in $C$, hence separated in $X$, contradicting our assumption.
\end{proof}

\begin{corollary}\label{connectedness:6}
  Let $X$ be a space and $C$ a subspace of $X$. If $C$ is connected, then for every pair $X_1$, $X_2$ of separated subsets of $X$ such that $C \subseteq X_1 \cup X_2$, we have $C \subseteq X_1$ or $C \subseteq X_2$.
\end{corollary}
\begin{proof}
  Let $X_1,X_2$ be separated subsets of $X$ such that $C \subseteq X_1 \cup X_2$, then $C \cap X_1$ and $C \cap X_2$ are also separated in $X$ and $C = C \cap (X_1 \cup X_2)$. Since $C$ is connected,  we have $C \cap X_1 = \phi$ or $C \cap X_2 = \phi$.
\end{proof}

\begin{thm}\label{connectedness:7}
  Let $\{C_{\alpha}\}_{\alpha \in \mathcal{A}}$ be a family of connected subspaces of a space $X$. If there exists $\alpha_0 \in \mathcal{A}$ such that $C_{\alpha_0}$ is not separated from any $C_{\alpha}$, then $\bigcup_{\alpha \in \mathcal{A}}C_{\alpha}$ is connected.
\end{thm}
\begin{proof}
  Let $C = \bigcup_{\alpha \in \mathcal{A}}C_{\alpha} = X_1 \cup X_2$ , where $X_1$ and $X_2$ are separated subsets of $X$. Since for every $\alpha \in \mathcal{A}$, $C_{\alpha}$ is connected, we have $C_{\alpha} \subseteq X_1$ or $C_{\alpha} \subseteq X_2$. Without loss of generality, assume $C_{\alpha_0} \subseteq X_1$. Since for every $\alpha \in \mathcal{A}$, $C_{\alpha_0}$ is not separated from $C_{\alpha}$, we have $C_{\alpha} \subseteq X_1$ for all $\alpha \in \mathcal{A}$. Therefore $C \subseteq X_1$.
\end{proof}

\begin{corollary}\label{connectedness:8}
  Let $\{C_{\alpha}\}_{\alpha \in \mathcal{A}}$ be a family of connected subspaces of a space $X$. If $\bigcap_{\alpha \in \mathcal{A}}C_{\alpha} \neq \phi$, then $\bigcup_{\alpha \in \mathcal{A}}C_{\alpha}$ is connected.
\end{corollary}

\begin{corollary}\label{connectedness:9}
  If a subspace $C$ of a space $X$ is connected, then every subspace $A$ of $X$ which satisfies $C \subseteq A \subseteq \overline{C}$ is also connected.
\end{corollary}
\begin{proof}
  Consider $A = \bigcup_{x \in A}\{x\}$. Since $C \subseteq A \subseteq \overline{C}$, $C$ is not separated from any connected subspace $\{x\}$. Then $A = C \cup (\bigcup_{x \in A}\{x\})$ is connected.
\end{proof}

\begin{corollary}\label{connectedness:10}
  If a space $X$ contains a connected dense subspace, then $X$ is connected.
\end{corollary}

\begin{corollary}\label{connectedness:11}
  If any two points of a space $X$ can be joined by a connected subspace of $X$, then $X$ is connected.
\end{corollary}
\begin{proof}
  Let $x_0$ be a fixed point of $X$, and for every $x \in X$, let $C_{x}$ be a connected subspace of $X$ joining $x_0$ and $x$. Then $\bigcap_{x \in X}C_x \neq \phi$, therefore $X = \bigcup_{x \in X} C_x$ is connected.
\end{proof}

\subsection{Zero dimension, Strongly Zero dimension, Total disconnectedness and Hereditary disconnectedness}

In this section, we are going to discuss several types of disconnected spaces. We first introduce the concepts of zero dimensional/strongly zero dimensional spaces, totally disconnected spaces and hereditarily disconnected spaces. Then we shall discuss the relations among them.

\begin{defn}\label{connectedness:12}
  A space $X$ is said to be hereditarily disconnected if any connected subspace of $X$ contains at most one point only.
\end{defn}

Let $X$ be a space, and $x \in X$. Then the quasi-component of $x$ is the intersection of all closed and open subsets of $X$ which contain $x$. The component of $x$ is the union of all connected subspaces of $X$ which contain $x$.

\begin{defn}\label{connectedness:13}
  A space $X$ is said to be totally disconnected if every quasi-component of $X$ consists of a single point only.
\end{defn}

\begin{defn}\label{connectedness:14}
  A $T_{1}$ space $X$ is said to be zero dimensional if it has a closed and open basis.
\end{defn}

\begin{corollary}\label{connectedness:15}
  Zero dimensional spaces are Tychonoff spaces.
\end{corollary}

\begin{corollary}\label{connectedness:26}
  Every zero dimensional space is totally disconnected.
\end{corollary}

\begin{defn}\label{connectedness:16}
  A space $X$ is said to be strongly zero dimensional if $X$ is a non-empty Tychonoff space and every finite functionally open cover of $X$ has a finite open pairwise disjoint refinement.
\end{defn}

\begin{thm}\label{connectedness:17}
  Every totally disconnected space $X$ is hereditarily disconnected.
\end{thm}
\begin{proof}
  Let $A$ be a subset of $X$ and $x_1,x_2 \in A$ with $x_1 \neq x_2$. Since $X$ is totally disconnected, there exists a closed and open subset $U_{x_1}$ such that $x_1 \in U_{x_1} \subseteq X \backslash \{x_2\}$. Since $A \backslash U_{x_1}$ and $A \cap U_{x_1}$ are separated and non-empty subsets of $X$, $A$ is disconnected.
\end{proof}

\begin{lem}\label{connectedness:18}
  For every pair $A,B$ of completely separated subsets of a strongly zero dimensional space $X$, there exists an open and closed set $U \subseteq X$ such that $A \subseteq U \subseteq X \backslash B$.
\end{lem}
\begin{proof}
  Since $A,B$ are completely separated, there exists a continuous function $f: X \to I$ from $X$ to the unit interval $I$ such that $f(A) \subseteq \{0\}$ and $f(B) \subseteq \{1\}$. Then $f^{-1}(]0,1])$ and $f^{-1}([0,1[)$ form a functionally open cover of $X$. Take a refinement $\mathcal{V}$ of it which consists of pairwise disjoint open sets. Then the set $U = \bigcup\{V \in \mathcal{V} \; | \; A \cap V \neq \phi \}$ is closed and open and $A \subseteq U \subseteq X \backslash B$.
\end{proof}

\begin{corollary}\label{connectedness:19}
  Every strongly zero dimensional space is zero dimensional.
\end{corollary}

\begin{lem}\label{connectedness:20}
  If for every pair $A,B$ of completely separated subsets of a space $X$, there exists a closed and open subset $U$ of $X$ such that $A \subseteq U \subseteq X \backslash B$, then every finite functionally open cover $\mathcal{U} = \{U_i \; | \; i = 1, 2, \ldots k\}$ of $X$ has a finite open refinement $\mathcal{V} = \{V_i \; | \; i = 1, 2, \ldots k\}$ such that $V_i \subseteq U_i$ and $V_i \cap V_j = \phi$ for $i \neq j$.
\end{lem}
\begin{proof}
  For $k = 1$, $U_1 = V_1 = X$. Hence the theorem is trivially true.

  Suppose the theorem holds for every $k < m$ where $m > 1$. Consider a functionally open cover $\mathcal{U}_m = \{ U_i' \; | \; i = 1, 2, \ldots m\}$. Consider $\mathcal{U}_{m -1} = \{ U_i \; | \; U_i = U_i' \in \mathcal{U}_m$ for $i < m -1$ and $U_{m -1} = U_{m - 1}' \cup U_m' \}$, then by hypothesis, there exists an open cover $\mathcal{V}_{m-1} = \{ V_i \; | \; i = 1, 2, \ldots m-1\}$ where $V_i \cap V_j = \phi$ for $i \neq j$ and $V_i \subseteq U_i$.

  Then the sets $V_{m -1} \backslash U_{m - 1}'$ and $V_{m - 1} \backslash U_{m}'$ are disjoint and functionally closed, hence they are completely separated. Therefore, there exists an open and closed subset $U$ of $X$ such that $V_{m -1} \backslash U_{m - 1}' \subseteq U$ and $U \subseteq X \backslash (V_{m - 1} \backslash U_{m}') = (X \backslash V_{m -1}) \cup U_m'$. Then we have $V_{m - 1} \backslash U \subseteq U_{m - 1}'$ and $V_{m - 1} \cap U \subseteq U_m'$.

  Now take $\mathcal{V}_m = \{ V_i' \; | \; V_i' = V_i $ for $i < m - 1$, $V_{m - 1}' = V_{m - 1} \backslash U$ and $V_m' = V_{m - 1} \cap U \}$. Clearly, $V_i'$ are open for all $i \leq m$ and $V_i' \cap V_j' = \phi$ for $i \neq j$. Since $V_{m - 1}' \cup V_{m}' = V_{m - 1}$, we have $\mathcal{V}_{m}$ is a cover for $X$. Therefore, the theorem holds for $m$. Hence by induction, it holds for all $i \in \mathbb{N}$.
\end{proof}

\begin{thm}\label{connectedness:21}
  A non-empty Tychonoff space $X$ is strongly zero dimensional if and only if for every pair $A,B$ of completely separated subsets of $X$, there exists a closed and open subset $U$ of $X$ such that $A \subseteq U \subseteq X \backslash B$.
\end{thm}
\begin{proof}
  Follows from Lemma \ref{connectedness:18} and Lemma \ref{connectedness:20}.
\end{proof}

\begin{thm}\label{connectedness:22}
  A non-empty normal space $X$ is strongly zero dimensional if and only if every finite open cover $\{U_i \; | \; i = 1,2, \ldots , k\}$ of $X$ has a finite open refinement $\{V_i \; | \; i = 1,2, \ldots , m\}$ such that $V_i \cap V_j = \phi$ for $i \neq j$.
\end{thm}
\begin{proof}
Follows from Theorem \ref{connectedness:21} and Lemma \ref{connectedness:20}.
\end{proof}


>From the above discussion, we can see that every strongly zero dimensional space is zero dimensional and zero dimensional spaces are totally disconnected and hence hereditarily disconnected. Next, we will discuss these properties in GO-spaces.
\section{Connectedness and Disconnectedness in GO-spaces}

\begin{thm}\label{connectedness:24}
  A GO-space is connected if and only if it has no jumps, pseudo gaps or gaps except for possible end gaps
\end{thm}
\begin{proof}
  Let $X$ be a GO-space, suppose $X$ has a jump or pseudo gap or non-end gap $(A,B)$, then by definition, $A,B$ form a separation for $X$. Therefore $X$ is disconnected.
  
  Conversely, say $X$ is disconnected, then it has a separation $A,B$. Then we can decompose $A$ and $B$ into their convex components $\{C_{\alpha}\}_{\alpha \in \mathcal{A}}$ and $\{C_{\alpha}\}_{\alpha \in \mathcal{B}}$, where $\mathcal{A} \cap \mathcal{B} = \phi$. Define $\mathcal{C} = \{C_{\alpha}\}_{\alpha \in \mathcal{A} \cup \mathcal{B}}$. Since $A \cap B = \phi$, we have $C_{\alpha_{1}} \cap C_{\alpha_{2}} = \phi$ for all $\alpha_{1} \neq \alpha_{2}$. Take $C_{\alpha_{0}} \in \mathcal{C}$, and let $c \in C_{\alpha_{0}}$. There exists $\alpha \in \mathcal{A} \cup \mathcal{B}$ such that $x < c$ for all $x \in C_{\alpha}$ or there exists $\alpha \in \mathcal{A} \cup \mathcal{B}$ such that $x > c$ for all $x \in C_{\alpha}$. Suppose that $x > c$ for all $x \in C_{\alpha}$, then define $U = \, ]\leftarrow, c[ \, \cup C_{\alpha_{0}}$. It is not difficult to see that $(U, X \backslash U)$ is a jump or a pseudo gap or a gap, hence $X$ is disconnected.
  %the sets $U$ and $X \backslash U$ form a separation for $X$, hence $X$ is disconnected.
\end{proof}

\begin{thm}\label{connectedness:23}
  In GO-spaces, zero dimensionality, strongly zero dimensionality, total disconnectedness and hereditary disconnectedness are equivalent.
\end{thm}
\begin{proof}
  Suppose $X$ is hereditarily disconnected and $x < y$. Now $[x,y]$ is disconnected, so that by Theorem \ref{connectedness:24}, it has a non-end gap, pseudo gap or jump $(A,B)$ of $X$. Then $A$ is a closed and open subset of $X$ such that $x \in A$ and $y \notin A$ so that $X$ is totally disconnected. 
  
  Let $X$ be a totally disconnected GO-space. Then for any $x \in X$, the quasi-component of $x$ contains $x$ only. Let $\{U_{\alpha}\}_{\alpha \in \mathcal{A}}$ be the family of all closed and open subsets of $X$ which contains $x$, then $\{x\} = \bigcap_{\alpha \in \mathcal{A}}U_{\alpha}$. Decompose each $U_{\alpha}$ into its convex components, then there exists a convex component $C_{\alpha}$ of $U_{\alpha}$ such that $x \in C_{\alpha}$. Hence, we have $\{x\} = \bigcap_{\alpha \in \mathcal{A}}C_{\alpha}$ and $C_{\alpha}$ is also closed and open in $X$. Now, let $U$ be any open neighbourhood of $x$, without loss of generality, choose $U$ to be convex. Assume that there exists a point $y_{1} \in U$, $y_{1} < x$, then there exists $C_{\alpha_{1}}$ such that for all $c \in C_{\alpha_{1}}$, we have $y_{1} < c$. Similarly, if there exists $y_{2} \in U$ and $y_{2} > x$, there exists $C_{\alpha_{2}}$ such that for all $c \in C_{\alpha_{2}}$, $c < y_{2}$. Therefore, $C = C_{\alpha_{1}} \cap C_{\alpha_2} \subseteq U$ and $C$ is also closed and open in $X$. If there does not exist such a $y_1$, then $[x, \rightarrow[$ is closed and open in $X$, so we take $C = [x, \rightarrow[ \, \cap C_{\alpha_2}$. The same argument holds for $y_2$. In case that $y_1$ and $y_2$ do not exist, we have that $C = \{x\}$ is closed and open. Therefore for all $x \in X$, $x$ has a closed and open neighbourhood basis, hence $X$ has a basis consisting of closed and open sets. Thus $X$ is zero dimensional.

  
  Now, suppose $X$ is zero dimensional. To show $X$ is also strongly zero dimensional, it is enough to show that for every closed subset $F$ of $X$ and open subset $U$ of $X$ with $F \subseteq U$, there is a closed and open set $V$ with $F \subseteq V \subseteq U$. We first show that the result is true for LOTS. Then since the GO-space $X$ can be considered to be a closed subspace of the LOTS $X^*$ defined in Definition \ref{X*}, every closed subset $F$ of $X$ is also closed in $X^*$. Thus if $F$ is closed in $X$ and $U \supseteq F$ is open in $X$, there exists a closed and open subset $V^*$ in $X^*$ such that $F \subseteq V^* \subseteq U^*$, where $U^*$ is an open subset of $X^*$ such that $U = U^* \cap X$. Then $F \subseteq V \subseteq U$, where $V = V^* \cap X$ is closed and open in $X$. So we can safely assume that $X$ is a LOTS. 

  Take $F$ to be closed, and let $U$ be an open set such that $F \subseteq U$. We decompose $U$ into their convex components $\{U_{\alpha}\}_{\alpha \in \mathcal{A}} $. We first show that for each $\alpha \in \mathcal{A}$, there exists an open and closed convex $V_{\alpha}$ such that $F \cap U_{\alpha} \subseteq V_{\alpha} \subseteq U_{\alpha}$, whenever $U_{\alpha} \cap F \neq \phi$.

  If $U_{\alpha} = \,]a,b[ \, \neq \phi$ for some $a,b \in X$, assume that $a$ has no right neighbour point and $b$ has no left neighbour point. Then there exists $c \in U_{\alpha}$ such that for all $x \in F \cap U_{\alpha}$, $x < c$, because $F$ is closed.  Similarly, there exists $c' \in U_{\alpha}$ such that for all $x \in F \cap U_{\alpha}$, $x > c'$. Since $X$ is zero dimensional, there exist open and closed convex neighbourhoods $U_c$, $U_{c'}$ of $c$ and $c'$ respectively such that $c \in U_c \subseteq U_{\alpha}$ and $c' \in U_{c'} \subseteq U_{\alpha}$. Then $V_{\alpha} = U_{c'} \cup  [c',c] \cup U_c$ is an open and closed convex subset of $U_{\alpha}$. If $U_{\alpha} = \, ]a,\rightarrow[$ and $a$ has no right neighbour point, then by using the same proof, we can show that there exists $c' \in U_{\alpha}$ such that for all $x \in F \cap U_{\alpha}$, $x > c'$. In this case, we define $V_{\alpha} = U_{c'} \cup [c',\rightarrow[$ , where $U_{c'}$ is an open and closed convex neighbourhood of $c'$. If $U_{\alpha} = \, ]\leftarrow, b[$ and $b$ has no left neighbour point, then we define  $V_{\alpha} = \,] \leftarrow, c] \cup U_c$. 

  If $a$ has a right neighbour point $a'$ or $b$ has a left neighbour point $b'$ or both, then $]a, \rightarrow[ \, = [a', \rightarrow[$ , $]\leftarrow, b[ \, = \, ]\leftarrow, b']$ and $]a,b[ \, = [a',b']$ are closed and open respectively.  Finally if $U_{\alpha}$ is of the form $A \cap \, ]a, \rightarrow[$ or $B \cap \, ]\leftarrow,b[$ , where $(A,B)$ is a gap in $X$, $a \in A$ and $b \in B$, then the result follows from the fact that $A$ and $B$ are closed and open. 

  Now, Let $V = \bigcup_{\alpha \in \mathcal{A}}V_{\alpha}$, then $V$ is clearly open. Now we show that $V$ is also closed.

  Suppose not, let $x \in \overline{V} \backslash V$. Then for any open convex neighbourhood $C_x$ of $x$, $C_x$ intersects infinitely many $V_{\alpha}$ for $\alpha \in \mathcal{A}_0 \subseteq \mathcal{A}$, otherwise we can find an open convex neighbourhood of $x$ which does not intersect V, contradicting that $x \in \overline{V}$. Because $V_{\alpha}$ are pairwise disjoint convex sets for $\alpha \in \mathcal{A}$ and $C_x$ is also convex, there exists $\alpha' \in \mathcal{A}_0$ such that $V_{\alpha'} \subseteq C_x$. However, for each $V_{\alpha}$, we have $F \cap U_{\alpha} \subseteq V_{\alpha}$. This implies $C_x \cap F \neq \phi$ for any convex open neighbourhood $C_x$ of $x$. Therefore $x \in \overline{F} = F \subseteq V$ contradicting $x \in \overline{V} \backslash V$.

  Therefore, we have for every closed subset $F$ of $X$ and open subset $U$ of $X$ with $F \subseteq U$, there exists an open and closed set $V$ with $F \subseteq V \subseteq U$. Hence, $X$ is strongly zero-dimensional. 
\end{proof}


\chapter{Lexicographic products of LOTS}
In this chapter, we will discuss connectedness, compactness, countable compactness and paracompactness in lexicographic products of LOTS. In the following discussion, we assume that any LOTS $X$ contains more than one point.
\begin{lem}
  Let $X_{\alpha}$ be LOTS and $X = \LP_{\alpha < \mu} X_{\alpha}$ the lexicographic product, where $\mu$ is a limit ordinal. If $\mu$ is not cofinal with $\omega_{0}$, then for every point $x \in X$, $x$ has no countable local base.
\end{lem}
\begin{proof}
  Let $x = (x_{\alpha})_{\alpha < \mu} \in X$. We can assume that for each $\alpha < \mu$, there exists an ordinal $\beta > \alpha$ such that $x_{\beta}$ is not the left end point of $X_{\beta}$. If such a $x_{\beta}$ does not exist, then there exists $\beta > \alpha$ such that $x_{\beta}$ is not the right end point of $X_{\beta}$, in which case we can continue the proof in the same manner. Hence we can assume that $x_{\beta}$ is not the left end point of $X_{\beta}$.

  Suppose $x$ has a countable local base. Then since $x_{\beta}$ is not the left end point of $X_{\beta}$, we can find a strictly increasing sequence $\{s(n)\}_{n = 1}^{\infty} = \{(s_{\alpha}(n))_{\alpha < \mu}\}_{n = 1}^{\infty}$ in $X$ such that $x = \lim_{n \to \infty}{s(n)}$. Now, for each $k \in \mathbb{N}$, let $\beta(k) < \mu$ be the first ordinal such that $s_{\beta(k)}(k) <_{\beta(k)} x_{\beta(k)}$. Define $\beta = \sup{\{\beta(k) \; | \; k \in \mathbb{N}\}}$. Then $\beta < \mu$ because $\mu$ is not cofinal with $\omega_{0}$. By our assumption, there exists an ordinal $\gamma > \beta$ and a point $x_{\gamma}' <_{\gamma} x_{\gamma}$. Now, we take $u = (u_{\alpha})_{\alpha < \mu} \in X$ such that $u_{\alpha} = x_{\alpha}$ for $\alpha < \gamma$, $x_{\gamma} = x_{\gamma}'$ and $u_{\alpha}$ arbitrary for $\gamma < \alpha < \mu$. Then, for all $n \in \mathbb{N}$, $s(n) < u < x$, contradicting  $x = \lim_{n \to \infty}{s(n)}$.    
\end{proof}

\section{Connectedness}

\begin{lem} \label{lp:c:1}
  Let $X_{\alpha}$ be LOTS for all $\alpha < \mu$, where $\mu$ is a limit ordinal, $X = \LP_{\alpha < \mu}X_{\alpha}$ the lexicographic product. If $X$ is connected, then we have
  \begin{enumerate}
  \item either for all $\alpha > 0$, $X_{\alpha}$ has a left end point or for all $\alpha > 0$, $X_{\alpha}$ has a right end point.
  \item for all $\alpha \geq \omega_0$, $X_{\alpha}$ has both a left end point and a right end point.
  \item for all $\alpha < \mu$, $X_{\alpha}$ has no gaps except for possible end gaps.
  \item for all $\alpha < \mu$, if for all $\beta > \alpha$ $X_{\beta}$ has both a left end point and a right end point, then $X_{\alpha}$ has no jumps.
  \item for all $\alpha < \mu$, if $X_{\beta}$ does not have a left end point for some $\beta > \alpha$, then each bounded strictly increasing sequence in $X_{\alpha}$ is finite.
  \item for all $\alpha < \mu$, if $X_{\beta}$ does not have a right end point for some $\beta > \alpha$, then each bounded strictly decreasing sequence in $X_{\alpha}$ is finite.
  \end{enumerate}
\end{lem}
\begin{proof}
  \begin{enumerate}
  \item Suppose there exist $0< \alpha, \alpha'$ such that $X_{\alpha}$ has no left end point and $X_{\alpha'}$ has no right end point, then it is not difficult to see that $X$ has no left or right end points. Choose $y_0 \in X_{0}$ such that $y_0$ is not the right end point of $X_0$, then define $A = \{x = (x_{\alpha})_{\alpha < \mu} \in X \; | \; x_0 \leq y_0\}$, then $A \neq \phi$ and $X \backslash A \neq \phi$. Therefore $(A,X \backslash A)$ is a gap for $X$. Contradiction.
  \item Without loss of generality, suppose there exists $\alpha_0 \geq \omega_0$ such that $X_{\alpha_0}$ has no left end point, i.e. it has a left end gap $c_{\alpha_0}$. Assume that $\alpha_0 \geq \omega_0$ is the first ordinal with this property. Now, consider $c = (c_{\alpha})_{\alpha < \mu}$, where for $\alpha < \omega_0$, $c_{\alpha} \in X_{\alpha}$ are not right end points in $X_{\alpha}$, for $\omega_0 \leq \alpha < \alpha_0$, $c_{\alpha}$ are the right end points in $X_{\alpha}$ and for $\alpha > \alpha_0$, $c_{\alpha} \in X_{\alpha}$. By Lemma \ref{lem:lgap:3}, $c$ is a gap. Contradiction.
  \item Suppose $X_{\alpha_0}$ has a non-end gap $c_{\alpha_0}$, then by Lemma \ref{lem:lgap:3}, $c = (c_{\alpha})_{\alpha < \mu}$, where $c_{\alpha} \in X_{\alpha}$ for $\alpha \neq \alpha_0$ is a non-end gap for $X$. Contradiction.
  \item Suppose there exists $\alpha_0$ such that for all $\beta > \alpha_0$, $X_{\beta}$ has both a left end point and a right end point and $X_{\alpha_0}$ has a jump $(A_{\alpha_0},B_{\alpha_0})$. Let $a_{\alpha_0}$ be the maximal point in $A_{\alpha_{0}}$ and $b_{\alpha_0}$ be the minimal point in $B_{\alpha_0}$. Define $x = (x_{\alpha})_{\alpha < \mu}$ such that $ x_{\alpha} = a_{\alpha} \in X_{\alpha}$ for $\alpha < \alpha_0$, $x_{\alpha_0} = a_{\alpha_{0}}$ and $x_{\beta} = r_{\beta}$, where $r_{\beta}$ are the right end points of $X_{\beta}$ for all $\beta > \alpha_0$ and $x' = (x_{\alpha})_{\alpha < \mu}$ such that $x_{\alpha} = a_{\alpha} \in X_{\alpha}$ for $\alpha < \alpha_0$, $x_{\alpha_0} = b_{\alpha_{0}}$, $x_{\beta} = l_{\beta}$ where $l_{\beta}$ are the left end points of $X_{\beta}$ for all $\beta > \alpha_0$. Then $A = \, ]\leftarrow, x]$ and $B = [x', \rightarrow[$ form a jump in $X$. Contradicting $X$ is connected.
  \item Suppose there exists $\alpha_0$ such that $X_{\beta}$ does not have a left end point for some $\beta > \alpha_0$ and a bounded strictly increasing sequence $\{s_i\}_{i = 1}^{\infty} = \{x_{\alpha_0}(i)\}_{i = 1}^{\infty}$ is not finite. Define
    
    \begin{align*}
      A = \{ x = (x_{\alpha})_{\alpha < \mu} \; | \; & \text{for } \alpha \neq \alpha_0, \ x_{\alpha} \in X_{\alpha}  \\
      &\text{and } x_{\alpha_0} \leq x_{\alpha_0}(i) \text{ for some } i \geq 1 \}
    \end{align*}
   
    Then $(A, X \backslash A)$ is a gap in $X$. Contradiction.
  \item Similar to the proof above.
  \end{enumerate}
\end{proof}

We now use Theorem \ref{connectedness:24} to find necessary and sufficient conditions for a lexicographic product of LOTS to be connected.
\begin{thm}\label{lp:thm:2}
  Let $X_{\alpha}$ be LOTS for $\alpha < \mu$, where $\mu$ is a limit ordinal. Then the lexicographic product $X = \LP_{\alpha < \mu}X_{\alpha}$ is connected if and only if one of the following collections of conditions is satisfied:
  \begin{itemize}
  \item
    \begin{enumerate}
    \item for all $ \alpha > 0$, $X_{\alpha}$ has a left end point.
    \item for all $ \alpha \geq \omega_{0}$, $X_{\alpha}$ has both a left end point and a right end point.
    \item for all $ \alpha < \mu$, $X_{\alpha}$ has no gaps, except for possible end gaps.
    \item for all $ \alpha < \mu$, if $X_{\beta}$ have both a left and right end points for any $\beta > \alpha$, then $X_{\alpha}$ has no jumps.
    \item for all $ \alpha < \mu$, if $X_{\beta}$ does not have a right end point for some $\beta > \alpha$, then each bounded strictly decreasing sequence in $X_{\alpha}$ is finite.
    \end{enumerate}
  \item
    \begin{enumerate}
    \item for all $ \alpha > 0$, $X_{\alpha}$ has a right end point.
    \item for all $ \alpha \geq \omega_{0}$, $X_{\alpha}$ has both a left end point and a right end point.
    \item for all $ \alpha < \mu$, $X_{\alpha}$ has no gaps, except for possible end gaps.
    \item for all $ \alpha < \mu$, if $X_{\beta}$ have both a left and right end points for any $\beta > \alpha$, then $X_{\alpha}$ has no jumps.
    \item for all $ \alpha < \mu$, if $X_{\beta}$ does not have a left end point for some $\beta > \alpha$, then each bounded strictly increasing sequence in $X_{\alpha}$ is finite
    \end{enumerate}
  \end{itemize}
\end{thm}
\begin{proof}
  ($\Rightarrow$) Follows from Lemma \ref{lp:c:1}.

  ($\Leftarrow$) Say $X$ satisfies the first collection of conditions. Suppose $X$ has a non-end gap $c = (c_{\alpha})_{\alpha < \mu}$, where $c_{\alpha} \in X_{\alpha}^+$, then by Lemma  \ref{lem:lgap:1}, there exists an ordinal $\alpha_0 < \mu$ such that $c_{\alpha_0}$ is a gap in $X_{\alpha_0}$. Without loss of generality, say $\alpha_0$ is the smallest ordinal such that $c_{\alpha_0}$ is a gap, i.e. $c_{\alpha} \in X_{\alpha}$ for all $\alpha < \alpha_0$. By (2) and (3) in the first group of conditions, $c_{\alpha_0}$ must be an end gap and $\alpha_0 < \omega_0$. By Lemma \ref{lem:lgap:3}, it is not hard to see that if $\alpha_0 = 0$, then $c$ is an end gap for $X$, contradicting our assumption. Hence we have $0 < \alpha_0 < \omega_0$. In addition, by (1), $c_{\alpha_0}$ cannot be a left end gap for $X_{\alpha_0}$. Therefore $c_{\alpha_0}$ is a right end gap. From Lemma \ref{lem:lgap:3}, we can conclude that either the point $(c_{\alpha})_{\alpha < \alpha_0}$ has no right neighbour point in $\LP_{\alpha < \alpha_0}X_{\alpha}$ or the LOTS $\LP_{\alpha_0 \leq \alpha < \mu}X_{\alpha}$ has no left end point. By (1), $\LP_{\alpha_0 \leq \alpha < \mu}X_{\alpha}$ must have the left end point. Therefore, we have the former case. However, in this case, there exists a bounded strictly decreasing infinite sequence $\{(x_{\alpha}(i))_{\alpha < \alpha_0}\}_{i = 1}^{\infty}$ in $\LP_{\alpha < \alpha_0}X_{\alpha}$ such that for $i \geq 1$, $(c_{\alpha})_{\alpha < \alpha_0} < (x_{\alpha}(i))_{\alpha < \alpha_0}$. Since $\alpha_0 < \omega_0$, there exists $\alpha' < \alpha_0$ such that the set $D = \{x_{\alpha'}(i) \; | \; i = 1,2,3, \ldots ,\}$ is infinite. One can assume that $\alpha'$ is the first such ordinal. If $\alpha' = 0$, then $D$ is bounded below by $c_0$, otherwise if $\alpha' \neq 0$, $D$ is bounded below by the left end point $l_{\alpha'}$ of $X_{\alpha'}$. Clearly, we can obtain a bounded strictly decreasing infinite sequence from $D$, which contradicts (5). Therefore, $X$ does not have non-end gaps.

  Now, suppose $X$ has a jump $(A,B)$, let $a$ be the maximal point in $A$ and $b$ be the minimal point in $B$. Then $a,b$ are neighbour points in $X$. By Lemma \ref{lem:ljump:1}, we have a contradiction.

  The same argument holds if $X$ satisfies the second collection of conditions.
\end{proof}

\section{Compactness and Paracompactness}
\begin{thm}\label{lp:thm:3}
  Let $X_{\alpha}$ be LOTS for all $\alpha < \mu$, where $\mu$ is a limit ordinal, $X = \LP_{\alpha < \mu} X_{\alpha}$ the lexicographic product. Then $X$ is compact if and only if $X_{\alpha}$ is compact for all $\alpha < \mu$.
\end{thm}
\begin{proof}
  By Theorem \ref{lem:lgap:1}, if $X_{\alpha}$ is compact for all $\alpha < \mu$, then $X$ also has no gaps, hence $X$ is compact.

  Now, say $X$ is compact. Suppose there exists $\alpha_{0} < \mu$ such that $X_{\alpha_{0}}$ has a gap $c_{\alpha_{0}}$. Without loss of generality, we choose $\alpha_{0}$ to be the smallest ordinal such that $X_{\alpha_{0}}$ has a gap. Define $c = (c_{\alpha})_{\alpha < \mu}$ where $c_{\alpha} \in X_{\alpha}$ for $\alpha \neq \alpha_{0}$ and $c_{\alpha_0} \in X^+ \backslash X$. If $c_{\alpha_{0}}$ is not an end gap or $\alpha_0 = 0$, then by Lemma \ref{lem:lgap:3}, $c$ is a gap in $X$. If $c_{\alpha_{0}}$ is an end gap and $\alpha_0 > 0$, without loss of generality, say it is a left end gap, then for all $\alpha < \alpha_{0}$, $X_{\alpha}$ has no gaps, hence $X_{\alpha}$ has the left end point $a_{\alpha}$ and the right end point $b_{\alpha}$. Take $c_{\alpha} = a_{\alpha}$ for $\alpha < \alpha_{0}$ and $c_{\alpha} \in X_{\alpha}$ arbitrary for all $\alpha > \alpha_0$, then clearly $c$ is a left end gap for $X$. Contradicting $X$ is compact.
\end{proof}

\begin{thm}\label{lp:thm:4}
  Let $X_1, X_2$ be LOTS, $X = X_1 \cdot X_2$ be the lexicographic product. If $X_1$ and $X_2$ are paracompact, then $X$ is paracompact.
\end{thm}
\begin{proof}
  Let $c = c_1 \cdot c_2$ be a gap in $X$, where $c_1 \in X_1^+$ and $c_2 \in X_2^+$. If $c_1$ is a gap in $X_1$, i.e. $c_1 \in X_1^+ \backslash X_1$, then it is not difficult to see that $c$ is a $Q-$gap in $X$. Suppose $c_1 = x_1 \in X_1$, if $c_2$ is a non-end gap in $X_2$, then $A_2 = \, ]\leftarrow, c_2[ \, \cap X$ has a discrete cofinal subset $A_2'$ and $B_2 = \, ]c_2, \rightarrow[ \, \cap X$ has a discrete coinitial subset $B_2'$. Then the set $A' = \{(x_1,a_2) \; | \; a_2 \in A_2' \}$ is a discrete cofinal subset of $A = \, ]\leftarrow, c[ \, \cap X$. Similarly, the set $B' = \{(x_1,b_2) \; | \; b_2 \in B_2' \}$ is a discrete coinitial subset of $B = \, ]c, \rightarrow[  \, \cap X$.

  If $c_2$ is an end gap, without loss of generality, suppose it is the left end gap of $X_2$. Then $B'$ is a discrete coinitial subset of $B$. Now, consider $A$. Since $c$ is a gap in $X$, from Lemma \ref{lem:lgap:2}, we have either $x_1$ has no left neighbour point in $X_1$ or $X_2$ has the right end gap.

  If $x_1$ has its left neighbour point $x_1'$ in $X_1$, then $X_2$ has the right end gap $c_2'$. Since $X_2$ is paracompact, $c_2'$ is a $Q-$gap, therefore $]\leftarrow, c_2'[$ has a discrete cofinal subset $S$. Then the set $\{(x_1',s) \; | \; s \in S \}$ is a discrete cofinal subset for $A$. If $x_1$ has no left neighbour point,
  %then let $S_1$ be a cofinal subset for the set $]\leftarrow, x_1[$.
  then consider the set $A' = \{ (s,b_2) \; | \; s < x_1, b_2 \in B_2' \}$, then $A'$ is a cofinal subset of $A$. Now we need to show that it is also discrete. Let $a =(a_1,a_2) \in A $, then there exists $a' = (a_1',b_2') \in A'$ such that $a < a'$ with $a_1 < a_1'$. Then since $B_2'$ is a discrete subset of $B_2$, there exists an open convex neighbourhood $I_{a_2}$ of $a_2$ such that the intersection of $I_{a_2}$ and $B_2'$ contains at most one point.

  If $a \notin A'$ then consider the open convex neighbourhood of $a$ $$I = \{ (a_1, x_2) \; | \; x_2 \in I_{a_2} \}.$$ We have $I \cap A' = \phi$. If $a \in A'$ then $I \cap A'$ contains at most one point. Therefore $c$ is a $Q-$gap in $X$. Hence $X$ is also paracompact.
\end{proof}

\begin{thm}\label{lp:thm:5}
  Let $X_{\alpha}$ be LOTS for all $\alpha < \mu$ where $\mu$ is a limit ordinal, $X = \LP_{\alpha < \mu}X_{\alpha}$ be the lexicographic product. If $X_{\alpha}$ is paracompact for all $\alpha < \mu$, then $X$ is paracompact.
\end{thm}
\begin{proof}
 The proof is similar to the one used for Theorem \ref{lp:thm:4}.   
\end{proof}
\section{Countable compactness}
\begin{thm}\label{lp:thm:6}
  Let $X_{\alpha}$ be countably compact LOTS for all $\alpha < \mu$ where $\mu$ is a limit ordinal, $X  =  \LP_{\alpha < \mu} X_{\alpha}$ be the lexicographic product. If for all $\alpha < \mu$, $X_{\alpha}$ has no end gaps, then $X$ is also countably compact.
\end{thm}
\begin{proof}
Let $c = (A,B)$ be a gap in $X$. Then by Lemma \ref{lem:lgap:1}, there exists a gap $c_{\alpha_0} = (A_{\alpha_0},B_{\alpha_0})$ in $X_{\alpha_0}$ such that $c = (c_{\alpha})_{\alpha < \mu}$, where $c_{\alpha} \in X_{\alpha}^+$ for all $\alpha < \mu$ and $\alpha_0$ is the smallest ordinal such that $c_{\alpha_0} \in X_{\alpha_0}^+ \backslash X_{\alpha_0}$. Now we have to show that the gap $c$ is not countable.

Without loss of generality, suppose that $A$ has a countable cofinal subset $A'$. Since $c_{\alpha_{0}}$ is not an end gap in $X_{\alpha_{0}}$, $A_{\alpha_{0}} \neq \phi$. Then consider a point $a  =  (a_{\alpha})_{\alpha < \mu}  \in  A$, where for all $\alpha < \alpha_{0}$, $a_{\alpha}  =  c_{\alpha}$, then it is clear that $a_{\alpha_{0}}  \in  A_{\alpha_{0}}$. Because suppose not, then we have $a_{\alpha_{0}}  \in  B_{\alpha_{0}}$, therefore $c_{\alpha_{0}} <^+_{\alpha_{0}} a_{\alpha_{0}}$, hence we have $c <^+ a$ which implies $a \in B$, contradiction. Since $A'$ is a cofinal subset of $A$, there exists $ a'  =  (a_{\alpha}')_{\alpha < \mu}  \in  A'$ such that $a < a'$. Clearly, $a_{\alpha}' = c_{\alpha} = a_{\alpha}$ for $\alpha < \alpha_0$. Hence we have $a_{\alpha_0} <_{\alpha_0} a_{\alpha_0}'$ and $a_{\alpha_{0}}' \in A_{\alpha_0}$.

  Hence the set $A'_{\alpha_{0}}  =  \{b_{\alpha_{0}} | b = (b_{\alpha})_{\alpha < \mu} \in A' \}$ is a countable cofinal subset of $A_{\alpha_{0}}$, contradicting $X_{\alpha_{0}}$ is countably compact. Therefore, $A$ has no countable cofinal subsets. Similarly, we can show that $B$ has no countable coinitial subsets.

    Hence, $c$ gives a gap in $ \LP_{\alpha < \mu} X_{\alpha}$ which is not countable. 
\end{proof}

\begin{thm}\label{lp:thm:7}
  Let $X,Y$ be countably compact LOTS, then
  \begin{enumerate}
  \item If $Y$ has no end gaps, then $X \cdot Y$ is countably compact.
  \item If $Y$ has end gaps and the following conditions are satisfied, then $X \cdot Y$ is countably compact.
    \begin{enumerate}
    \item If $Y$ has the left end gap then for all $x  \in  X$ such that $x$ has no left neighbour points, we have there exists no countable sequence $\{x_{n}\}_{n \in \mathbb{N}}$ in $X$ converging to $x$ from its left.
    \item If $Y$ has the right end gap then for all $x  \in  X$ such that $x$ has no right neighbour points, we have there exists no countable sequence $\{x_{n}\}_{n \in \mathbb{N}}$ in $X$ converging to $x$ from its right.
    \end{enumerate}
  \end{enumerate}
\end{thm}
\begin{proof}
  As noted in Remark \ref{remk:lem:lgap:2}, every gap in $X \cdot Y$ has one of the forms listed in $(1) - (6)$ of Lemma \ref{lem:lgap:2}. Consequently, suppose $c_{X},c_{Y}$ are gaps in $X$ and $Y$ respectively. If the gap $c$ in $X \cdot Y$ is of the form $(c_X, y)$ for $y \in Y$ (or equivalently $(c_X,c_Y)$), then because $X$ is countably compact, $c_X$ is not a countable gap. It is then easy to see that $(c_X,Y)$ (or $(c_X,c_Y)$) is not countable.
  
  Next consider the case when the gap is of the form $(x, c_{Y})$ and define $$A  = \, ] \leftarrow,  (x, c_{Y}) [ \cap X \cdot Y \text{ and }B  = \, ](x,c_{Y}), \rightarrow [ \cap X \cdot Y,$$
  $$A_{Y}  = \,  ] \leftarrow,  c_{Y} [ \cap Y \text{ and } B_{Y}  = \, ] c_{Y}, \rightarrow [ \cap Y.$$
  \begin{enumerate}
  \item If $c_{Y}$ is not an end gap for $Y$, then $(x, c_{Y})$ gives a gap in $X \cdot Y$. Assume that $A$ has a countable cofinal subset $A'$, and let $(x,y_{1})  \in  A$, then there exists $(x,y')  \in  A'$ such that $(x,y_{1}) < (x,y')$.

    Hence, we have the set $A''  =  \{ (x,y) \; | \; (x,y_{1}) < (x,y) \} \subseteq A'$ is also countable cofinal subset of $A$. Therefore the set $A_{Y}''  =  \{ y \; | \; (x,y)  \in A'' \}$ is a countable cofinal subset for $A_Y$, which implies the gap $c_{Y}  =  (A_{Y},B_{Y})$ is countable, contradicting $Y$ is countably compact. Similarly, $B$ has no countable coinitial subset.

    Therefore, $(x, c_{Y})$ gives a gap which is not countable.
  \item  Say $c_{Y}$ is the left end gap for $Y$. By using a similar proof as above, we can show that $B$ has no countable coinitial subsets. We now consider $A$.

    If $x$ has the left neighbour point in $X$, then  $(x, c_{Y})$ gives a gap in $X \cdot Y$ if and only if $Y$ has the right end gap.
    If $Y$ has no right end gap,  $(x, c_{Y})$ is not a gap, it is not a countable gap.
    If $Y$ has the right end gap, then since $Y$ is countably compact, its right end gap is not countable, hence $A$ has no countable cofinal subset. Therefore $(x, c_{Y})$ does not give a countable gap.

    If $x$ has no left neighbour points in $X$, then by hypothesis, there exists no countable sequence $\{x_{n}\}_{n =1}^{\infty}$ in $X$ converging to $x$ from the left, which implies that the set $] \leftarrow, x [ \, \subseteq X$ has no countable cofinal subset, therefore $A$ has no countable cofinal subset.

    Hence, $(x, c_{Y})$ gives a gap which is not countable.
    Similarly, we can show that if $c_{Y}$ is the right end gap for $Y$, $(x, c_{Y})$ does not give a countable gap in $X \cdot Y$.
  \end{enumerate}
 
 
\end{proof}

Now, we shall give a counter example which does not satisfy the condition in Lemma \ref{lp:thm:7} and $X \cdot Y$ is not countably compact.
\begin{eg}
  Let $X_{1}  = \, ] 0, \omega_{1} [$, $X_{2}  = \, ] \omega_{1}, 0 ]$ and the lexicographic product $X  =  X_{1} \cdot X_{2}$. Then we show that $X$ is not countably compact.
\end{eg}
\begin{proof}
  Clearly, $X_{1}$ and $X_{2}$ are countably compact, however $X$ is not, because the gap $(\omega_{0}, \omega_{1})$ is a countable gap.

  The proof is easy, take $A  = \, ] \leftarrow, (\omega_{0}, \omega_{1}) [$, then the set $\{(n,0)  \in  X \; | \; n \, \in \, \mathbb{N} \}$ is a countable cofinal subset of $A$. Hence the gap  $(\omega_{0}, \omega_{1})$ is a countable gap.
  
\end{proof}

One may want to extend the theorem \ref{lp:thm:7} to a product of a limit ordinal number $\mu$ of LOTS, however, it cannot be done. Here we give a counterexample.
\begin{eg}
  Let $X = \LP_{\alpha < \mu}X_{\alpha}$ be a LOTS where $\mu$ is a limit ordinal greater than $\omega_0$, $X_{\omega_0} = \, ]\omega_1,0]$ with the inverse order and $X_{\alpha} = \{0,1\}$ with $0 < 1$ for $\alpha \neq \omega_0$. Then $X_{\alpha}$ is compact (hence countably compact) for all $\alpha \neq \omega_0$ and $X_{\omega_0}$ is countably compact. Now consider the gap $c = (c_{\alpha})_{\alpha < \mu}$ in $X$ where $c_{\alpha} = 1$ for $\alpha \neq \omega_0$ and $c_{\omega_0} = \omega_1$. It is clearly a countable gap in $X$, therefore, $X$ is not countably compact.  
\end{eg}

Here is another more complicated example. 
\begin{eg}
 Let $X = \LP_{\alpha < \mu} X_{\alpha}$ where $X_{\alpha} = \{0,1\}$ for $\alpha < \omega_{\omega_{0}}$, $X_{\omega_{\omega_{0}}} = \, ]\omega_{1}, 0]$ and $X_{\alpha} = [0, \omega_{1}]$ for $\alpha > \omega_{\omega_{0}}$. Then $X_{\alpha}$ is compact for $\alpha \neq \omega_{\omega_{0}}$ and $X_{\omega_{\omega_{0}}}$ is countably compact. Consider the point $c = (c_{\alpha})_{\alpha < \mu} \in X$ where $c_{\omega_{\omega_{0}}} = \omega_{1}$, $c_{\alpha} = 1$ for $\alpha < \omega_{\omega_{0}}$ and $c_{\alpha} = x_{\alpha} \in X_{\alpha}$ for $\alpha > \omega_{\omega_{0}}$. Then $c$ is a gap in $X$, and there is no countable sequence converging to $c$ from the right in $X$. However, consider the sequence $\{s^{n} = (s^{n}_{\alpha})_{\alpha < \omega_{\omega_0}} \}_{n = 1}^{\infty}$ in  $\LP_{\alpha < \omega_{\omega_0}} X_{\alpha}$, such that $s_{\omega_i}^{n} = 1$ for $i \leq n$ and $s_{\omega_i}^{n} = 0$ for $i > n$, then the sequence is a countable sequence in $\LP_{\alpha < \omega_{\omega_0}} X_{\alpha}$ converging to the point $(c_{\alpha})_{\alpha < \omega_{\omega_0}}$. Hence we can obtain a countable sequence in $X$ covering to $c$. Therefore $X$ is not countably compact. 
\end{eg}

\nocite{*}
\bibliographystyle{amsplain}
\bibliography{general_topology}
\end{document}
